\documentclass[10pt,aspectratio=169]{beamer}

% All the boilerplate is in ccaslides.sty
% Note that this also pulls in a custom vogtwidebar.sty
\usepackage{ccaslides}

\author{Ji\v{r}\'i Lebl}

\institute[OSU]{%
Departemento pri Matematiko de Oklahoma {\^S}tata Universitato}

\title{Cultivating Complex Analysis:\\%
Simply connected domains (4.3 part 2)}

\date{}

\begin{document}

\begin{frame}
\titlepage
\end{frame}

\begin{frame}

There is a whole list of conditions equivalent
to being simply connected for domains in $\C$.

\pause

\begin{proposition}
Let $U \subset \C$ be a domain.  The following are equivalent:
\begin{enumerate}[(i)]
\item \label{thm:simplyconnected:i}
$U$ is simply connected (in the homology sense).
\pause
\item \label{thm:simplyconnected:ii}
Every holomorphic $f \colon U \to \C$ has a primitive.
\pause
\item \label{thm:simplyconnected:iii}
Every nowhere zero holomorphic $f \colon U \to \C$ there exists
a holomorphic $g \colon U \to \C$ such that $e^{g(z)} = f(z)$.
\pause
\item \label{thm:simplyconnected:iv}
$\dfrac{1}{z-p}$ has a primitive in $U$ for every $p \in \C \setminus U$.
\pause
\item \label{thm:simplyconnected:v}
For every holomorphic $f \colon U \to \C$ and every
cycle $\Gamma$ in $U$, \quad
$\displaystyle
\int_\Gamma f(z) \, dz = 0$.
\pause
\item \label{thm:simplyconnected:vi}
For every $p \in \C \setminus U$ and every
cycle $\Gamma$ in $U$, \quad
$\displaystyle
\int_\Gamma \frac{1}{z-p} \, dz = 0$.
\end{enumerate}
\end{proposition}

\end{frame}

\begin{frame}[fragile]

\textbf{Proof:}
We just proved
\eqref{thm:simplyconnected:i} (simply connected) $\Rightarrow$
\eqref{thm:simplyconnected:ii} (existence of primitives),
and

\pause
then we proved \eqref{thm:simplyconnected:ii} $\Rightarrow$ \eqref{thm:simplyconnected:iii}
(existence of logs).

\medskip
\pause

Suppose \eqref{thm:simplyconnected:iii}. \pause

For $p \in \C \setminus U$, find a $g$ such that
$e^{g(z)} = z-p$.\pause
\[
1 = 
\frac{d}{dz} \left[
z-p
\right]
\pause
=
\frac{d}{dz} \left[
e^{g(z)}
\right]
\pause
=
e^{g(z)} g'(z)
\pause
=
(z-p) g'(z) .
\]
\pause
So \eqref{thm:simplyconnected:iv} holds ($\frac{1}{z-p}$ has a primitive).

%\medskip
\pause

By Cauchy's theorem for derivatives
\eqref{thm:simplyconnected:iv} implies
$\displaystyle
\int_\Gamma \frac{1}{z-p} \, dz = 0 
\ \  \forall p \in \C \setminus U$,
\pause \ \ so \eqref{thm:simplyconnected:vi} holds.

%\medskip
\pause

As 
$\displaystyle
n(\Gamma;p) = 
\frac{1}{2\pi i}
\int_\Gamma \frac{1}{z-p} \, dz$,
\eqref{thm:simplyconnected:vi} is a restatement of
\eqref{thm:simplyconnected:i}.

\medskip
\pause

By Cauchy's theorem for derivatives,
\eqref{thm:simplyconnected:ii} $\Rightarrow$
\eqref{thm:simplyconnected:v} (Cauchy in $U$).

\medskip
\pause

\eqref{thm:simplyconnected:v} $\Rightarrow$ \eqref{thm:simplyconnected:vi} is immediate.

\pause
\medskip

We proved
$
\begin{tikzcd}[cramped, row sep=small]
& \text{\eqref{thm:simplyconnected:ii}} \arrow[d, Rightarrow] \arrow[r, Rightarrow] &
\text{\eqref{thm:simplyconnected:iii}} \arrow[d, Rightarrow] \\
\text{\eqref{thm:simplyconnected:i}} \arrow[ur, Rightarrow] & 
\text{\eqref{thm:simplyconnected:v}} \arrow[d, Rightarrow] &
\text{\eqref{thm:simplyconnected:iv}} \arrow[dl, Rightarrow] \\
& \text{\eqref{thm:simplyconnected:vi}} \arrow[ul, Leftrightarrow] &
\end{tikzcd}
$
\qed

\end{frame}

\begin{frame}

\textbf{Remark:} The existence of roots can also be on the list, but it will
be easier to prove later.

\medskip
\pause

Another criterion that can be put on the list is the following

(although we'll only prove one direction right now).

\pause

\begin{proposition}
Let $U \subset \C$ be a domain.  If
$\C_\infty \setminus U$ is connected, then $U$ is simply connected.
\end{proposition}

\pause
\medskip

\textbf{Proof:}
Take $S = \C_\infty \setminus U$ and let $\Gamma$ be a cycle in $U$.

\medskip
\pause

$\varphi(z)= n(\Gamma;z)$ is continuous on $\C \setminus \Gamma$
\pause
\quad $\Rightarrow$ \quad $\varphi$ is continuous on $S \setminus \{ \infty \}$.

\medskip
\pause

On the unbounded component of $\C \setminus \Gamma$ we have $\varphi=0$,
so $\varphi=0$ in a neighborhood of $\infty$.

\medskip
\pause

Set $\varphi(\infty) = 0$ to make a continuous function on $\C_\infty
\setminus \Gamma$ (hence on $S$).

\medskip
\pause

$S$ is contained in a single component of 
$\C_\infty \setminus \Gamma$ ($S$ is connected)
\pause \quad $\Rightarrow$ \quad so $\varphi|_S$ is
constant.

\medskip
\pause

$\varphi(\infty) = 0$, $\infty \in S$ \quad $\Rightarrow$ \quad $\varphi|_S \equiv 0$
\pause
\quad $\Rightarrow$ \quad
$U$ is simply connected.
\qed

\medskip
\pause

\textbf{Remark:}
It is important to use $\C_\infty$ and not $\C$:
\pause
If $U = \C \setminus \{ 0 \}$, then
$\C \setminus U = \{ 0 \}$ is connected, but 
$\C_\infty \setminus U = \{ 0, \infty \}$ is not connected.
\end{frame}


\begin{frame}

\textbf{Exercise:}
Let $U_1,U_2 \subset \C$ be two simply connected domains.

\medskip
\pause

1) $U_1 \cup U_2$ is not necessarily simply connected.

\medskip
\pause

2) If
$U_1 \cap U_2$ is nonempty and connected, then $U_1 \cup U_2$ is simply connected.


\end{frame}

\end{document}
