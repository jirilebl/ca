\documentclass[10pt,aspectratio=169]{beamer}

% All the boilerplate is in ccaslides.sty
% Note that this also pulls in a custom vogtwidebar.sty
\usepackage{ccaslides}

\author{Ji\v{r}\'i Lebl}

\institute[OSU]{%
Departemento pri Matematiko de Oklahoma {\^S}tata Universitato}

\title{Cultivating Complex Analysis:\\%
Derivative is holomorphic and Morera (3.3.2)}

\date{}

\begin{document}

\begin{frame}
\titlepage
\end{frame}

\begin{frame}
Last time we proved holomorphic functions are analytic.

\medskip
\pause

Let us restate a theorem that we proved
for analytic functions for holomorphic functions.

\pause

\begin{theorem}
Let $U \subset \C$ be open and $f \colon U \to \C$ holomorphic.  Then
$f$ is infinitely complex differentiable.

\pause
In particular, $f'$ is holomorphic.
\end{theorem}

\pause

Nothing like this is true for real differentiable functions.

\medskip
\pause

Any continuous $g \colon (a,b) \to \R$ is the derivative of a real differentiable function

E.g., $f(x) = \int_c^x g(t)\,dt$ for $c \in (a,b)$.

\medskip
\pause

Even worse, the real derivative could even be discontinuous.

\end{frame}

\begin{frame}
In complex analysis, we can differentiate by integrating.

\pause

\begin{theorem}[Cauchy integral formula for derivatives]
Suppose $U \subset \C$ is open, $f \colon U \to \C$ is holomorphic,
$\overline{\Delta_r(p)} \subset U$.
\pause
Then for all $k \in \N$,
\[
f^{(k)}(z)
=
\frac{k!}{2\pi i}
\int_{\partial \Delta_r(p)}
\frac{f(\zeta)}{(\zeta-z)^{k+1}}
\,
d \zeta
\qquad
\text{for all } z \in \Delta_r(p) .
\]
\end{theorem}

\pause

\textbf{Proof:}
All complex derivatives exist.

\pause

We can compute them by
the Wirtinger operator
$\frac{\partial}{\partial z} =
\frac{1}{2}
\left(
\frac{\partial}{\partial x} - i
\frac{\partial}{\partial y}
\right)$,
(where $z=x+iy$).
\pause

Suppose conclusion holds for some $k$ (Cauchy formula is $k=0$) and fix some $z \in \Delta_r(p)$.

\medskip
\pause

$\displaystyle
\quad
f^{(k+1)}(z)
\pause
=
\frac{\partial }{\partial z}
\bigl[ f^{(k)}(z) \bigr]
\pause
=
\frac{\partial }{\partial z}
\left[
\frac{k!}{2\pi i}
\int_{\partial \Delta_r(p)}
\frac{f(\zeta)}{(\zeta-z)^{k+1}}
\,
d \zeta
\right]
$

\medskip
\pause

$\displaystyle
\qquad \qquad \,\,\,\,
= 
\frac{k!}{2\pi i}
\int_{\partial \Delta_r(p)}
f(\zeta)
\,
\frac{\partial }{\partial z}
\left[
\frac{1}{(\zeta-z)^{k+1}}
\right]
\,
d \zeta
\pause
= 
\frac{(k+1)!}{2\pi i}
\int_{\partial \Delta_r(p)}
\frac{f(\zeta)}{(\zeta-z)^{k+2}}
\,
d \zeta .
$

\pause
\medskip

We passed $x$ and $y$ derivatives under the integral sign (Leibniz rule),
\pause
which is valid as

the $x$ and $y$ derivatives of
$\frac{f(\zeta)}{(\zeta-z)^{k+1}}$ are continuous
functions of $(z,\zeta) \in \Delta_r(p) \times \partial \Delta_r(p)$.
\qed
\end{frame}

\begin{frame}

As an aside we mention a result that will be needed later.

\medskip
\pause

\textbf{Exercise:}
Suppose $f(z,t)$ is a continuous function of $(z,t) \in U \times (a,b)$,
where $U \subset \C$ is open, and for every fixed $t \in (a,b)$, the function
$z \mapsto f(z,t)$ is holomorphic.

\pause

Prove that
$\frac{\partial f}{\partial z}$ is a continuous function of $U \times
(a,b)$.

\pause

Then show
$\frac{\partial f}{\partial x}$
and
$\frac{\partial f}{\partial y}$ are continuous.

\medskip
\pause

The above is not true for real differentiable functions:

\medskip
\pause

Let $f(x,t) = t \sin(\nicefrac{x}{t})$ for $t \not= 0$ and $f(x,0) =  0$.

\medskip
\pause

Then (exercise), $f$ is continuous on $\R^2$, and for each $t$
$x \mapsto f(x,t)$ is differentiable.

\medskip
\pause

But $\frac{\partial f}{\partial x}$ is not continuous as a function of both
$x$ and $t$.
\end{frame}

\begin{frame}
That $f'$ is holomorphic gives us a very useful converse to Cauchy.

\pause

\begin{theorem}[Morera]
Let $U \subset \C$ be open and $f \colon U \to \C$ continuous.
Suppose that
\[
\int_{\partial T} f(z) \, dz = 0
\]
for every triangle such that $T \subset U$.
\pause
Then $f$ is holomorphic.
\end{theorem}

\pause

It is far easier to integrate a continuous $f$ than to show that $f'$ exists.

\medskip
\pause

\textbf{Proof:}
Holomorphicity is local, so assume $U$ is a disc.

\medskip
\pause

A disc is star-like, and the hypothesis is precisely what we used
to show that $f$ has a primitive $F$ in a star-like $U$.

\medskip
\pause

$f = F'$ in $U$, and complex derivatives are holomorphic.
\qed

\end{frame}

\begin{frame}
The reduction to a disc is necessary:

E.g., $\nicefrac{1}{z}$
does not have a primitive in $U = \C \setminus \{ 0 \}$,

but does
satisfy hypotheses of Morera.

\medskip
\pause

Typical application of Morera is something like the following exercise:

\medskip
\pause

\textbf{Exercise:}
Show that if $f \colon \C \to \C$ is continuous and holomorphic on $\C
\setminus \R$, then $f$ is holomorphic everywhere.
\end{frame}

\end{document}
