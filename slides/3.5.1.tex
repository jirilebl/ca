\documentclass[10pt,aspectratio=169]{beamer}

% All the boilerplate is in ccaslides.sty
% Note that this also pulls in a custom vogtwidebar.sty
\usepackage{ccaslides}

\author{Ji\v{r}\'i Lebl}

\institute[OSU]{%
Departemento pri Matematiko de Oklahoma {\^S}tata Universitato}

\title{Cultivating Complex Analysis:\\%
Schwarz's lemma (3.5.1)}

\date{}

\begin{document}

\begin{frame}
\titlepage
\end{frame}

\begin{frame}
\begin{lemma}[Schwarz's lemma]
Suppose $f \colon \D \to \D$ is holomorphic and $f(0) = 0$,
then 
\pause
\begin{enumerate}[(i)]
\item $\sabs{f(z)} \leq \sabs{z}$ for all $z \in \D$, and
\pause
\item $\sabs{f'(0)} \leq 1$.
\end{enumerate}
\pause
Furthermore, if $\sabs{f(z_0)} = \sabs{z_0}$ for some $z_0 \in \D \setminus
\{ 0 \}$
or $\sabs{f'(0)} = 1$, then
there is a $\theta \in \R$ such that $f(z) =
e^{i\theta} z$ for all $z \in \D$.
\end{lemma}

\pause

This may sound very specialized, but

\pause
\medskip

a) A disc is a basic neighborhood and any disc can be translated and rescaled
into $\D$. \pause

The lemma is telling us about local behavior of a holomorphic function.

\pause
\medskip

b) We will see later that every domain ``without holes'' is biholomorphic to a
disc,

so it tells us about global behavior as well.

\end{frame}

\begin{frame}
\textbf{Proof:}
As $f(0) = 0$, we can expand is
\[
f(z) = \sum_{n=1}^\infty c_n z^n \pause = z \sum_{n=1}^\infty c_n z^{n-1}
\pause = z g(z) \qquad \text{($g(z)$ a holomorphic function on $\D$.)}
\]
\pause
For $z \in \partial \Delta_r(0)$, $r < 1$,
\[
\sabs{g(z)} = \frac{\sabs{f(z)}}{\sabs{z}} \pause \leq \frac{1}{r} .
\]
\pause
By the maximum modulus principle
$\sabs{g(z)} \leq \dfrac{1}{r}$ for all $z \in \Delta_r(0)$.

\pause
\medskip

Fix any $z \in \D$ and take limit as $r \uparrow 1$ to find
$\sabs{g(z)} \leq 1$.
\pause
So
\[
\sabs{f(z)} \leq \sabs{z} \quad \text{ for all } z \in \D,
\pause \qquad \text{and} \qquad
\abs{f'(0)}
=
\abs{\lim_{z \to 0} \frac{f(z)}{z}} = \sabs{g(0)} \leq 1 .
\]
\pause
If $\sabs{f(z_0)} = \sabs{z_0}$ for some $z_0 \in \D \setminus \{ 0 \}$
\pause $\Rightarrow$
$g$ attains a maximum in $\D$
\pause $\Rightarrow$
$g$ is constant

\pause $\Rightarrow$ $f(z) = e^{i \theta} z$.

\medskip
\pause

As $g(0) = f'(0)$, the same conclusion holds if $\sabs{f'(0)} = 1$.
\qed
\end{frame}

\begin{frame}
Consider the statement for $f(z) = z^n$ for $n > 1$.

\pause
\medskip

$f$ takes $\D$ to $\D$ and $f(0) = 0$.

\pause
\medskip

For $z \in \D \setminus \{ 0 \}$,
\[
\sabs{z^n} =
\sabs{z}^n < \sabs{z} .
\]
\pause
As $f'(z) = n z^{n-1}$,
$\sabs{f'(0)} = 0 < 1$.

\pause
\medskip

Notice that a bound on the derivative does not hold at other points:
Picking the right $z$ and $n$,
can make $\sabs{f'(z)}$ arbitrarily large.

\medskip
\pause

We can make $\sabs{z^n}$ arbitrarily small for a fixed $z \in \D$ by picking
a large enough $n$, but we cannot make it bigger than $\sabs{z}$.

\medskip
\pause

Schwarz's lemma says all holomorphic functions
behave this way, not just $z^n$.
\end{frame}

\begin{frame}
It is a useful exercise to rewrite the lemma for
$f \colon \Delta_r(p) \to \Delta_s(q)$ with $f(p)=q$.

\medskip
\pause

Using an LFT we can find a half-plane ($\bH = \{ z \in \C : \Im z > 0 \}$)
version:

\medskip
\pause

If $f \colon \bH \to \bH$ holomorphic and $f(i) = i$, then
\[
\abs{\frac{f(z)-i}{\overline{f(z)}-i}} \leq
\abs{\frac{z-i}{\bar{z}-i}} 
\qquad
\text{and}
\qquad
\abs{f'(i)} \leq 1 .
\]

\medskip
\pause
Another interesting (and more challenging) exercise is the following
generalization (sometimes called Cartan's uniqueness theorem)

\medskip
\pause

Suppose $U \subset \C$ is bounded, $f \colon U \to U$ is holomorphic, $p \in U$,
and $f(p)=p$.  Then
\begin{enumerate}[(i)]
\item
$\abs{f'(p)} \leq 1$.
\item
If $f'(p) = 1$, then $f(z) = z$ for all $z \in U$.
\end{enumerate}
\pause
Hint: WLOG $p=0$,
\pause
then consider the power series
expansions of
$f^{\ell}$, the $\ell$\textsuperscript{th} composition of $f$ with itself,
$f(f(f(\cdots f(z) \cdots)))$.

\pause

For a) consider the linear term of $f^{\ell}$.

\pause

For b) use Cauchy estimates on the first nonzero nonlinear term of
$f^{\ell}$.

\end{frame}

\end{document}
