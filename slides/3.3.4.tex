\documentclass[10pt,aspectratio=169]{beamer}

% All the boilerplate is in ccaslides.sty
% Note that this also pulls in a custom vogtwidebar.sty
\usepackage{ccaslides}

\author{Ji\v{r}\'i Lebl}

\institute[OSU]{%
Departemento pri Matematiko de Oklahoma {\^S}tata Universitato}

\title{Cultivating Complex Analysis:\\%
Cauchy estimates, Liouville, and\\%
the fundamental theorem of algebra (3.3.4)}

\date{}

\begin{document}

\begin{frame}
\titlepage
\end{frame}

\begin{frame}
We have the tools to make three fundamental results pop out with little
work:

\begin{enumerate}
\pause
\item
The triangle inequality on the integral formula
for the coefficients of the power series
gives estimates on their size: \textbf{Cauchy's estimates}.

\pause
\item
Cauchy's estimates imply \textbf{Liouville's theorem}: Bounded entire (defined on all
of $\C$) holomorphic functions are constant.

\pause
\item
Liouville's theorem gives \textbf{the fundamental theorem of algebra}:
Every nonconstant polynomial has a root.
\end{enumerate}

\end{frame}

\begin{frame}
For a set $K$, denote the \emph{\myindex{supremum norm}} or
\emph{\myindex{uniform norm}}:
\begin{equation*}
\snorm{f}_K
\overset{\text{def}}{=}
\sup_{z \in K} \sabs{f(z)} .
\end{equation*}

\pause

\begin{theorem}[Cauchy estimates]
Let $U \subset \C$ be open, $f \colon U \to \C$ be
holomorphic, and $\overline{\Delta_r(p)} \subset U$
be a closed disc.  Expand $f(z) = \sum c_n {(z-p)}^n$.
Then for all $n$,
\[
\sabs{c_n} =
\abs{\frac{f^{(n)}(p)}{n!}}
\leq
\frac{\snorm{f}_{\partial \Delta_r(p)}}{r^{n}} .
\]
\end{theorem}

\pause

In other words, the sequence $\bigl\{ \sabs{c_n} r^n \bigr\}$ is bounded by
$\snorm{f}_{\partial \Delta_r(p)}$.

\medskip
\pause

\textbf{Proof:}
\[
\sabs{c_n}  = 
\abs{
\frac{1}{2\pi i}
\int_{\partial \Delta_r(p)}
\frac{f(\zeta)}{{(\zeta-p)}^{n+1}}
\,
d \zeta 
}
\pause
\leq
\frac{1}{2\pi}
\int_{\partial \Delta_r(p)}
\frac{\snorm{f}_{\partial \Delta_r(p)}}{r^{n+1}}
\,
\sabs{d \zeta} 
\pause
=
\frac{\snorm{f}_{\partial \Delta_r(p)}}{r^{n}} .
\qed
\]

\end{frame}

\begin{frame}

A better estimate than
\[
\sabs{c_n} =
\abs{\frac{f^{(n)}(p)}{n!}}
\leq
\frac{\snorm{f}_{\partial \Delta_r(p)}}{r^{n}}
\]
is not possible.

\medskip
\pause

Consider \[
f(z) = \frac{M}{r^n} {(z-p)}^n .
\]

\medskip
\pause

Then
\[
M = \snorm{f}_{\partial \Delta_r(p)} .
\]

\medskip
\pause

And
\[
\sabs{c_n} = \frac{M}{r^n} .
\]

\end{frame}

\begin{frame}

\begin{definition}
A holomorphic $f \colon \C \to \C$ is called
an \emph{\myindex{entire holomorphic function}} or perhaps
just \emph{entire}.
\end{definition}

\pause

Examples: Polynomials, $e^z$, $\sin z$, $\cos z$, but not $\frac{1}{z}$.

\pause

\begin{theorem}[Liouville]
A bounded entire holomorphic function is constant.
\end{theorem}

\pause

\textbf{Proof:}
Suppose $f$ is entire and $\sabs{f(z)} \leq M$ for all $z \in \C$.
\pause
Expand around the origin:
\[
f(z) = \sum_{n=0}^\infty c_n z^n .
\]
\pause
$f$ is holomorphic on a disc of arbitrary radius
\pause \qquad $\Rightarrow$ \qquad
the Cauchy estimates say
\[
\sabs{c_n} \leq \frac{\snorm{f}_{\partial \Delta_r(p)}}{r^n} \leq
\frac{M}{r^n}
\qquad \text{for all } r > 0 .
\]
\pause
Letting $r \to \infty$ shows that $c_n = 0$ for $n \geq 1$.
\pause
In other
words, $f(z) = c_0$ for all $z$.
\qed

\end{frame}

\begin{frame}
\begin{theorem}[Fundamental theorem of algebra]
If $P(z)$ is a nonconstant polynomial, then $P$ has a root.
\end{theorem}

\pause

\textbf{Proof:}
If $P(z)$ does not have a root, then $R(z) = \frac{1}{P(z)}$ is
an entire holomorphic function.

\medskip
\pause

Suppose $P(z)$ is nonconstant.  Then (via an exercise)
\[
\lim_{z \to \infty} P(z) = \infty
\pause
\qquad \Rightarrow \qquad
\lim_{z \to \infty} R(z) = 0.
\]
\pause
So $R(z)$ is bounded.

\medskip
\pause

Liouville says that $R(z)$ and hence $P(z)$ must be constant, a
contradiction.
\qed
\end{frame}




\end{document}
