\documentclass[10pt,aspectratio=169]{beamer}

% All the boilerplate is in ccaslides.sty
% Note that this also pulls in a custom vogtwidebar.sty
\usepackage{ccaslides}

\author{Ji\v{r}\'i Lebl}

\institute[OSU]{%
Departemento pri Matematiko de Oklahoma {\^S}tata Universitato}

\title{Cultivating Complex Analysis:\\%
Montel's theorem (6.2)}

\date{}

\begin{document}

\begin{frame}
\titlepage
\end{frame}

\begin{frame}
\begin{definition}
Let $U \subset \C$ be open.
A set $\sF$ of holomorphic functions $f \colon U \to \C$ is called a
\emph{\myindex{normal family}} if every sequence in $\sF$ has a subsequence
that converges uniformly on compact subsets (the limit need not be in $\sF$).

\medskip
\pause

A set $\sF$ of functions on $U$ is \emph{\myindex{locally bounded}}
if for every $p \in U$, there is a disc $\Delta_r(p) \subset U$ and $M > 0$
such that 
$\snorm{f}_{\Delta_r(p)} \leq M$ for all $f \in \sF$
\pause \quad (i.e., $\sabs{f(z)} \leq M$ for all $z \in \Delta_r(p)$ and all $f \in \sF$).
\end{definition}

\pause

In more modern language:

\medskip

$\sF$ normal family

\qquad $=$

$\sF$ precompact in the space
of holomorphic functions on $U$
with the topology of uniform convergence on compact subsets.

\medskip
\pause

\textbf{Exercise:}
Prove that ``locally bounded'' means ``bounded on compact subsets.''
\end{frame}

\begin{frame}
\begin{theorem}[Montel]
Let $U \subset \C$ be open and let $\sF$
be a locally bounded set of holomorphic functions on $U$.
Then $\sF$ is a normal family \pause (every sequence has a subsequence that
converges uniformly on compact subsets).
\end{theorem}

\pause
We apply Arzel\`a--Ascoli,
so we prove $\sF$
being holomorphic means $\sF$ is equicontinuous.

\medskip

\pause
\textbf{Proof:}
Consider $p \in U$, $\overline{\Delta_r(p)} \subset U$, such that
$\snorm{f}_{\overline{\Delta_r(p)}} \leq M$ for all $f \in \sF$.
\pause

If $z \in \overline{\Delta_{r/2}(p)}$ and $\zeta \in \partial \Delta_r(p)$,
then $\sabs{\zeta-z} \geq \nicefrac{r}{2}$.
\pause
So for $z \in \overline{\Delta_{r/2}(p)}$ and all $f \in \sF$,
\[
\abs{f'(z)}
=
\abs{
\frac{1}{2\pi i}
\int_{\partial \Delta_r(p)}
\frac{f(\zeta)}{{(\zeta-z)}^2} \, d\zeta
}
\pause \leq
\frac{1}{2\pi}
\int_{\partial \Delta_r(p)}
\frac{\sabs{f(\zeta)}}{\sabs{\zeta-z}^2} \, \sabs{d\zeta}
\pause \leq
\frac{1}{2\pi}
\int_{\partial \Delta_r(p)}
\frac{M}{{(r/2)}^2} \, \sabs{d\zeta}
\pause =
\frac{4 M}{r}.
\]
\pause
\[
\abs{f(z)-f(p)}
=
\abs{
\int_{[p,z]} f'(\zeta) \, d\zeta
}
\pause
\leq
\int_{[p,z]} \abs{f'(\zeta)} \, \sabs{d\zeta}
\pause
\leq
\frac{4 M}{r} \sabs{z-p} .
\]
\pause
\thus \quad $\sF$ is equicontinuous ($\frac{4M}{r}$ does not depend on $f$).

\medskip
\pause

\thus \quad
Arzel\`a--Ascoli applies
to any sequence in $\sF$
\pause
\wthus
$\sF$ is a normal family.
\qed
\end{frame}

\begin{frame}
Montel's theorem is good for finding extremal functions.
\pause

E.g., we will prove the Riemann mapping theorem:
\pause
A biholomorphism of a simply
connected $U$ to $\D$ is a map that maximizes the derivative at a point.

\pause
Montel gives us a way of finding a maximizer.

\medskip

\pause

Another commonly used consequence of Montel is Vitali's theorem.

\begin{theorem}[Vitali]
Suppose $U \subset \C$ is a domain, $\{ f_n \}$ is a locally bounded
sequence of holomorphic functions
that converges pointwise on a set $E \subset U$,
and $E$ has a limit point in $U$.  Then $\{ f_n \}$
converges uniformly on compact subsets in $U$.
\end{theorem}

\pause
Proof is an exercise.

\end{frame}

\begin{frame}
\textbf{Exercise:}
Prove the converse to Montel:
If $\sF$ is a normal family of holomorphic functions on an open set $U \subset \C$,
then $\sF$ is locally bounded.

\pause
\medskip

\textbf{Exercise:}
Let $U \subset \C$ be open and $\sF$ a normal family of
holomorphic functions on $U$.
Show that $\{ f' : f \in \sF \}$ is a normal family.
\pause
Note: The converse is false without an extra hypothesis.

\pause
\medskip

\textbf{Exercise:}
Let $U \subset \C$ be a domain, $p \in U$,
and suppose $\exists$ a nonconstant
bounded holomorphic function on $U$.

\pause
a)
Prove $\exists$ a holomorphic $F \colon U \to \D$
such that
$F'(p) \not= 0$, and
$\sabs{f'(p)} \leq \sabs{F'(p)}$
for all holomorphic
$f \colon U \to \D$.

\pause
b)
Show
$F(p) = 0$.

\medskip
\pause

\textbf{Exercise:}
Show that if the partial sums of a power series centered at $p$
are uniformly bounded on $\Delta_r(p)$ for some $r > 0$,
then the power series converges in $\Delta_r(p)$.


\end{frame}

\end{document}
