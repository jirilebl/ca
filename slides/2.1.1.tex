\documentclass[10pt,aspectratio=169]{beamer}

% All the boilerplate is in ccaslides.sty
% Note that this also pulls in a custom vogtwidebar.sty
\usepackage{ccaslides}

\author{Ji\v{r}\'i Lebl}

\institute[OSU]{%
Departemento pri Matematiko de Oklahoma {\^S}tata Universitato}

\title{Cultivating Complex Analysis:\\%
Holomorphic functions (2.1.1)}

\date{}

\begin{document}

\begin{frame}
\titlepage
\end{frame}

\begin{frame}

\section{}
Consider a polynomial
$P(z) = a_n z^n + a_{n-1} z^{n-1} + \cdots + a_1 z + a_0$.

\medskip
\pause

Polynomials are nice, but there aren't many of them.

\medskip
\pause

We can't even solve a simple equation like: $f'=f$.

\medskip
\pause

Fix $z_0 \in \C$ and expand $P(z)$:

\[
P(z) = c_0 + c_1 (z-z_0) + c_2 {(z-z_0)}^2 + \cdots + c_n {(z-z_0)}^n
\]
\pause
or
\[
P(z_0+h) = c_0 + c_1 h + c_2 h^2 + \cdots + c_n h^n
\]
\pause
Then
\[
\lim_{h \to 0} \frac{P(z_0+h) - P(z_0)}{h} =
\lim_{h \to 0} \frac{P(z_0+h) - c_0}{h} = c_1 .
\]
\pause
$P(z_0+h)$ is approximated (locally) by $c_0 + c_1 h$
up to an error that vanishes faster than $h$.

\medskip
\pause

\textbf{Key point:} The limits are ``as a \emph{complex} $h$ goes to $0$.''

\end{frame}

\begin{frame}
We want functions with such a derivative.  That is, functions approximated
by $c_0 + c_1 h$:
\[
f(z_0+h) = \underbrace{f(z_0)}_{c_0} + \underbrace{\xi h}_{c_1 h} + o(\sabs{h})
\]

\medskip
\pause

\begin{definition}
Suppose $U \subset \C$ is open.
Given $f \colon U \to \C$ and $z_0 \in U$, we say 
$f$ is \emph{complex differentiable} at $z_0$ if
the limit
\[
f'(z_0) \overset{\text{def}}{=}
\lim_{h \to 0} \frac{f(z_0+h) - f(z_0)}{h}
\qquad \text{exists.}
\]
\pause
We call $f'(z_0)$ the \emph{complex derivative} of $f$
at $z_0$.  Sometimes $\frac{df}{dz}$ is used.
\pause

$f \colon U \to \C$ is
\emph{holomorphic}
if it is complex differentiable at every point.
\end{definition}

\end{frame}

\begin{frame}

We have essentially proved:

\begin{proposition}
If $P(z)$ is a polynomial, then $P \colon \C \to \C$ is holomorphic.
\end{proposition}
\pause

An exercise:

\begin{proposition}
If $U \subset \C$ is open and $f \colon U \to \C$ is holomorphic, then $f$
is continuous.
\end{proposition}

\end{frame}

\end{document}
