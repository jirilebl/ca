\documentclass[10pt,aspectratio=169]{beamer}

% All the boilerplate is in ccaslides.sty
% Note that this also pulls in a custom vogtwidebar.sty
\usepackage{ccaslides}

\author{Ji\v{r}\'i Lebl}

\institute[OSU]{%
Departemento pri Matematiko de Oklahoma {\^S}tata Universitato}

\title{Cultivating Complex Analysis:\\%
Zeros of holomorphic functions (5.1)}

\date{}

\begin{document}

\begin{frame}
\titlepage
\end{frame}

\begin{frame}
Recall, a \emph{zero} of a function $f$ is a point $z$ such that $f(z) = 0$.

\medskip
\pause

By the identity theorem,
zeros of (nonconstant) holomorphic functions are isolated.
\pause

\begin{lemma}
Let $U \subset \C$ be open, $f \colon U \to \C$ be holomorphic, $p \in U$,
and $f$ has an isolated zero at $p$.

\pause
Then there exists a unique $k \in \N$ and a holomorphic $g \colon U
\to \C$ such that
\[
f(z) = {(z-p)}^k g(z)
\qquad
\text{and} \qquad
g(p) \not= 0 .
\]
\pause
Furthermore, $k$ is the smallest integer such that the $k$\textsuperscript{th}
derivative $f^{(k)}(p) \not= 0$.
\end{lemma}

\pause

Before we prove the lemma, let us give a name to this integer $k$.

\pause

\begin{definition}
%Suppose $f$ has an isolated zero at $p$.
The $k$ from the lemma is called the \emph{order}
of the zero at $p$.
\pause

If the order is $1$, we say $p$ is a \emph{simple zero}.

\pause

We also say that $k$ is the \emph{multiplicity} of the zero.
\end{definition}
\end{frame}

\begin{frame}
An equivalent definition of 
\emph{order} is the largest $k$ such that
$\frac{f(z)}{{(z-p)}^{k}}$ is bounded near $p$.

\medskip
\pause

In fact (exercise), there is a disc $\Delta_r(p)$ and some
$C_1 > 0$ and $C_2 > 0$ such that
\[
C_1 {\sabs{z-p}}^k
\leq
\sabs{f(z)}
\leq
C_2 {\sabs{z-p}}^k \qquad
\forall z \in \Delta_r(p) .
\]

\pause

The conclusion of the lemma still holds if $f(p)\not=0$, in which case
order is $k=0$ (and $g=f$).

\medskip
\pause

So we could say that if $f(p) \not=0$, then \emph{$f$ has a zero of order $0$.}

\medskip
\pause

We'll even see negative orders in just a bit.

\medskip
\pause

To avoid confusion, by ``$f$ has a zero,'' we mean an honest zero of positive order.

\medskip
\pause

If $f^{(k)}(p) = 0$ for all $k$, one could say that $f$
has a zero of infinite order.

\medskip
\pause

For $f$ holomorphic, infinite order $\Rightarrow$ the power series is
zero $\Rightarrow$ $f$ is identically zero.

\medskip
\pause

So every zero of a nonconstant holomorphic function has finite order.

\medskip
\pause

This is not true for just real differentiable (not holomorphic) functions (see the exercises):

E.g., let $f(0)=0$ and $f(x)=e^{-1/x^2}$ for $x \in \R \setminus \{ 0\}$.

$f$ is infinitely differentiable, $f^{(k)}(0)=0$ for all $k$, but
$f$ has an isolated zero.
\end{frame}

\begin{frame}
\textbf{Proof of the lemma:}
On $U \setminus \{ p \}$, $g(z) = \frac{f(z)}{{(z-p)}^k}$ is
holomorphic for any $k$.

\medskip
\pause

For $z$ in some disc $\Delta_r(p)$,
\[
f(z)
=
\sum_{n=0}^\infty c_n {(z-p)}^n
\pause
=
\sum_{n=k}^\infty c_n {(z-p)}^n 
\pause
= {(z-p)}^k
\sum_{n=0}^\infty c_{n+k} {(z-p)}^{n} ,
\]
\pause
where $k$ is the smallest $n$ such that $c_n \not= 0$ (hence the ``Furthermore'').

\medskip
\pause

Clearly $k > 0$.

\medskip
\pause

The series $\sum_{n=0}^\infty c_{n+k}{(z-p)}^n$
is equal to $\frac{f(z)}{{(z-p)}^k}$ on
$\Delta_r(p)\setminus \{ p \}$.

\medskip
\pause

So this series gives $g$ near $p$.  Let $g(p)=c_k$, and $g \colon U \to \C$ is 
holomorphic.

\medskip
\pause

Uniqueness of $k$:
\pause
Suppose ${(z-p)}^{k_1} g_1(z) = {(z-p)}^{k_2} g_2(z)$, where
$g_1(p) \not= 0$ and $g_2(p) \not= 0$.

\medskip
\pause

WLOG suppose $k_1 \leq k_2$,
\pause
then $g_1(z) = {(z-p)}^{k_2-k_1} g_2(z)$.

\medskip
\pause

Plug in $z=p$ to see $k_2 = k_1$.
\qed
\end{frame}

\begin{frame}

Near a zero of order $k$, a holomorphic function acts like
$z^k$ acts near the origin:

\pause

\begin{theorem}
Suppose $U \subset \C$ is open, $f \colon U \to \C$ is holomorphic,
and $p \in U$ is a zero of $f$ of order $k \in \N$ (an honest zero).
\pause
Then there exists an open neighborhood $V$
of $p$ and a holomorphic $g \colon V \to \C$ such that
\[
f(z) = {\bigl( g(z) \bigr)}^k,
\qquad
\text{where $g(p) = 0$ and $g'(p) \not= 0$.}
\]
\end{theorem}

\medskip
\pause

More fancy: $g$ is a local biholomorphic change of variables near
$p$
that makes $p$ into the origin, and makes $f$ into $z^k$.

\medskip
\pause

\textbf{Proof:}
Let $V = \Delta_r(p)$ be such that $f(z) \not= 0$ for any $z \in
\Delta_r(p) \setminus \{ p \}$.

\medskip
\pause

Lemma says $\exists$ holomorphic
$h \colon \Delta_r(p) \to \C$ such that
$h(p) \not= 0$ and $f(z) = {(z-p)}^k h(z)$.

\medskip
\pause

In particular,
$h(z) \not= 0$ for any $z \in \Delta_r(p)$ ~~($h$ is nowhere zero).

\medskip
\pause

As
$\Delta_r(p)$ is simply connected
$\Rightarrow$
$\exists$ holomorphic $\varphi \colon \Delta_r(p) \to \C$ such that
$\varphi^k = h$.

\medskip
\pause

Let $g(z) = (z-p)\varphi(z)$. \qquad (So $g^k = f$ and $g(p)=0$)

\medskip
\pause

As
$g'(z) = (z-p) \varphi'(z) + \varphi(z)$, we have $g'(p) = \varphi(p) \not= 0$.
\qed
\end{frame}

\begin{frame}

Every semester, someone wants to use L'H\^{o}pital on homework without
proof.  You can prove it now.

\medskip
\pause

\textbf{Exercise:}
Prove L'H\^{o}pital's rule: If $f$ and $g$ are holomorphic near $p$,
both with an isolated zero at $p$, and
$\lim\limits_{z\to p} \frac{f'(z)}{g'(z)}$ exists (including possibly $\infty$), then
$\lim\limits_{z\to p} \frac{f(z)}{g(z)}$ exists and equals the same thing.
\end{frame}

\end{document}
