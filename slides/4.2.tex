\documentclass[10pt,aspectratio=169]{beamer}

% All the boilerplate is in ccaslides.sty
% Note that this also pulls in a custom vogtwidebar.sty
\usepackage{ccaslides}

\author{Ji\v{r}\'i Lebl}

\institute[OSU]{%
Departemento pri Matematiko de Oklahoma {\^S}tata Universitato}

\title{Cultivating Complex Analysis:\\%
Homology versions of Cauchy (4.2)}

\date{}

\begin{document}

\begin{frame}
\titlepage
\end{frame}

\begin{frame}
\begin{definition}
Let $U \subset \C$ be open and $\Gamma$
a cycle
in $U$
such that $n(\Gamma;p) = 0$ for all $p \in \C \setminus U$,
 then we
say $\Gamma$ is \emph{homologous to zero in $U$}.
\end{definition}

\pause

``homologous to zero in $U$'' $=$ ``does not wind around any point in $\C
\setminus U$.''

\medskip
\pause

\textbf{Note:} ``homologous to zero`` does \textbf{not} mean ``equivalent to
zero.''

\medskip
\pause

\textbf{Example:}
In $U = \C$, every $\Gamma$ is homologous to zero.

\medskip
\pause

\textbf{Example:}
The unit circle is homologous to zero in $\C$, but not homologous to
zero in $\C \setminus \{ 0 \}$.

\pause

\begin{theorem}[Cauchy integral formula (homology version)]
Suppose $U \subset \C$ is open, $f \colon U \to \C$ is holomorphic, and
$\Gamma$ is
a cycle
in $U$
homologous to zero in $U$.
Then for $z \in U \setminus \Gamma$,
\begin{equation*}
n(\Gamma;z)
\,
f(z)
=
\frac{1}{2\pi i}
\int_{\Gamma}
\frac{f(\zeta)}{\zeta-z}
\,
d \zeta .
\end{equation*}
\end{theorem}

\end{frame}

\begin{frame}
\textbf{Proof:}
Define $g \colon U \times U \to \C$ by
\[
g(\zeta,z) =
\begin{cases}
\,
\frac{f(\zeta)-f(z)}{\zeta-z} & \text{if } \zeta \not= z , \\
\,
f'(\zeta)                 & \text{if } \zeta = z .
\end{cases}
\]
\pause

\textbf{Exercise:}
$g(\zeta,z)$ is continuous in $U \times U$,
and 

\pause 
$z \mapsto g(\zeta,z)$ is holomorphic for every fixed $\zeta \in U$.

\medskip
\pause

Let
\vspace*{-5pt}
\[
h(z) = 
\begin{cases}
\int_\Gamma g(\zeta,z) \, d\zeta & \text{if } z \in U , \\
\int_\Gamma \frac{f(\zeta)}{\zeta-z} \, d\zeta & \text{if } z \not \in
\Gamma \text{ and } n(\Gamma;z) = 0 .
\end{cases}
\]
\pause

$h(z)$ is defined for all $z \in \C$ (as $n(\Gamma;z) = 0$ for all $z \in \C
\setminus U$).

\medskip
\pause

But, at some points we have two definitions!

\medskip
\pause

%Must show that if
%$z \in U \setminus \Gamma$ and $n(\Gamma;z) = 0$, then the two definitions agree.
%
%\medskip
%\pause

Suppose
$z \in U \setminus \Gamma$ and $n(\Gamma;z) = 0$.
\pause
\[
\int_\Gamma g(\zeta,z) \, d\zeta
\pause
=
\int_\Gamma \frac{f(\zeta)-f(z)}{\zeta-z} \, d\zeta
\pause
=
\int_\Gamma \frac{f(\zeta)}{\zeta-z} \, d\zeta
-
f(z) (2\pi i) n(\Gamma;z)
\pause
=
\int_\Gamma \frac{f(\zeta)}{\zeta-z} \, d\zeta .
\]
\pause
So $h \colon \C \to \C$ is well-defined.
\end{frame}

\begin{frame}
WTS $h \colon \C \to \C$ is holomorphic.

\medskip
\pause

Holomorphicity is a local property, so only consider $h$ on open sets
covering $\C$.

\medskip
\pause

The set of $z \notin \Gamma$ where $n(\Gamma;z) = 0$ is open, and 
$\displaystyle z \mapsto \int_\Gamma \frac{f(\zeta)}{\zeta-z} \, d\zeta$ is
holomorphic there.

\medskip
\pause

Similarly $U$ is open and $\displaystyle
z \mapsto
\int_\Gamma g(\zeta,z) \, d\zeta$ is holomorphic there, so $h$ is
holomorphic.

\medskip
\pause

Consider $z$ in the unbounded component of
$\C \setminus \Gamma$.
\pause
Then $n(\Gamma;z) = 0$, so
$h(z) = 
\int_\Gamma \frac{f(\zeta)}{\zeta-z} \, d\zeta$.

\medskip
\pause

Let $M$ be such that
$\sabs{f(\zeta)} \leq M$ for $\zeta \in \Gamma$ and
\pause
$\ell$ be the length of
$\Gamma$.
\pause
\[
\sabs{h(z)}
\pause
=
\abs{
\int_\Gamma \frac{f(\zeta)}{\zeta-z} \, d\zeta
}
\pause
\leq
\int_\Gamma \abs{\frac{f(\zeta)}{\zeta-z}} \, \sabs{d\zeta}
\pause
\leq
\frac{M \ell}{d(z,\Gamma)}
\qquad \text{($d(z,\Gamma)$ is the distance of $z$ and $\Gamma$).}
\]
\pause
$z \to \infty$ \quad $\Rightarrow$ \quad $d(z,\Gamma) \to \infty$ \quad
$\Rightarrow$ \quad $h(z) \to 0$

\medskip
\pause

So $h$ is bounded, by Liouville $h$ is constant, and the constant is zero.

\medskip
\pause

Suppose $z \in U \setminus \Gamma$.  Then
\[
0 = h(z)
\pause
=
\int_\Gamma \frac{f(\zeta)-f(z)}{\zeta-z} \, d\zeta
\pause
=
\int_\Gamma \frac{f(\zeta)}{\zeta-z} \, d\zeta
-
f(z) \, (2\pi i) \, n(\Gamma;z) . \qquad \qed
\]

\end{frame}

\begin{frame}

\begin{theorem}[Cauchy's theorem (homology version)]
Suppose $U \subset \C$ is open,
$f \colon U \to \C$ is holomorphic,
and $\Gamma$ is
a cycle
in $U$
homologous to zero in $U$.
Then
\begin{equation*}
\int_\Gamma f(z) \, dz = 0 .
\end{equation*}
\end{theorem}

\pause

\textbf{Proof:}
Fix $z \in U \setminus \Gamma$.

\medskip
\pause
Apply 
the Cauchy integral formula to $\zeta \mapsto (\zeta-z)\,f(\zeta)$ at
$\zeta=z$:
\[
\pause
0 = n(\Gamma;z) (z-z)f(z)
\pause
=
\frac{1}{2\pi i} \int_\Gamma \frac{(\zeta-z)\,f(\zeta)}{\zeta-z} \, d\zeta
\pause
=
\frac{1}{2\pi i} \int_\Gamma f(\zeta) \, d\zeta .
\qquad
\qed
\]

\pause

\textbf{Remark:} Cauchy integral formula and Cauchy's theorem are equivalent
logically (if you prove one the other follows).

\end{frame}

\begin{frame}

\begin{definition}
Cycles
$\Gamma_0$ and $\Gamma_1$ in $U
\subset \C$ are \emph{homologous} in $U$
if $n(\Gamma_0;p) = n(\Gamma_1;p)$ for all $p \in \C \setminus U$.
\end{definition}

Equivalently, $\Gamma_0-\Gamma_1$ is homologous to zero in $U$.

\pause

\begin{corollary}
Let $U \subset \C$ be open and $f \colon U \to \C$ holomorphic.
If two cycles
$\Gamma_0$ and $\Gamma_1$ in $U$
are homologous in $U$, then
\begin{equation*}
\int_{\Gamma_0} f(z)\, dz = 
\int_{\Gamma_1} f(z)\, dz .
\end{equation*}
\end{corollary}

\pause

That is useful for computing integrals:

Given a complicated cycle, find a
simple one that is homologous.

\pause
\medskip

E.g. (exercise), any cycle in $\C \setminus \{ 0 \}$ is homologous to
$n \partial \D$ for some $n \in \Z$.

\pause
\medskip

\textbf{Remark:} Being ``homologous'' is an equivalence relation and the set of 
equivalence classes of cycles is an abelian group (under the cycle addition,
exercise).  This group is called the
first \emph{homology group} of $U$.
\end{frame}

\end{document}
