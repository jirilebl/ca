\documentclass[10pt,aspectratio=169]{beamer}

% All the boilerplate is in ccaslides.sty
% Note that this also pulls in a custom vogtwidebar.sty
\usepackage{ccaslides}

\author{Ji\v{r}\'i Lebl}

\institute[OSU]{%
Departemento pri Matematiko de Oklahoma {\^S}tata Universitato}

\title{Cultivating Complex Analysis:\\%
Winding numbers (4.1.2)}

\date{}

\begin{document}

\begin{frame}
\titlepage
\end{frame}

\begin{frame}
Let $\Gamma$ be
a cycle,
and $p \notin \Gamma$.  Define
\begin{equation*}
n(\Gamma;p)
\overset{\text{def}}{=}
\frac{1}{2\pi i} \int_\Gamma \frac{1}{z-p} \, dz .
\end{equation*}
\pause
$n(\Gamma;p)$ is called the
\emph{\myindex{winding number}} of $\Gamma$ around $p$, or
the 
\emph{\myindex{index}} of $\Gamma$ with respect to $p$.

\medskip
\pause

It is the number of times $\Gamma$ winds around $p$ in the counterclockwise
direction.

\medskip
\pause

\textbf{Example:} $\gamma(t) = e^{it}$ for $t \in [0,2\pi]$ goes once around
$0$ in the counterclockwise direction,
\[
\frac{1}{2\pi i} \int_\gamma \frac{1}{z} \, dz = 1.
\]

\pause
\textbf{Example:} $\gamma(t) = e^{i2t}$ for $t \in [0,2\pi]$ goes twice around
$0$,
\[
\frac{1}{2\pi i} \int_\gamma \frac{1}{z} \, dz = 2.
\]

\pause
\textbf{Example:} $\gamma(t) = e^{-it}$ for $t \in [0,2\pi]$ goes once
around in the clockwise direction,
\[
\frac{1}{2\pi i} \int_\gamma \frac{1}{z} \, dz = -1.
\]
\end{frame}

\begin{frame}
\begin{proposition}
Suppose $\Gamma$ is a cycle and $p \notin \Gamma$.  Then
$n(\Gamma;p)$ is an integer.
\end{proposition}

\pause

Idea: Follow a branch of log, then the argument differs
by an integer multiple of $2\pi$.

\medskip
\pause
\textbf{Proof:}
$\Gamma$ is a ``sum'' of closed paths, so WLOG consider a closed
piecewise-$C^1$ path
$\gamma \colon [0,1] \to \C$.

\medskip
\pause

$\gamma$ can be covered by finitely many discs $D_1,\ldots,D_n$
none of which contain $p$.

\medskip
\begin{center}
\subimport*{../figures/}{followbranch.pdf_t}
\end{center}

\pause
(cover the whole closed curve, of course)

\end{frame}

\begin{frame}
The discs $D_1,\ldots,D_n$ cover $\gamma$.

\medskip
\pause

And the discs can be chosen (exercise) so that
there is a partition

$0 = t_0 < t_1 < t_2 < \cdots < t_n = 1$, where
$\gamma\bigl([t_{j-1},t_j]\bigr) \subset D_j$ for every $j$.

\medskip
\pause

Each $D_j$ is star-like and $p \notin D_j$ \quad $\Rightarrow$ \quad
$\exists$ a branch $L_j$ 
of $\log (z-p)$ on each $D_j$, such that
\[
L_j\bigl(\gamma(t_j)\bigr) = L_{j+1}\bigl(\gamma(t_j)\bigr)
\qquad\qquad
\text{($L_1$ is an arbitrary branch).}
\]
\pause
Call $z_0 = \gamma(0) = \gamma(1)$. \pause
So

\medskip
$\displaystyle
\qquad n(\gamma;p)
=
\frac{1}{2\pi i} \int_\gamma \frac{1}{z-p} \, dz
\pause
=
\frac{1}{2\pi i} \int_0^1 \frac{\gamma'(t)}{\gamma(t)-p} \, dt
\pause
=
\frac{1}{2\pi i} \sum_{j=1}^n \int_{t_{j-1}}^{t_j} \frac{\gamma'(t)}{\gamma(t)-p} \, dt
$
\pause

$\displaystyle
\phantom{\qquad n(\gamma;p)}
=
\frac{1}{2\pi i} \sum_{j=1}^n L_j\bigl(\gamma(t_j)\bigr) -
L_j\bigl(\gamma(t_{j-1})\bigr)
\pause
=
\frac{1}{2\pi i} \bigl( L_n(z_0) - L_1(z_0) \bigr) .
$

\medskip
\pause

$L_n$ and $L_1$ are branches of $\log$,

\pause
each is $\log\sabs{z_0} + i \arg z_0$ for some value of $\arg$,

\pause
their difference is $2\pi k i$ for some $k \in \Z$.
\qed
\end{frame}

\begin{frame}
\begin{proposition}
Given a cycle $\Gamma$,
the function $z \mapsto n(\Gamma;z)$ is constant on the
topological components of $\C \setminus \Gamma$.
\pause
Furthermore, $n(\Gamma;z) = 0$ for $z$ on the unbounded component
of $\C \setminus \Gamma$.
\end{proposition}

\pause

As $\Gamma$ is compact, is a unique unbounded component
of $\C \setminus \Gamma$.

\medskip
\pause

\textbf{Example:}

\begin{center}
\scalebox{0.9}{
\subimport*{../figures/}{indexconstant.pdf_t}
}
\end{center}
\end{frame}

\begin{frame}
\textbf{Proof:}
We start by showing that
$\displaystyle
p \mapsto n(\Gamma;p) = \frac{1}{2\pi i} \int_\Gamma \frac{1}{z-p} \, dz
$
is continuous on $\C \setminus \Gamma$.

\medskip
\pause

Fix $p_0 \in \C \setminus \Gamma$, and let
$d = d(p_0,\Gamma)$ be the
distance from 
$p_0$ to $\Gamma$ ($d > 0$ as $\Gamma$ is compact).

\medskip
\pause

If $p \in \Delta_{d/2}(p_0)$, then $\sabs{z-p} \geq \nicefrac{d}{2}$ for $z \in \Gamma$.

\medskip
\pause
Let $\ell = \int_\Gamma \sabs{dz}$ (length of $\Gamma$). \pause Then

\medskip
$\displaystyle
\abs{n(\Gamma;p_0)-n(\Gamma;p)}
\pause
=
\abs{\frac{1}{2\pi i} \int_\Gamma \frac{p_0-p}{(z-p_0)(z-p)} \, dz}
\pause
\leq
\frac{1}{2\pi} \int_\Gamma \frac{\sabs{p_0-p}}{\sabs{z-p_0}\sabs{z-p}} \, \sabs{dz}
%$
%
%$\displaystyle
\pause
\leq 
\frac{\ell}{\pi {d}^2} \sabs{p_0-p} .
$

\pause
\medskip
So, $p \mapsto n(\Gamma;p)$ is a continuous.

\medskip
\pause

Being continuous and integer-valued, it is constant on every component
of $\C \setminus \Gamma$.

\medskip
\pause

For $p \in \C \setminus \Gamma$,
\[
\sabs{n(\Gamma;p)}
\pause
\leq
\frac{1}{2\pi} \int_\Gamma \frac{1}{\sabs{z-p}} \, \sabs{dz}
\pause
\leq \frac{1}{2\pi} \frac{\ell}{d(p,\Gamma)} .
\]
\pause
On the unbounded component,
there are $p$ with arbitrarily large
$d(p,\Gamma)$.

\medskip
\pause

So $n(\Gamma;p)=0$ on this component.
\qed
\end{frame}

\begin{frame}
An exercise (direct computation) which we will often use is:

\medskip
\pause

\textbf{Exercise:}

\medskip

$n\bigl(\partial \Delta_r(p);z\bigr) = 0$ if $z \notin
\overline{\Delta_r(p)}$, and

\medskip
\pause

$n\bigl(\partial \Delta_r(p);z\bigr) = 1$ if $z \in \Delta_r(p)$.

\medskip
\medskip
\pause

\textbf{Exercise:}
Compute the winding numbers in an ``annulus``:

\medskip
\pause

Suppose $0 < r_1 < r_2 < \infty$ and
$\Gamma = \partial \Delta_{r_2}(p) - \partial \Delta_{r_1} (p)$.

\medskip
\pause

Then

\medskip

$n(\Gamma;z) = 0$ if $\sabs{z-p} < r_1$,

\medskip
\pause

$n(\Gamma;z) = 1$ if $r_1 < \sabs{z-p} < r_2$,

\medskip
\pause

$n(\Gamma;z) = 0$ if $r_2 < \sabs{z-p}$.
\end{frame}




\end{document}
