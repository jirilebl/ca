\documentclass[10pt,aspectratio=169]{beamer}

% All the boilerplate is in ccaslides.sty
% Note that this also pulls in a custom vogtwidebar.sty
\usepackage{ccaslides}

\author{Ji\v{r}\'i Lebl}

\institute[OSU]{%
Departemento pri Matematiko de Oklahoma {\^S}tata Universitato}

\title{Cultivating Complex Analysis:\\%
Line integrals, chains (3.1 part 3)}

\date{}

\begin{document}

\begin{frame}
\titlepage
\end{frame}

\begin{frame}
It is useful to combine paths to obtain so-called \emph{chains}.
\pause
For two paths $\gamma$ and $\alpha$,
\[
\int_{\gamma + \alpha} f(z) \, dz
\overset{\text{def}}{=}
\int_{\gamma} f(z) \, dz +
\int_{\alpha} f(z) \, dz .
\]

\pause

\begin{definition}
A \emph{chain} in $U \subset \C$ is an expression
$\Gamma = a_1 \gamma_1 + \cdots + a_n \gamma_n$,
where $a_1,\ldots,a_n \in \Z$ and $\gamma_1,\ldots,\gamma_n$
are piecewise-$C^1$ paths in $U$. \pause  We integrate over $\Gamma$ as
\[
\int_{\Gamma} f(z) \, dz
=
\int_{a_1 \gamma_1 + \cdots + a_n \gamma_n} f(z) \, dz
\overset{\text{def}}{=}
a_1 \int_{\gamma_1} f(z) \, dz +
\cdots
+
a_n \int_{\gamma_n} f(z) \, dz .
\]
\pause
Two chains $\Gamma_1$ and $\Gamma_2$ in 
$U$ are
\emph{equivalent} (we will write
$\Gamma_1=\Gamma_2$) if
\[
\int_{\Gamma_1} f(z) \, dz = 
\int_{\Gamma_2} f(z) \, dz
\qquad
\text{for all continuous } f \colon U \to \C.
\]
\pause
Define the \emph{zero chain} $0$ by defining 
$\int_0 f(z) \, dz = 0$ for all continuous $f \colon U \to \C$.
\end{definition}
\end{frame}

\begin{frame}
Chain arithmetic is done in the obvious way:

\medskip
\pause

If $\Gamma_1 = 2 \gamma_1 + \gamma_2$ and $\Gamma_2 = 3 \gamma_2 +
\gamma_3$,

then $\Gamma_1 + \Gamma_2 = 2 \gamma_1 + 4 \gamma_2 + \gamma_3$.

\medskip
\pause

Similarly, $3 \Gamma_1 = 6 \gamma_1 + 3 \gamma_2$.

\medskip
\pause

We write $-\Gamma$ for $(-1)\Gamma$.

\medskip
\pause

$\Gamma$ is equivalent to the zero chain if
\begin{equation*}
\int_\Gamma f(z)\, dz = 0
\end{equation*}
for all continuous $f$, and
the chains $\Gamma_1$ and $\Gamma_2$ are equivalent if $\Gamma_1 - \Gamma_2 = 0$.

\medskip
\pause

\textbf{Remark:}
The domain of the continuous $f$ is not a big deal.  Whether on
$U$, $\Gamma_1 \cup \Gamma_2$, or $\C$.
By Tietze's extension theorem
every continuous function on a closed subset of $\C$ (e.g. $\Gamma_1 \cup
\Gamma_2$) extends to a
continuous function on $\C$.

\medskip
\pause

\textbf{Remark:}
Equivalence is for all \emph{continuous} functions.
We will show later that for many $U$ and many $\Gamma$,
$\int_\Gamma f(z) \, dz = 0$ for all holomorphic $f$.
\end{frame}

\begin{frame}
For any piecewise-$C^1$ path, we could construct an equivalent
chain from $C^1$ paths.

\medskip
\pause

Vice versa, for any chain where the paths connect together end to end,
we could replace them with a single path.

\medskip
\pause

Given two points $z,w \in \C$, the \emph{segment} $[z,w]$ is the path
$\gamma \colon [0,1] \to \C$ given by $\gamma(t) = (1-t)z + tw$.
\pause
(In chain arithmetic, $-[z,w] = [w,z]$.)

\medskip
\pause

A path is \emph{polygonal} if it is equivalent to a chain
$[z_1,z_2] + [z_2,z_3] + \cdots + [z_{k-1},z_k]$.

\medskip
\pause

Note that for any continuous $f$ and a path, we can find a polygonal
path where the integral of $f$ is arbitrarily close to the original (exercise).

\medskip
\pause

So really, knowing how to integrate polygonal paths is
good enough.

\medskip
\pause

Most often used paths are composed of segments and arcs of circles.
\end{frame}

\end{document}
