\documentclass[10pt,aspectratio=169]{beamer}

% All the boilerplate is in ccaslides.sty
% Note that this also pulls in a custom vogtwidebar.sty
\usepackage{ccaslides}

\author{Ji\v{r}\'i Lebl}

\institute[OSU]{%
Departemento pri Matematiko de Oklahoma {\^S}tata Universitato}

\title{Cultivating Complex Analysis:\\%
The geometry and topology of the plane (1.1.2)\\%
Complex-valued functions (1.1.3)}

\date{}

\begin{document}

\begin{frame}
\titlepage
\end{frame}

\begin{frame}
We measure distance with the \emph{modulus} (euclidean distance from zero):
\[
\sabs{z} \overset{\text{def}}{=} \sqrt{z \bar{z}} = \sqrt{x^2+y^2} .
\]
\pause
The (euclidean) distance 
\(
\sabs{z-w}
\)
makes $\C$ into a complete metric space.

\medskip
\pause

\begin{proposition}[Cauchy--Schwarz and the triangle inequality]
If $z,w \in \C$, then
\begin{enumerate}[(i)]
\item
$\sabs{\Re z\bar{w}} \leq \sabs{z} \sabs{w}$ \quad (Cauchy--Schwarz inequality, note: $\Re z
\bar{w}$ is the real dot product),
\pause
\item
$\sabs{z+w} \leq \sabs{z} + \sabs{w}$ \quad (Triangle inequality).
\end{enumerate}
\end{proposition}

\pause

\textbf{Proof:}
Cachy--Schwarz:

\medskip

\(
\qquad
0  \leq {\sabs{z\bar{w}-\bar{z}w}}^2 \pause
   =    (z\bar{w}-\bar{z}w)(\bar{z}w-z\bar{w})  \pause
   =    2z\bar{z}w\bar{w} - z^2\bar{w}^2 - \bar{z}^2w^2 \pause
\)

\( \qquad \qquad
   =    4z\bar{z}w\bar{w} - {(z\bar{w}+\bar{z}w)}^2 \pause
   =    {\bigl(2\sabs{z}\sabs{w}\bigr)}^2 - {\bigl(2 \Re z\bar{w}\bigr)}^2 .
\)

\medskip
\pause

Triangle inequality:

\medskip
\(
\qquad
{\sabs{z+w}}^2  =    (z+w)(\bar{z}+\bar{w}) \pause
                =    z\bar{z} + w\bar{w} + z\bar{w} + \bar{z}w \pause
                \leq z\bar{z} + w\bar{w} + 2 \sabs{z}\sabs{w} \pause
                =    {\bigl(\sabs{z}+\sabs{w}\bigr)}^2 .
\) \qed

\end{frame}

\begin{frame}

\begin{proposition}
Complex addition, multiplication, division, and conjugation are continuous:
Suppose $\{ a_n \}$ and $\{ b_n \}$ are two convergent sequences
of complex numbers.  Then,
\begin{enumerate}[(i)]
\item
$\lim\limits_{n\to\infty} (a_n + b_n) = 
\left(\lim\limits_{n\to\infty} a_n \right) +
\left(\lim\limits_{n\to\infty} b_n \right)$,
\item
$\lim\limits_{n\to\infty} a_n b_n = 
\left(\lim\limits_{n\to\infty} a_n \right)
\left(\lim\limits_{n\to\infty} b_n \right)$,
\item
$\lim\limits_{n\to\infty} \frac{1}{a_n} = \frac{1}{\lim\limits_{n\to\infty} a_n}$,
as long as $\lim\limits_{n\to\infty} a_n \not= 0$,
\item
$\lim\limits_{n\to\infty} \bar{a}_n = 
\overline{\lim\limits_{n\to\infty} a_n}$.
\end{enumerate}
\end{proposition}

\medskip
\pause

All these operations are defined in terms of operations on the real an
imaginary parts which are continuous.  Details left as exercise.

\end{frame}

\begin{frame}

If $p \in \C$ and $r > 0$, define the \emph{disc} of radius $r$ around $p$
as
\begin{equation*}
\Delta_r(p)
\overset{\text{def}}{=}
\bigl\{ z \in \C : \sabs{z-p} < r \bigr\} .
\end{equation*}
\pause
\begin{equation*}
\D
\overset{\text{def}}{=}
\Delta_1(0)
=
\bigl\{ z \in \C : \sabs{z} < 1 \bigr\} 
\qquad \text{\emph{unit disc}}.
\end{equation*}

\medskip
\pause
A useful ``version'' of $\D$ is the \emph{upper half-plane}:
\begin{equation*}
\bH 
\overset{\text{def}}{=}
\bigl\{
z \in \C : \Im z > 0
\bigr\} .
\end{equation*}

\medskip
\pause

\begin{definition}
An open and connected set $U \subset \C$ is called a
\emph{domain}.
\end{definition}

\end{frame}

\begin{frame}
If $f \colon X \to \C$, write $u = \Re f$ and $v = \Im f$:
\qquad
\(\displaystyle
f = u+iv .
\)
\pause
\medskip

If $X \subset \C$ ($z = x+iy$):
\qquad
\(
\displaystyle
\frac{\partial f}{\partial x} = 
\frac{\partial u}{\partial x} + i
\frac{\partial v}{\partial x}
\qquad\text{and}\qquad
\frac{\partial f}{\partial y} = 
\frac{\partial u}{\partial y} + i
\frac{\partial v}{\partial y} .
\)
\medskip
\pause

If $X \subset \R$:
\qquad
\(
f' = u' + iv'.
\)
\medskip
\pause

If $f \colon [a,b] \to \C$, $f$
is (Riemann) integrable if $u$ and $v$ are, and
\begin{equation*}
\int_a^b f(t) \, dt = 
\int_a^b u(t) \, dt + i \int_a^b v(t) \, dt .
\end{equation*}
\pause

\begin{proposition}
Suppose $f \colon [a,b] \to \C$ is (Riemann) integrable.  Then $\sabs{f}$ is
(Riemann) integrable and
\begin{equation*}
\bbabs{\int_a^b f(t) \, dt} \leq 
\int_a^b \abs{f(t)} \, dt .
\end{equation*}
\end{proposition}

\end{frame}

\end{document}
