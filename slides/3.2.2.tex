\documentclass[10pt,aspectratio=169]{beamer}

% All the boilerplate is in ccaslides.sty
% Note that this also pulls in a custom vogtwidebar.sty
\usepackage{ccaslides}

\author{Ji\v{r}\'i Lebl}

\institute[OSU]{%
Departemento pri Matematiko de Oklahoma {\^S}tata Universitato}

\title{Cultivating Complex Analysis:\\%
Cauchy--Goursat, the ``Cauchy for triangles'' (3.2.2)}

\date{}

\begin{document}

\begin{frame}
\titlepage
\end{frame}

\begin{frame}
A set $X$ is \emph{convex} 
if $[a,b] \subset X$ for all $a,b \in X$.

\medskip
\pause

Let $a,b,c \in \C$ be noncollinear.

\medskip
\pause

A \emph{triangle} $T$ is with vertices $a,b,c$
is the convex hull of $\{ a,b,c \}$, that is,
the smallest convex set containing the points.

\medskip
\pause

In other words, $T$ is the set of points
\begin{equation*}
t_1 a + t_2 b + t_3 c ,
\end{equation*}
where $t_1,t_2,t_3 \in [0,1]$ and $t_1+t_2+t_3 = 1$.

\medskip
\pause

The triangle is oriented positively if the vertices 

are ordered so that $a,b,c$ goes counterclockwise.

\vspace*{-0.4in}
\hspace*{3.1in}%
\subimport*{../figures/}{triang.pdf_t}

\vspace*{-0.4in}
\pause

The boundary $\partial T$ of $T$ is defined as the cycle

\medskip

$\partial T = [a,b] + [b,c] + [c,a]$.

\medskip
\pause

Note that our triangle $T$ is the \textbf{solid} triangle

(includes the interior).

\end{frame}

\begin{frame}
\begin{theorem}[Cauchy--Goursat]
Suppose $U \subset \C$ is open, $f \colon U \to \C$ is
holomorphic,
and $T \subset U$ is a triangle.  Then
\[
\int_{\partial T} f(z) \, dz = 0 .
\]
\end{theorem}

\pause

It is important is that $T \subset U$ means the whole solid triangle
is in $U$, not just the boundary.

\pause
\medskip

\textbf{Remark:} It is a ``Goursat'' theorem not just ``Cauchy''
because of the proof: We do not assume that
$f'$ is continuous as we have not proved that yet.

\pause
\medskip

\textbf{Proof:}
We prove the contrapositive.

\medskip
\pause

Suppose $f$ is continuous and suppose $\exists$ $T \subset U$ such that
\[
\abs{\int_{\partial T} f(z) \, dz} = c \not= 0 .
\]
\pause
We will find a point where $f$ is not complex differentiable.
\end{frame}

\begin{frame}
Cut $T$ into four subtriangles $T_1$, $T_2$, $T_3$, $T_4$

(cut each side in half)

\vspace*{-0.5in}
\hspace*{2.5in}
\subimport*{../figures/}{trianggoursat.pdf_t}

\vspace*{-0.7in}
\pause

Orient each $T_j$ positively:

\medskip

The inner sides cancel.

\medskip
\pause

So

\medskip

$\displaystyle
\quad
c = 
\abs{\int_{\partial T} f(z) \, dz }
=
\abs{\int_{\partial T_1} f(z) \, dz 
+
\int_{\partial T_2} f(z) \, dz 
+
\int_{\partial T_3} f(z) \, dz 
+
\int_{\partial T_4} f(z) \, dz } . $

\medskip
\pause

So for some triangle $T_j$, the integral is at least $\frac{c}{4}$.

\medskip
\pause

Label that subtriangle $T^1 = T_j$ and
$\displaystyle
\quad
\abs{\int_{\partial T^1} f(z) \, dz } \geq \frac{c}{4} .$

\medskip
\pause

Cut $T^1$ into subtriangles $T_1^1$, $T_2^1$, $T_3^1$, $T_4^1$.
\pause
Integral over some $\partial T_j^1$ is at least $\frac{c}{4^2}$, so label it $T^2$.

\pause

Rinse and repeat.

\pause
\medskip

After $k$ iterations for the $k$\textsuperscript{th} triangle $T^k$,
$\displaystyle
\quad
\abs{\int_{\partial T^k} f(z) \, dz } \geq \frac{c}{4^k}.$
\end{frame}

\begin{frame}
$T^k \subset T^{k-1} \subset \cdots \subset T$
and in each step the triangle is exactly half the size (similar triangles):
\pause
\[
\operatorname{diam}(T^k) =
\frac{1}{2} \operatorname{diam}(T^{k-1})
=
\frac{1}{2^k} \operatorname{diam}(T) .
\]
\pause
The triangles are compact $\Rightarrow$ the intersection is
nonempty.

\pause
The diameter goes to zero $\Rightarrow$ the intersection is a single point:
\[
\{ z_0 \} = \bigcap_{k=1}^\infty T^k .
\]
\pause

Write
$f(z) = f(z_0) + \alpha (z-z_0) + g(z)$ for some $\alpha \in \C$.

\medskip
\pause

Were $f$ complex differentiable at $z_0$, then for some $\alpha$,
$\frac{g(z)}{z-z_0}$ would go to zero as $z \to z_0$.

\medskip
\pause

We will prove
$\frac{g(z)}{z-z_0}$ never goes to zero (no matter what $\alpha$ is).

\medskip
\pause

Fix $\alpha$.  \pause  If $g(z_0) \not= 0$, we are done.

\medskip
\pause

So assume $g(z_0) = 0$.  \pause
Cauchy's theorem for polynomials says
\[
\int_{\partial T^k} f(z) \, dz =
\int_{\partial T^k} \bigl( f(z_0) + \alpha (z-z_0) + g(z) \bigr) \, dz =
\int_{\partial T^k} g(z) \, dz .
\]
\end{frame}

\begin{frame}
\[
\frac{c}{4^k} \leq
\abs{
\int_{\partial T^k} f(z) \, dz
}
\pause
=
\abs{
\int_{\partial T^k} g(z) \, dz 
}
\pause
\leq
\int_{\partial T^k} \sabs{g(z)} \, \sabs{dz} .
\]
\pause
Let $\ell$ be the length of $\partial T$.

\medskip
\pause

The length of $\partial T^k$ is $\frac{\ell}{2^k}$,
\pause
by the mean value theorem for integrals,
$\exists$ $z_k \in \partial T^k$ such that
\[
\sabs{g(z_k)} = 
\frac{2^k}{\ell}
\int_{\partial T^k} \sabs{g(z)} \, \sabs{dz} .
\]
\pause
$z_k \not= z_0$ as $g(z_0)=0$.

\medskip
\pause

Let $d = \operatorname{diam}(T)$.  Then
$\sabs{z_k-z_0} \leq \frac{d}{2^k}$ and
\pause
\[
\abs{\frac{g(z_k)}{z_k-z_0}}
\pause
\geq
\frac{2^k\sabs{g(z_k)}}{d}
\pause
=
\frac{4^k}{d \ell}
\int_{\partial T^k} \sabs{g(z)} \, \sabs{dz}
\pause
\geq
\frac{4^k}{d \ell}
\frac{c}{4^k}
\pause
= \frac{c}{d \ell} .
\]
\pause
Since $z_k \to z_0$, we have that $\frac{g(z)}{z-z_0}$ does not 
go to zero as $z \to z_0$.
\pause
\medskip

So $f$ is not complex differentiable at $z_0$.
\qed
\end{frame}

\begin{frame}
A useful version of this result is the following exercise:

\medskip

\textbf{Exercise:}
Suppose $T \subset \C$ is a triangle and $f \colon T \to \C$ a
continuous function whose restriction to the interior of $T$ is holomorphic.
Prove that $\int_{\partial T} f(z) \, dz = 0$.

\medskip
\pause

Hint: Passing some sort of limit under the integral is required.

\end{frame}

\end{document}
