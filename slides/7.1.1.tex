\documentclass[10pt,aspectratio=169]{beamer}

% All the boilerplate is in ccaslides.sty
% Note that this also pulls in a custom vogtwidebar.sty
\usepackage{ccaslides}

\author{Ji\v{r}\'i Lebl}

\institute[OSU]{%
Departemento pri Matematiko de Oklahoma {\^S}tata Universitato}

\title{Cultivating Complex Analysis:\\%
Harmonic functions\\%
Real and imaginary parts of holomorphic functions (7.1.1)}

\date{}

\begin{document}

\begin{frame}
\titlepage
\end{frame}

\begin{frame}
\begin{definition}
Let $U \subset \C$ be open.
A twice continuously differentiable $f \colon U \to \R$ is
\emph{harmonic} if
\begin{equation*}
\nabla^2 f =
\frac{\partial^2 f}{\partial x^2} +
\frac{\partial^2 f}{\partial y^2} = 0
\quad \text{ on $U$.}
\end{equation*}
\end{definition}

\pause

$\nabla^2$ (sometimes written $\Delta$) is the \emph{Laplacian}.

\pause
\medskip

Convenient to write using Wirtinger operators:
\begin{equation*}
4
\frac{\partial^2}{\partial \bar{z}\partial z} \, f =
4
\left[
\frac{1}{2}
\left(
\frac{\partial}{\partial x} + i
\frac{\partial}{\partial y}
\right)
\right]
\left[
\frac{1}{2}
\left(
\frac{\partial}{\partial x} - i
\frac{\partial}{\partial y}
\right)
\right]
f
\pause
=
\left[
\frac{\partial^2}{\partial x^2} +
\frac{\partial^2}{\partial y^2}
\right]
f
=
\nabla^2 f .
\end{equation*}

\pause
\medskip

$f$ is harmonic \wiffif $\dfrac{\partial f}{\partial z}$ is holomorphic.

\end{frame}

\begin{frame}
Suppose $f \colon U \to \R$ is harmonic.
\quad
\pause
Find (locally) a primitive $g$ of the holomorphic
$\dfrac{\partial f}{\partial z}$.

%\medskip
\pause

Let $c(z) = f(z)-g(z)$
\pause
\wthus
$c$ is $C^2$ ~and~ $\dfrac{\partial c}{\partial z}
=
\dfrac{\partial f}{\partial z}
-
\dfrac{\partial g}{\partial z}
\pause
=
\dfrac{\partial f}{\partial z}
-
\dfrac{\partial f}{\partial z}
= 0$.

\medskip
\pause

Let $h = \bar{c}$
\pause
\wthus
$\displaystyle
\frac{\partial h}{\partial \bar{z}}
=
\frac{\partial \bar{c}}{\partial \bar{z}}
=
\overline{
\frac{\partial c}{\partial z}
}
=
0$
\pause
\wthus
$h$ is holomorphic.

\pause
\medskip
\thus \quad
$f(z) = g(z) + \overline{h(z)}$,
\quad
$g$ and $h$ holomorphic.

\medskip
\pause

Let $\varphi = g+h$.
\pause  \quad As $f$ is real-valued,
\[
f(z) = \Re f(z)
\pause
=
\frac{g(z) + \overline{h(z)} + \overline{g(z)}+h(z)}{2}
=
\frac{g(z) + h(z) + \overline{g(z)+h(z)}}{2}
\pause
=
\Re \varphi(z) .
\]
\pause
\textbf{Note:} This works locally, or in a simply
connected $U$ where we can find the primitive $g$.

\pause

\begin{proposition}
Let $U \subset \C$ be a simply connected domain and $f \colon U \to \R$ a
harmonic function.  Then there exists a holomorphic $\varphi \colon
U \to \C$ such that $f = \Re \varphi$.
\end{proposition}

\pause

Similarly $f$ is the imaginary part of some holomorphic function.

\end{frame}

\begin{frame}
Conversely suppose
\qquad
$
f(z) = \Re \varphi(z) =
\tfrac{1}{2}\bigl( \varphi(z) + \overline{\varphi(z)} \bigr)$
\qquad ($\varphi$ holomorphic).

\medskip
\pause

$\displaystyle
\nabla^2 =
4 \frac{\partial^2}{\partial \bar{z} \partial z}
=
4 \frac{\partial^2}{\partial z \partial \bar{z}}
$
\pause
\wthus
$\displaystyle
\nabla^2 f =
4 \frac{\partial^2}{\partial \bar{z} \partial z}
\left(
\frac{1}{2}\bigl( \varphi(z) + \overline{\varphi(z)} \bigr)
\right)
=
2
\left(
\frac{\partial}{\partial z}
\left(
\frac{\partial \varphi}{\partial \bar{z}}
\right)+
\frac{\partial}{\partial \bar{z}}
\left(
\frac{\partial \bar{\varphi}}{\partial z}
\right)
\right)
=
0$.

\medskip

\pause
We proved:
\begin{proposition}
Let $U \subset \C$ be open and $f \colon U \to \R$ a function.
\begin{enumerate}[(i)]
\item
The function $f$ is harmonic if and only if
for every $p \in U$ there exists an open neighborhood $V$ of $p$ and a
holomorphic $\varphi \colon V \to \C$ such that $f = \Re \varphi$
on $V$.
\pause
\item
The function $f$ is harmonic if and only if
for every $p \in U$, there exists a power series expansion
\begin{equation*}
f(z) =
c_0 +
\sum_{n=1}^\infty c_n {(z-p)}^n + \bar{c}_n {(\overline{z-p})}^n
\end{equation*}
converging uniformly absolutely
on every closed disc $\overline{\Delta_r(p)} \subset U$.
\end{enumerate}
\end{proposition}

\end{frame}

\begin{frame}
A quick corollary:

\begin{proposition}
If $U \subset \C$ is open and $f \colon U \to \R$ is harmonic,
then $f$ is infinitely (real) differentiable.
\end{proposition}

\textbf{Proof:} Holomorphic functions are infinitely differentiable.

\end{frame}

\begin{frame}
\begin{definition}
Let $U \subset \C$ be open and $f \colon U \to \R$ harmonic.
If $g \colon U \to \R$ is harmonic and $f + i g$ is holomorphic,
then $g$ is called the \emph{harmonic conjugate} of $f$.
\end{definition}

\pause

Every harmonic $f$ on a simply connected domain has a harmonic conjugate.

\medskip
\pause

On $\C \setminus \{ 0 \}$,
$z \mapsto \log \sabs{z}$ is harmonic, but
fails to have a harmonic conjugate.

\medskip
\pause

If it did have a harmonic conjugate
then $\log$ would have a branch in $\C \setminus \{0\}$.

\pause
Which follows from:

\begin{proposition}
If $U \subset \C$ is a domain $f \colon U \to \R$ is harmonic
and $g_1$ and $g_2$ are two harmonic conjugates of $f$,
then $g_1 = g_2 + C$ for some $C \in \R$.
\end{proposition}

\pause
\textbf{Proof:}
$\dfrac{(f + i g_1) - (f + i g_2)}{i} =  g_1-g_2$
is holomorphic, real-valued ~\thus~ constant.

\end{frame}

\begin{frame}
The real and imaginary parts of a holomorphic function are harmonic.

\pause
\medskip

The modulus $\sabs{f(z)}$ is not.

\pause
\medskip

But $\log \sabs{f(z)}$ is harmonic (where $f$ is nonzero).

\pause

\begin{proposition}
Suppose $U \subset \C$ is open, $f \colon U \to \C$ is holomorphic
and never zero.  Then
\begin{equation*}
z \mapsto \log \sabs{f(z)}
\end{equation*}
is harmonic.
\end{proposition}

\textbf{Proof:} Exercise.
\end{frame}

\begin{frame}
\textbf{Exercise:}
Suppose $U \subset \C$ is a simply connected domain
and $f \colon U \to \R$ harmonic.  Prove there exists
a holomorphic $\varphi \colon U \to \C$ such that
$f(z) = \log \sabs{\varphi(z)}$.

\medskip
\pause

\textbf{Exercise:}
Let $U,V \subset \C$ be open sets and
$f \colon U \to V$ be holomorphic. Prove:

a)
If $g \colon V \to \R$ is harmonic, then
$g \circ f$ is harmonic.

b)
If $f$ is a biholomorphism,
then $g \colon V \to \R$ is harmonic if and only if
$g \circ f$ is harmonic.

\medskip
\pause

\textbf{Exercise:}
Prove the Liouville theorem for
harmonic functions:  If $f \colon \C \to \R$ is harmonic
and nonnegative, then $f$ is constant.

\bigskip
\pause

``bounded'' for holomorphic functions
\quad $\leftrightarrow$ \quad
``nonnegative'' for harmonic functions:

\medskip
\pause

If $f$ is bounded and holomorphic,

then ~$\log \sabs{f(z)+M}$~ or ~$\Re f(z) + M$~
is nonnegative for large enough $M$.

\medskip
\pause

Conversely, if ~$\log \sabs{f(z)} \geq 0$,~
then ~$\dfrac{1}{f(z)}$~ is bounded,

and if ~$\Re f(z) \geq 0$,~
then ~$\dfrac{f(z)-1}{f(z)+1}$~ is bounded.
%
%(composing $f$ with an LFT taking the
%right half-plane to the disc).



\end{frame}

\begin{frame}
\textbf{Remark:}
Writing
$\displaystyle
\frac{\partial^2}{\partial x^2}+ \frac{\partial^2}{\partial y^2} = 
4 \frac{\partial^2}{\partial \bar{z} \partial z}$
so that we can integrate twice may sound familiar.

\medskip
\pause

It is like the D'Alembert solution of the one-dimensional wave equation.

\medskip
\pause

The wave operator is
(using $(x,t)$ for tradition's sake):
\[
\frac{\partial^2}{\partial t^2}- \frac{\partial^2}{\partial x^2}
\pause
=
\left[ \frac{\partial}{\partial t}- \frac{\partial}{\partial x}
\right]
\left[ \frac{\partial}{\partial t}+ \frac{\partial}{\partial x}
\right].
\]

\medskip
\pause

Write $\mu = x+t$ and $\eta = x-t$ (characteristic coordinates), then
\[
\frac{\partial^2}{\partial t^2}- \frac{\partial^2}{\partial x^2} =
-4 \frac{\partial^2}{\partial \eta \partial \mu} .
\]
\pause
A solution $f$ to the wave equation is
\[
f(x,t) = A(\mu) + B(\eta) = A(x+t)+B(x-t).
\]
\pause
Two waves travelling in opposite directions.

\end{frame}

\end{document}
