\documentclass[10pt,aspectratio=169]{beamer}

% All the boilerplate is in ccaslides.sty
% Note that this also pulls in a custom vogtwidebar.sty
\usepackage{ccaslides}

\author{Ji\v{r}\'i Lebl}

\institute[OSU]{%
Departemento pri Matematiko de Oklahoma {\^S}tata Universitato}

\title{Cultivating Complex Analysis:\\%
Residue theorem, applications (5.3 part 2)}

\date{}

\begin{document}

\begin{frame}
\titlepage
\end{frame}

\begin{frame}
Recall that the residue theorem says that for a finite $S \subset U$,
$\Gamma$ a cycle in $U \setminus S$ homologous to zero in $U$
and $f \colon U \setminus S \to \C$ holomorphic, then
\[
\frac{1}{2\pi i} \int_{\Gamma} f(z) \, dz = \sum_{p \in S} n(\Gamma;p) \operatorname{Res}(f;p) ,
\]
where the residue $\operatorname{Res}(f;p)$ is the $c_{-1}$ term of the
Laurent series at $p$.

\medskip
\pause

$c_{-1}$ has a formula in terms of an integral.
\pause
So how is the residue theorem useful if it takes an integral and replaces
it with integrals?

\medskip
\pause

That's because we have lots of other tricks to compute $c_{-1}$.
We'll go over a few.

\end{frame}

\begin{frame}
\begin{proposition}
Suppose $f$ is holomorphic in an open neighborhood of $p$ and $g$ is holomorphic
with an isolated singularity at $p$, then
$\operatorname{Res}(f+g;p) = \operatorname{Res}(g;p)$.
\end{proposition}

\pause

\textbf{Proof:}
For a small enough $\epsilon > 0$,

\medskip
\quad
$\displaystyle
\operatorname{Res}(f+g;p)
=
\frac{1}{2\pi i}
\int_{\partial \Delta_{\epsilon}(p)}
\bigl(f(z)+g(z)\bigr) \, dz
$

\medskip
\pause

\hfill
\hfill
\hfill
$\displaystyle
=
\cancelto{0}{
\frac{1}{2\pi i}
\int_{\partial \Delta_{\epsilon}(p)}
f(z) \, dz
}
+
\frac{1}{2\pi i}
\int_{\partial \Delta_{\epsilon}(p)}
g(z) \, dz
\pause
=
\operatorname{Res}(g;p)$.
\qed
\end{frame}

\begin{frame}


\begin{proposition}
Suppose $f$ has a pole at $p$.
If $p$ is a simple pole of $f$, then
\quad
$\displaystyle
\operatorname{Res}(f;p) = \lim_{z\to p} (z-p) f(z)$.

\pause
\medskip

If $p$ is a pole of $f$ of order $k$, then
\quad
$\displaystyle
\operatorname{Res}(f;p) = \frac{1}{(k-1)!} \lim_{z\to p}
\frac{d^{k-1}}{dz^{k-1}}\bigl[ (z-p)^{k} f(z) \bigr]$.
\end{proposition}

\pause

\textbf{Proof:} Exercise.

\pause

\begin{proposition}
Suppose $f(z) = \frac{h(z)}{g(z)}$ where $h$ and $g$ are holomorphic
at $p$ and $g$ has a simple zero at $p$. % (so $f$ has a simple pole at $p$).
Then
\begin{equation*}
\operatorname{Res}(f;p) = \frac{h(p)}{g'(p)} .
\end{equation*}
\end{proposition}

\pause

\textbf{Proof:} Exercise.

\end{frame}

\begin{frame}

\textbf{Example:}
\begin{equation*}
\int_{-\infty}^\infty \frac{1}{1+x^2} \, dx .
\end{equation*}
OK\@, this one is easy to compute by classical calculus, but let us ignore
that fact for the sake of the simplicity of the example.

Define the cycle $\Gamma_r = [-r,r] + \gamma_r$, where $\gamma_r(t) =
re^{it}$ for $t \in [0,\pi]$, that is, $\gamma_r$ is the upper semi-circle
of the circle of radius $r$ centered at the origin oriented counterclockwise.
See {fig:rhalfcircle}.

\scalebox{0.8}{
\subimport*{../figures/}{rhalfcircle.pdf_t}
}

The cycle $\Gamma_r$.

We ``complexify''
$\frac{1}{1+x^2}$ to make it $\frac{1}{1+z^2}$.
That's just a fancy way of saying we are going to plug complex numbers
into a real formula that happens to also work for complex numbers.
By partial fractions
\begin{equation*}
\frac{1}{1+z^2} = \frac{1}{(z+i)(z-i)} =
\frac{i}{2} \frac{1}{z+i} - 
\frac{i}{2} \frac{1}{z-i} .
\end{equation*}
There are isolated singularities at $\pm i$, both simple poles.  The cycle
$\Gamma_r$ goes around $i$ once, so $n(\Gamma_r;i) = 1$,
but not around $-i$, that is, $n(\Gamma_r;-i) = 0$.
So we only need to compute
the residue around $i$.  We can use any one of the techniques:
\begin{equation*}
\operatorname{Res}\left(\frac{1}{1+z^2};i\right) =
\operatorname{Res}\left(
\frac{-i}{2} \frac{1}{z-i};
i\right) = \frac{-i}{2} ,
\qquad
\operatorname{Res}\left(\frac{1}{1+z^2};i\right) =
\lim_{z \to i} \frac{z-i}{1+z^2}
=
\frac{1}{2i} = \frac{-i}{2}.
\end{equation*}

We compute,
\begin{equation*}
\pi 
=
2 \pi i \operatorname{Res}\left(\frac{1}{1+z^2};i\right) =
\int_{\Gamma_r} \frac{1}{1+z^2} \, dz
=
\int_{-r}^r \frac{1}{1+x^2} \, dx
+
\int_{\gamma_r} \frac{1}{1+z^2} \, dz .
\end{equation*}
Let us find the limit as $r \to \infty$ of the second term.
Assume $r > 1$.
The length of $\gamma_r$ is $r\pi$,
and on $\gamma_r$,
$\sabs{1+z^2} \geq r^2-1$.  So
\begin{equation*}
\abs{
\int_{\gamma_r} \frac{1}{1+z^2} \, dz 
}
\leq
r \pi \frac{1}{r^2-1}
\quad\underset{\text{as } r \to \infty}{\to}\quad 0 .
\end{equation*}
Hence,
\begin{equation*}
\int_{-\infty}^\infty
\frac{1}{1+x^2} \, dx
=
\lim_{r\to \infty} \int_{-r}^r 
\frac{1}{1+x^2} \, dx
= \pi .
\end{equation*}
Why taking the symmetric limit is sufficient to compute the double improper
integral is left to the reader.  After all, usually one has to
take two independent limits.


\medskip
\pause

\textbf{Exercise:}
Rigorously prove that in the example above
$n(\Gamma_r;i) = 1$ and
$n(\Gamma_r;i) = 0$.

\end{frame}

\begin{frame}

Another application to real integrals is to recognize 
a path integral.  For example, integrals of trigonometric functions
are often integrals over the unit circle.  On the unit
circle, $\bar{z} = \nicefrac{1}{z}$.  So if $z=e^{i\theta}$,
$\cos \theta = \Re z = \frac{z+1/z}{2}$ and
$\sin \theta = \Im z = \frac{z-1/z}{2i}$.

\textbf{Example:}
If $c > 1$, then
\begin{equation*}
\int_0^{2\pi} \frac{1}{c+\cos \theta} \, d\theta 
=
\int_{\partial \D} \frac{1}{c+\frac{z+1/z}{2}} \frac{1}{iz} \, dz
=
-2i
\int_{\partial \D} \frac{1}{z^2 + 2cz + 1} \, dz
.
\end{equation*}
The function $\frac{1}{z^2 + 2cz + 1}$ has two poles $-c \pm \sqrt{c^2-1}$,
one inside and one outside the unit circle.  Thus
\begin{equation*}
\int_0^{2\pi} \frac{1}{c+\cos \theta} \, d\theta 
=
(-2i)
(2 \pi i)
\operatorname{Res}
\left(\frac{1}{z^2 + 2cz + 1}; -c+\sqrt{c^2-1}\right)
=
\frac{2\pi}{\sqrt{c^2-1}}
.
\end{equation*}

\end{frame}

\begin{frame}

A very common application of the residue theorem is to compute
inverse Laplace transforms.
The \emph{Mellin's inversion formula} says that given a transform $F(s)$,
the original $f(t)$ is given by
\begin{equation*}
f(t) = \sL^{-1}\bigl[F(s)\bigr] =
\frac{1}{2 \pi i }
\lim_{r \to \infty}
\int_{c-ir}^{c+ir}
e^{st}F(s) \, ds
\end{equation*}
for some $c \in \R$ (usually $c \geq 0$) is the inverse.

\medskip
\pause

As an exercise, try your hand at computing a few.  Say

\medskip

\quad$\displaystyle \sL^{-1}\left[ \frac{1}{s(s+1)} \right]$,
\qquad \text{or} \qquad
$\displaystyle \sL^{-1}\left[ \frac{s^2}{(s+2)^2(s^2+1)} \right]$.

\medskip
\pause
Hint: Pick the correct vertical line (pick a $c$) and an arc that goes
around all the poles.
\end{frame}

\end{document}
