\documentclass[10pt,aspectratio=169]{beamer}

% All the boilerplate is in ccaslides.sty
% Note that this also pulls in a custom vogtwidebar.sty
\usepackage{ccaslides}

\author{Ji\v{r}\'i Lebl}

\institute[OSU]{%
Departemento pri Matematiko de Oklahoma {\^S}tata Universitato}

\title{Cultivating Complex Analysis:\\%
Residue theorem, applications (5.3 part 2)}

\date{}

\begin{document}

\begin{frame}
\titlepage
\end{frame}

\begin{frame}
The residue theorem is supposed to be useful in computing line integrals.
But at first glance it seems ridiculous.  How does one compute $c_{-1}$?  By
an integral.  Well how does that help then?  It helps because there are
easier ways to compute $c_{-1}$ than by the line integral.  The first one is
almost criminally trivial, but it may be good to emphasize all of them
by making them propositions.

\begin{proposition}
Suppose $f$ is holomorphic in an open neighborhood of $p$ and $g$ is holomorphic
with an isolated singularity at $p$, then
$\operatorname{Res}(f+g;p) = \operatorname{Res}(g;p)$.
\end{proposition}

\begin{proof}
For a small enough $\epsilon > 0$,
\begin{equation*}
\begin{split}
\operatorname{Res}(f+g;p)
& =
\frac{1}{2\pi i}
\int_{\partial \Delta_{\epsilon}(p)}
\bigl(f(z)+g(z)\bigr) \, dz
\\
& =
\cancelto{0}{
\frac{1}{2\pi i}
\int_{\partial \Delta_{\epsilon}(p)}
f(z) \, dz
}
+
\frac{1}{2\pi i}
\int_{\partial \Delta_{\epsilon}(p)}
g(z) \, dz
=
\operatorname{Res}(g;p) . \qedhere
\end{split}
\end{equation*}
\end{proof}

\begin{proposition}
Suppose $f$ has a pole at $p$.
If $p$ is a simple pole of $f$, then
\begin{equation*}
\operatorname{Res}(f;p) = \lim_{z\to p} (z-p) f(z) .
\end{equation*}
More generally, if $p$ is a pole of $f$ of order $k$, then
\begin{equation*}
\operatorname{Res}(f;p) = \frac{1}{(k-1)!} \lim_{z\to p}
\frac{d^{k-1}}{dz^{k-1}}\bigl[ (z-p)^{k} f(z) \bigr] .
\end{equation*}
\end{proposition}

\begin{exercise}
Prove the proposition.
\end{exercise}

\begin{proposition}
Suppose $f(z) = \frac{h(z)}{g(z)}$ where $h$ and $g$ are holomorphic
at $p$ and $g$ has a simple zero at $p$ (so $f$ has a simple pole at $p$).
Then
\begin{equation*}
\operatorname{Res}(f;p) = \frac{h(p)}{g'(p)} .
\end{equation*}
\end{proposition}

\textbf{Proof:} Exercise.

\medskip

A common application of the residue theorem is to compute certain real
integrals that are difficult by classical calculus.  Let us compute
a couple of examples.  They will also show you how one often computes
residues.

\begin{example}
\begin{equation*}
\int_{-\infty}^\infty \frac{1}{1+x^2} \, dx .
\end{equation*}
OK\@, this one is easy to compute by classical calculus, but let us ignore
that fact for the sake of the simplicity of the example.

Define the cycle $\Gamma_r = [-r,r] + \gamma_r$, where $\gamma_r(t) =
re^{it}$ for $t \in [0,\pi]$, that is, $\gamma_r$ is the upper semi-circle
of the circle of radius $r$ centered at the origin oriented counterclockwise.
See {fig:rhalfcircle}.

\subimport*{../figures/}{rhalfcircle.pdf_t}

The cycle $\Gamma_r$.

We ``complexify''
$\frac{1}{1+x^2}$ to make it $\frac{1}{1+z^2}$.
That's just a fancy way of saying we are going to plug complex numbers
into a real formula that happens to also work for complex numbers.
By partial fractions
\begin{equation*}
\frac{1}{1+z^2} = \frac{1}{(z+i)(z-i)} =
\frac{i}{2} \frac{1}{z+i} - 
\frac{i}{2} \frac{1}{z-i} .
\end{equation*}
There are isolated singularities at $\pm i$, both simple poles.  The cycle
$\Gamma_r$ goes around $i$ once, so $n(\Gamma_r;i) = 1$,
but not around $-i$, that is, $n(\Gamma_r;-i) = 0$.
So we only need to compute
the residue around $i$.  We can use any one of the techniques:
\begin{equation*}
\operatorname{Res}\left(\frac{1}{1+z^2};i\right) =
\operatorname{Res}\left(
\frac{-i}{2} \frac{1}{z-i};
i\right) = \frac{-i}{2} ,
\qquad
\operatorname{Res}\left(\frac{1}{1+z^2};i\right) =
\lim_{z \to i} \frac{z-i}{1+z^2}
=
\frac{1}{2i} = \frac{-i}{2}.
\end{equation*}

We compute,
\begin{equation*}
\pi 
=
2 \pi i \operatorname{Res}\left(\frac{1}{1+z^2};i\right) =
\int_{\Gamma_r} \frac{1}{1+z^2} \, dz
=
\int_{-r}^r \frac{1}{1+x^2} \, dx
+
\int_{\gamma_r} \frac{1}{1+z^2} \, dz .
\end{equation*}
Let us find the limit as $r \to \infty$ of the second term.
Assume $r > 1$.
The length of $\gamma_r$ is $r\pi$,
and on $\gamma_r$,
$\sabs{1+z^2} \geq r^2-1$.  So
\begin{equation*}
\abs{
\int_{\gamma_r} \frac{1}{1+z^2} \, dz 
}
\leq
r \pi \frac{1}{r^2-1}
\quad\underset{\text{as } r \to \infty}{\to}\quad 0 .
\end{equation*}
Hence,
\begin{equation*}
\int_{-\infty}^\infty
\frac{1}{1+x^2} \, dx
=
\lim_{r\to \infty} \int_{-r}^r 
\frac{1}{1+x^2} \, dx
= \pi .
\end{equation*}
Why taking the symmetric limit is sufficient to compute the double improper
integral is left to the reader.  After all, usually one has to
take two independent limits.
\end{example}

\begin{exercise}
Rigorously prove that in the example above
$n(\Gamma_r;i) = 1$ and
$n(\Gamma_r;i) = 0$.
\end{exercise}

Another application to real integrals is to recognize 
a path integral.  For example, integrals of trigonometric functions
are often integrals over the unit circle.  On the unit
circle, $\bar{z} = \nicefrac{1}{z}$.  So if $z=e^{i\theta}$,
$\cos \theta = \Re z = \frac{z+1/z}{2}$ and
$\sin \theta = \Im z = \frac{z-1/z}{2i}$.

\begin{example}
If $c > 1$, then
\begin{equation*}
\int_0^{2\pi} \frac{1}{c+\cos \theta} \, d\theta 
=
\int_{\partial \D} \frac{1}{c+\frac{z+1/z}{2}} \frac{1}{iz} \, dz
=
-2i
\int_{\partial \D} \frac{1}{z^2 + 2cz + 1} \, dz
.
\end{equation*}
The function $\frac{1}{z^2 + 2cz + 1}$ has two poles $-c \pm \sqrt{c^2-1}$,
one inside and one outside the unit circle.  Thus
\begin{equation*}
\int_0^{2\pi} \frac{1}{c+\cos \theta} \, d\theta 
=
(-2i)
(2 \pi i)
\operatorname{Res}
\left(\frac{1}{z^2 + 2cz + 1}; -c+\sqrt{c^2-1}\right)
=
\frac{2\pi}{\sqrt{c^2-1}}
.
\end{equation*}
\end{example}

\begin{exercise}
For all integers $n \in \Z$, compute
\begin{equation*}
\int_{\partial \D} z^n e^{1/z} \, dz .
\end{equation*}
\end{exercise}

\begin{exercise}
Compute using the residue theorem
(hint: $\cos(3x) = \Re e^{i3x}$):
\smallskip
\begin{expartshor}{2}
\item
$\displaystyle \int_{-\infty}^\infty \frac{1}{{(x^2+1)}^2} \, dx$,
\item
$\displaystyle \int_{-\infty}^\infty \frac{\cos(3x)}{x^4+1} \, dx$.
\end{expartshor}
\end{exercise}

\begin{exercise}[Inverse Laplace transform]
A common integral computed via the Residue theorem is
the inverse Laplace transform via \emph{Mellin's inversion formula}. Given $F(s)$,
\begin{equation*}
f(t) = \sL^{-1}\bigl[F(s)\bigr] =
\frac{1}{2 \pi i }
\lim_{r \to \infty}
\int_{c-ir}^{c+ir}
e^{st}F(s) \, ds
\end{equation*}
for some $c \in \R$ (usually $c \geq 0$) is the inverse.  Compute
(using the residue theorem):
\smallskip
\begin{expartshor}{2}
\item
$\sL^{-1}\left[ \frac{1}{s(s+1)} \right]$,
\item
$\sL^{-1}\left[ \frac{s^2}{(s+2)^2(s^2+1)} \right]$.
\end{expartshor}
\smallskip
\noindent
Hint: Pick the correct vertical line (pick a $c$) and an arc that goes
around all the poles.
\end{exercise}

\begin{exercise}
\pagebreak[2]
Compute (using the residue theorem):
\smallskip
\begin{expartshor}{2}
\item
$\displaystyle \int_0^{2\pi} \frac{\cos \theta}{2+\cos \theta} \, d\theta$,
\item
$\displaystyle \int_0^{\pi} \frac{\sin^2 \theta}{2+\cos \theta} \, d\theta$.
\end{expartshor}
\end{exercise}

\begin{exercise}
Suppose that $r > 1$, $f \colon \Delta_r(0) \setminus \{ 1 \} \to \C$ is
holomorphic, and suppose $f$ has a simple pole with $\operatorname{Res}(f;1) = 1$.
If the power series for $f$ at 0 is $\sum_{n=0}^\infty c_n z^n$, show that
$\lim_{n\to \infty} c_n$ exists and compute what it is.  Hint: Try
subtracting the pole away.
\end{exercise}

\begin{exercise}
Suppose $f$ is holomorphic on $U = \{ z \in \C : \sabs{z} > R \}$ for 
some $R > 0$.
Define the residue of $f$ at $\infty$,
$\operatorname{Res}(f;\infty)$, to be the residue
of $g(z) = -z^{-2} f(z^{-1})$ at $0$.
\begin{exparts}
\item
Prove that for all $r > R$,
\begin{equation*}
\operatorname{Res}(f;\infty) = \frac{-1}{2\pi i} \int_{\partial \Delta_r(0)}
f(z) \, dz .
\end{equation*}
That is, going around a circle in reverse is going around infinity rather
than the center (if what we are ``going around'' is defined to be whatever
is on our left).
\item
If $f$ is holomorphic on $\C$ except for finitely many isolated
singularities.  Prove that the sum of all residues of $f$ including the
residue at $\infty$ is zero.
\end{exparts}
\end{exercise}

\begin{exercise}
Use the function $f(z) = \frac{e^{-z^2/2}}{1+e^{-\sqrt{\pi}(1+i)z}}$ and
the rectangular path with vertices $-r$, $r$, $r+i\sqrt{\pi}$,
and $-r+i\sqrt{\pi}$
to compute the integral $\int_{-\infty}^\infty e^{-x^2/2} \, dx$.\footnote{%
This nifty solution is due to H.\ Kneser. The tricky bit with using the
residue theorem is that $e^{-z^2/2}$ has no singularities itself,
so one has to find a function that does.}
\end{exercise}
\end{frame}

\end{document}
