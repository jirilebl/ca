\documentclass[10pt,aspectratio=169]{beamer}

% All the boilerplate is in ccaslides.sty
% Note that this also pulls in a custom vogtwidebar.sty
\usepackage{ccaslides}

\author{Ji\v{r}\'i Lebl}

\institute[OSU]{%
Departemento pri Matematiko de Oklahoma {\^S}tata Universitato}

\title{Cultivating Complex Analysis:\\%
Residue theorem, applications (5.3 part 2)}

\date{}

\begin{document}

\begin{frame}
\titlepage
\end{frame}

\begin{frame}
Recall that the residue theorem says that for a finite $S \subset U$,
$\Gamma$ a cycle in $U \setminus S$ homologous to zero in $U$
and $f \colon U \setminus S \to \C$ holomorphic, then
\[
\frac{1}{2\pi i} \int_{\Gamma} f(z) \, dz = \sum_{p \in S} n(\Gamma;p) \operatorname{Res}(f;p) ,
\]
where the residue $\operatorname{Res}(f;p)$ is the $c_{-1}$ term of the
Laurent series at $p$.

\medskip
\pause

$c_{-1}$ has a formula in terms of an integral.
\pause
So how is the residue theorem useful if it takes an integral and replaces
it with integrals?

\medskip
\pause

That's because we have lots of other tricks to compute $c_{-1}$.
We'll go over a few.

\end{frame}

\begin{frame}
\begin{proposition}
Suppose $f$ is holomorphic in an open neighborhood of $p$ and $g$ is holomorphic
with an isolated singularity at $p$, then
$\operatorname{Res}(f+g;p) = \operatorname{Res}(g;p)$.
\end{proposition}

\pause

\textbf{Proof:}
For a small enough $\epsilon > 0$,

\medskip
\quad
$\displaystyle
\operatorname{Res}(f+g;p)
=
\frac{1}{2\pi i}
\int_{\partial \Delta_{\epsilon}(p)}
\bigl(f(z)+g(z)\bigr) \, dz
$

\medskip
\pause

\hfill
\hfill
\hfill
$\displaystyle
=
\cancelto{0}{
\frac{1}{2\pi i}
\int_{\partial \Delta_{\epsilon}(p)}
f(z) \, dz
}
+
\frac{1}{2\pi i}
\int_{\partial \Delta_{\epsilon}(p)}
g(z) \, dz
\pause
=
\operatorname{Res}(g;p)$.
\qed
\end{frame}

\begin{frame}


\begin{proposition}
Suppose $f$ has a pole at $p$.
If $p$ is a simple pole of $f$, then
\quad
$\displaystyle
\operatorname{Res}(f;p) = \lim_{z\to p} (z-p) f(z)$.

\pause
\medskip

If $p$ is a pole of $f$ of order $k$, then
\quad
$\displaystyle
\operatorname{Res}(f;p) = \frac{1}{(k-1)!} \lim_{z\to p}
\frac{d^{k-1}}{dz^{k-1}}\bigl[ (z-p)^{k} f(z) \bigr]$.
\end{proposition}

\pause

\textbf{Proof:} Exercise.

\medskip
\pause

\begin{proposition}
Suppose $f(z) = \frac{h(z)}{g(z)}$ where $h$ and $g$ are holomorphic
at $p$ and $g$ has a simple zero at $p$. % (so $f$ has a simple pole at $p$).
Then
\begin{equation*}
\operatorname{Res}(f;p) = \frac{h(p)}{g'(p)} .
\end{equation*}
\end{proposition}

\pause

\textbf{Proof:} Exercise.

\end{frame}

\begin{frame}

\textbf{Example:}
\begin{equation*}
\int_{-\infty}^\infty \frac{1}{1+x^2} \, dx .
\end{equation*}
\pause
{\small (Yes, I know you can compute this in other ways, but it's nice and
simple.)}

\pause
\medskip

Let $\Gamma_r = [-r,r] + \gamma_r$,
~~
where $\gamma_r(t) =
re^{it}$ for $t \in [0,\pi]$

\vspace*{-8pt}
\hspace*{3in}%
\scalebox{0.8}{
\subimport*{../figures/}{rhalfcircle.pdf_t}
}

\pause
\vspace*{-1.2in}

``complexify''
$\dfrac{1}{1+x^2}$ to make it $\dfrac{1}{1+z^2}$.

\medskip
\pause

$\displaystyle
\frac{1}{1+z^2} = \frac{1}{(z+i)(z-i)} \pause =
\frac{i}{2} \frac{1}{z+i} - 
\frac{i}{2} \frac{1}{z-i}$

\medskip
\pause

Two singularities, $\pm i$, both simple poles. 

$n(\Gamma_r;i) = 1$ and $n(\Gamma_r;-i) = 0$ \quad (exercise).

\medskip
\pause

Let's compute the residue at $i$ in two ways.

\medskip
\pause

$\displaystyle
\operatorname{Res}\left(\frac{1}{1+z^2};i\right) =
\operatorname{Res}\left(
\frac{-i}{2} \frac{1}{z-i};
i\right) = \frac{-i}{2} ,
\qquad
\pause
\operatorname{Res}\left(\frac{1}{1+z^2};i\right) =
\lim_{z \to i} \frac{z-i}{1+z^2}
=
\frac{1}{2i} = \frac{-i}{2}$.

\medskip
\pause

Compute: \quad
$\displaystyle
\pi 
=
2 \pi i \operatorname{Res}\left(\frac{1}{1+z^2};i\right)
\pause
=
\int_{\Gamma_r} \frac{1}{1+z^2} \, dz
\pause
=
\int_{-r}^r \frac{1}{1+x^2} \, dx
+
\int_{\gamma_r} \frac{1}{1+z^2} \, dz$.

\end{frame}

\begin{frame}
\qquad $\displaystyle \pi 
=
\int_{-r}^r \frac{1}{1+x^2} \, dx
+
\int_{\gamma_r} \frac{1}{1+z^2} \, dz$.

\pause
\medskip

Want to take the limit $r \to \infty$. \pause Assume $r > 1$.

\medskip
\pause

Length of $\gamma_r$ is $r\pi$.  \quad \pause On $\gamma_r$, $\sabs{1+z^2} \geq r^2-1$.

\medskip
\pause

\qquad $\displaystyle
\abs{
\int_{\gamma_r} \frac{1}{1+z^2} \, dz 
}
\leq
r \pi \frac{1}{r^2-1}
\pause
\quad\underset{\text{as } r \to \infty}{\to}\quad 0 .
$

\medskip
\pause

So

\medskip
\qquad
$\displaystyle
\int_{-\infty}^\infty
\frac{1}{1+x^2} \, dx
\pause
=
\lim_{r\to \infty} \int_{-r}^r 
\frac{1}{1+x^2} \, dx
\pause
= \pi$.

\bigskip
\pause

\emph{Minor technicality:} why the symmetric limit is
sufficient?

\end{frame}

\begin{frame}

Sometimes we just recognize a path integral.

\medskip
\pause

Often, integrals of trigonometric functions are integrals over the unit
circle.

On the unit circle
$\bar{z} = \nicefrac{1}{z}$. \quad  So if $z=e^{i\theta}$, \quad
$\cos \theta = \Re z = \frac{z+1/z}{2}$ ~and~
$\sin \theta = \Im z = \frac{z-1/z}{2i}$.

\pause
\medskip

\textbf{Example:}
Suppose $c > 1$.

\medskip

\qquad$\displaystyle
\int_0^{2\pi} \frac{1}{c+\cos \theta} \, d\theta 
\pause
=
\int_{\partial \D} \frac{1}{c+\frac{z+1/z}{2}} \frac{1}{iz} \, dz
\pause
=
-2i
\int_{\partial \D} \frac{1}{z^2 + 2cz + 1} \, dz$.

\pause
\medskip

$\dfrac{1}{z^2 + 2cz + 1}$ has two poles: $-c \pm \sqrt{c^2-1}$,
one inside and one outside the unit circle.

\pause
\medskip

\qquad$\displaystyle
\int_0^{2\pi} \frac{1}{c+\cos \theta} \, d\theta 
=
(-2i)
(2 \pi i)
\operatorname{Res}
\left(\frac{1}{z^2 + 2cz + 1}; -c+\sqrt{c^2-1}\right)
\pause
=
\frac{2\pi}{\sqrt{c^2-1}}$.

\end{frame}

\begin{frame}

A common computation via the residue theorem are
inverse Laplace transforms.

\emph{Mellin's inversion formula} says that given a transform $F(s)$,
the original $f(t)$ is given by
\begin{equation*}
f(t) = \sL^{-1}\bigl[F(s)\bigr] =
\frac{1}{2 \pi i }
\lim_{r \to \infty}
\int_{c-ir}^{c+ir}
e^{st}F(s) \, ds
\end{equation*}
for some $c \in \R$ (usually $c \geq 0$) is the inverse.

\medskip
\pause

As an exercise, try your hand at computing a few.  Say

\medskip

\qquad$\displaystyle \sL^{-1}\left[ \frac{1}{s(s+1)} \right]$,
\qquad \text{or} \qquad
$\displaystyle \sL^{-1}\left[ \frac{s^2}{(s+2)^2(s^2+1)} \right]$.

\medskip
\pause
Hint: Pick the correct vertical line (pick a $c$) and an arc that goes
around all the poles.
\end{frame}

\end{document}
