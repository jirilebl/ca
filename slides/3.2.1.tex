\documentclass[10pt,aspectratio=169]{beamer}

% All the boilerplate is in ccaslides.sty
% Note that this also pulls in a custom vogtwidebar.sty
\usepackage{ccaslides}

\author{Ji\v{r}\'i Lebl}

\institute[OSU]{%
Departemento pri Matematiko de Oklahoma {\^S}tata Universitato}

\title{Cultivating Complex Analysis:\\%
Primitives, cycles, and Cauchy for derivatives (3.2.1)}

\date{}

\begin{document}

\begin{frame}
\titlepage
\end{frame}

\begin{frame}
\begin{definition}
Let $U \subset \C$ be open and $f \colon U \to \C$
a function.  A holomorphic $F \colon U \to \C$ with $f = F'$
is called a (holomorphic)
\emph{primitive} (or an \emph{antiderivative})
of $f$.
\end{definition}
\pause

Primitives do not always exist, but if they do, then they are unique up
to a constant.

\begin{proposition}
Suppose $U \subset \C$ is a domain, and
$F \colon U \to \C$ and
$G \colon U \to \C$ are holomorphic such that $F' = G'$.  Then
$F(z) = G(z) + C$ for some $C \in \C$.
\end{proposition}

\pause
Proof is an exercise.
\end{frame}

\begin{frame}
\begin{theorem}[Fundamental theorem of calculus for line integrals]
Suppose $U \subset \C$ is open and $f \colon U \to \C$
is continuous with a primitive $F \colon U \to \C$ (so $F' = f$).

\pause
Let $\gamma \colon [a,b] \to U$ be a piecewise-$C^1$ path.
Then
\begin{equation*}
\int_\gamma f(z) \, dz =
F\bigl( \gamma(b) \bigr) - F\bigl( \gamma(a)
\bigr) .
\end{equation*}
\end{theorem}

\medskip
\pause

\textbf{Proof:}
\[
\int_\gamma F'(z) \, dz
\pause
=
\int_a^b F'\bigl(\gamma(t)\bigr) \gamma'(t) \, dt 
\pause
=
\int_a^b \frac{d}{dt} \Bigl( F\bigl(\gamma(t)\bigr) \Bigr) \, dt 
\pause
=
F\bigl( \gamma(b) \bigr) - F\bigl( \gamma(a) \bigr) .
\]
\pause
We used the chain rule for composing real differentiable
and a holomorphic function,

\medskip
\pause
and the standard (real)
fundamental theorem of calculus (applied to the real and imaginary parts).
\qed

\medskip
\pause
\textbf{Remark:}
The hypothesis that $f=F'$ is continuous is extraneous (we will
prove later that $f$ is better than continuous, it is holomorphic).
\end{frame}

\begin{frame}

\begin{definition}
A chain $\Gamma$ that is equivalent to
$a_1 \gamma_1 + \cdots + a_n \gamma_n$, where $\gamma_1, \ldots, \gamma_n$
are closed piecewise-$C^1$ paths
and $a_1,\ldots,a_n \in \Z$,
is called a \emph{cycle}.
\end{definition}

\pause

A cycle need not be a sum of closed paths, just equivalent to it.
\pause

E.g., the unit square path we saw before is a cycle and could be written
as the chain
$[0,1] + [1,1+i]+[1+i,i]+[i,0]$.

\pause
\medskip

The fundamental theorem has an immediate corollary, noting that
for a closed path $\gamma \colon[a,b] \to \C$, $\gamma(a)=\gamma(b)$.

\pause

\begin{corollary}[Cauchy's theorem for derivatives]
Suppose $U \subset \C$ is open and $f \colon U \to \C$
is continuous with a primitive
$F \colon U \to \C$.
Let $\Gamma$ be
a cycle
in $U$.
Then
\begin{equation*}
\int_\Gamma f(z) \, dz = 0 .
\end{equation*}
\end{corollary}

\end{frame}

\begin{frame}
We will prove several versions of Cauchy, though usually we will put
restrictions on $U$ or the $\Gamma$ rather than the function.

\medskip
\pause

``Cauchy's theorem'' (in general) is a sort of an independence of path result
used to define a primitive,
so either the function has a primitive in $U$
or we have to restrict which $\Gamma$s are allowed.

\pause
\medskip

We can now prove a very neat version of the theorem (exercise):

\begin{corollary}[Cauchy's theorem for polynomials]
Suppose $P(z)$ is a polynomial and $\Gamma$ is
a cycle (in $\C$).
Then
\begin{equation*}
\int_\Gamma P(z) \, dz = 0 .
\end{equation*}
\end{corollary}
\end{frame}

\end{document}
