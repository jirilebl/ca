\documentclass[10pt,aspectratio=169]{beamer}

% All the boilerplate is in ccaslides.sty
% Note that this also pulls in a custom vogtwidebar.sty
\usepackage{ccaslides}

\author{Ji\v{r}\'i Lebl}

\institute[OSU]{%
Departemento pri Matematiko de Oklahoma {\^S}tata Universitato}

\title{Cultivating Complex Analysis:\\%
Power series (2.3 part 2)}

\date{}

\begin{document}

\begin{frame}
\titlepage
\end{frame}

\begin{frame}
Consider a power series
\quad $\displaystyle
\sum_{n=0}^\infty c_n {(z-p)}^n .
$

\medskip
\pause

Define
\quad
$\displaystyle
R = \frac{1}{\limsup\limits_{n \to \infty} \sqrt[n]{\sabs{c_n}}} .
$
\quad
We interpret $\nicefrac{1}{\infty} = 0$ and $\nicefrac{1}{0} = \infty$,
so $R=\infty$ is allowed.

\medskip
\pause

Let $r=\abs{z-p}$.
By the root test, the series $\sum \sabs{c_n} r^n$ converges if
\[
\limsup_{n \to \infty} \sqrt[n]{\sabs{c_n} r^n} = 
r \limsup_{n \to \infty} \sqrt[n]{\sabs{c_n}} = r \frac{1}{R} < 1 .
\]
\pause
So the power series converges absolutely when $r < R$.

\medskip
\pause

If $r \frac{1}{R} > 1$, then for infinitely many $n$,
$\sabs{c_n {(z-p)}^n} > 1$ and the power series diverges
if $r > R$.

\pause


\begin{proposition}[Cauchy--Hadamard theorem]
$\sum c_n {(z-p)}^n$ converges absolutely if
$\sabs{z-p} < R$ and diverges if
$\sabs{z-p} > R$.
\end{proposition}

\end{frame}

\begin{frame}
$\displaystyle
R = \frac{1}{\limsup\limits_{n \to \infty} \sqrt[n]{\sabs{c_n}}}
$

\vspace*{-0.7in}
\hspace*{2in}%
\scalebox{0.9}{
\subimport*{../figures/}{radiusconvcomplex.pdf_t}
}

\vspace*{-0.1in}

If $0 < R < \infty$,

the power series converges absolutely
in the disc $\Delta_R(p)$.

\medskip
\pause

It diverges in the complement of the closure $\overline{\Delta_R(p)}$.

\medskip
\pause

Convergence (or divergence) on the boundary circle $\partial \Delta_R(p)$
is tricky.

\medskip
\pause

$R=0$ means the power series diverges, $R=\infty$ means it converges in
$\C$.

\medskip
\pause

$R$ is called the \emph{radius of convergence}.
\end{frame}

\begin{frame}

\begin{proposition}
The series $\sum c_n {(z-p)}^n$ converges in $\Delta_{R}(p)$ for some
$R > 0$ if and only if
for every $r$ with
$0 < r < R$, there exists an $M > 0$ such that
\begin{equation*}
\sabs{c_n} \leq \frac{M}{r^n} \qquad \text{for all } n .
\end{equation*}
\end{proposition}

\pause
So the sequence
$\bigl\{ \sabs{c_n} r^n \bigr\}$ is bounded whenever $0 < r < R$.

\medskip
\pause

But $\bigl\{ \sabs{c_n} R^n \bigr\}$ not necessarily bounded:

\medskip
\pause

$\sum z^n$ and $\sum n z^n$ have
radius of convergence $R=1$.
\pause The sequence of coefficients
is bounded in the first case and
not in the second.

\medskip
\pause

However, $\{ n r^n \}$ is bounded for every $r < 1$.
\end{frame}

\begin{frame}

\textbf{Proof:}
Suppose the series converges in $\Delta_{R}(p)$ and
$0 < r < R$. 

\medskip
\pause
Then $\sum \sabs{c_n}r^n$ converges, and the terms are thus bounded.

\medskip
\pause

Conversely, fix $r$, suppose 
$\sabs{c_n} r^n \leq M$ for all $n$.

\medskip
\pause

Suppose $0 < s < r$.

\vspace*{-0.6in}
\hspace*{3.5in}%
\subimport*{../figures/}{threediscs.pdf_t}

\pause

\vspace*{-0.7in}

\qquad
$\displaystyle
\sqrt[n]{\sabs{c_n} s^n} \pause =
\frac{s}{r}\sqrt[n]{\sabs{c_n} r^n} \pause
\leq \frac{s}{r} \sqrt[n]{M} .
$

\medskip
\pause

The limsup of the RHS is strictly less than 1 as $\nicefrac{s}{r} < 1$.

\medskip
\pause

The series converges absolutely in
$\overline{\Delta_s(p)}$ by the root test.

\medskip
\pause

As $s$ and $r$ with $0 < s < r < R$ were arbitrary,
the series converges (absolutely) in $\Delta_R(p)$.
\qed
\end{frame}

\end{document}

