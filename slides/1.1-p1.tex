\documentclass[10pt,aspectratio=169]{beamer}

% All the boilerplate is in ccaslides.sty
% Note that this also pulls in a custom vogtwidebar.sty
\usepackage{ccaslides}

\author{Ji\v{r}\'i Lebl}

\institute[OSU]{%
Departemento pri Matematiko de Oklahoma {\^S}tata Universitato}

\title{Cultivating Complex Analysis:\\%
The complex numbers as the plane (1.1.1)}

\date{}

\begin{document}

\begin{frame}
\titlepage
\end{frame}

\begin{frame}
\emph{Modern mathematics} takes a false statement such as

\medskip

\emph{"All polynomials have a root."}

\medskip
\pause

and redefines what it means to have a \emph{root}
to make the statement true.

\medskip
\pause

Examples:

Start with $\N = \{ 1,2,3,\ldots \}$,
and arrive at $\Z$ to solve something like $n+2=1$.

\medskip
\pause

Start with $\Z$ and arrive at $\Q$ to solve
$2x=1$.

\medskip
\pause

Start with $\Q$ and arrive at $\R$ to be able to take limits.

\medskip
\pause

Start with $\R$ and arrive at $\C$ to be able to solve $z^2+1=0$.
\end{frame}

\begin{frame}
Interestingly, analysis with $\C$ is very different to analysis with $\R$.

\medskip
\pause

Mainly once we ask for the complex derivative we get lots of odd statements:

\medskip

\emph{If you can differentiate once, you can differentiate twice.}

\pause
\emph{Every function acts sort of like a linear function.}

\pause
\emph{If all derivatives are zero at a point, the function is constant.}

\pause
etc.

\medskip
\pause

A graduate real analysis course crushes the student's dreams.

\medskip
\pause

A graduate complex analysis course fills the student with unrealistic
optimism.

\end{frame}

\begin{frame}
Definition of the complex numbers (complex field, complex plane):

\[
\C \overset{\text{def}}{=} \R^2
\]

with multiplication:
\begin{align*}
(a,b) + (c,d) & \overset{\text{def}}{=} (a+b,c+d) , \\
(a,b) (c,d) & \overset{\text{def}}{=} (ac-bd,bc+da) .
\end{align*}

\pause

This makes $\C$ into a field (exercise).

\pause
\medskip

$\R \subset \C$ by identifying $x \in \R$ with $(x,0)$.

\pause
\medskip

\emph{Imaginary unit:}
\[
i \overset{\text{def}}{=} (0,1)
\]

\pause
\medskip

$
i^2 = -1
$
\qquad so
$z^2+1=0$ has the two solutions $i$ and $-i$.
\end{frame}

\begin{frame}

$(x,y) = x+iy$, so use $x+iy$ from now on (\emph{cartesian form}).

\medskip
\pause

The evil twin of $z=x+iy$ is the \emph{complex conjugate}
\[
\bar{z} \overset{\text{def}}{=} x-iy.
\]

\medskip
\pause

Let's give names to the $x$ and $y$:

\[
\Re z = 
\Re (x+iy)
\overset{\text{def}}{=}
\frac{z+\bar{z}}{2}
= x
\qquad \text{\emph{real part} of } z .
\]

\pause

\[
\Im z = 
\Im (x+iy)
\overset{\text{def}}{=}
\frac{z-\bar{z}}{2i}
= y
\qquad \text{\emph{imaginary part} of } z .
\]

\pause

An expression in $x,y$ can be written in terms of $z,\bar{z}$
and vice versa:

\[
x^3 + y^3 + 3ixy
=
{\left( \frac{z+ \bar{z}}{2} \right)}^3 + 
{\left( \frac{z- \bar{z}}{2i} \right)}^3 + 
3i {\left( \frac{z+ \bar{z}}{2} \right)} 
{\left( \frac{z- \bar{z}}{2i} \right)} ,
\]
\pause
or
\[
z^2 - i \bar{z}^2 + z \bar{z}
=
{(x+iy)}^2 - i {(x-iy)}^2 + 
(x+iy)(x-iy) .
\]
\pause

Almost looks as if $z$ and $\bar{z}$ were independent variables.
\end{frame}

\end{document}
