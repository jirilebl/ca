\documentclass[10pt,aspectratio=169]{beamer}

% All the boilerplate is in ccaslides.sty
% Note that this also pulls in a custom vogtwidebar.sty
\usepackage{ccaslides}

\author{Ji\v{r}\'i Lebl}

\institute[OSU]{%
Departemento pri Matematiko de Oklahoma {\^S}tata Universitato}

\title{Cultivating Complex Analysis:\\%
Rouch\'e's theorem (5.4.2)}

\date{}

\begin{document}

\begin{frame}
\titlepage
\end{frame}

\begin{frame}
Two holomorphic functions that are close on a cycle
have the same number of zeros inside.

\pause
\bigskip

For meromorphic functions the difference of zeros and poles
is the same.

\pause
\bigskip

\textbf{Example:} $z^2$ and $(z-\epsilon)(z+\epsilon)$ are close on $\partial \D$.

\pause
\bigskip

\textbf{Example:} $1$ and $\frac{z-\epsilon}{z+\epsilon}$ are close on $\partial \D$.

\end{frame}

\begin{frame}
\begin{theorem}[Rouch\'e]
Suppose $U \subset \C$ is open, $\Gamma$ is
a cycle in $U$ homologous to zero in $U$,
and $n(\Gamma;z)$ is either $0$ or $1$ for all $z \notin \Gamma$.
\pause
Suppose that $f \colon U \to \C_\infty$ and $g \colon U \to \C_\infty$
are meromorphic functions with no zeros or poles on
$\Gamma$ such that
\[
\sabs{f(z)-g(z)} < \sabs{f(z)}+\sabs{g(z)}
\qquad
\text{for all } z \in \Gamma .
\]
\pause
Let $V = \{ z \in U \setminus \Gamma : n(\Gamma;z) = 1 \}$.
Let $N_f$, $N_g$ be the number of zeros in $V$
and $P_f$, $P_g$ the number of poles in $V$ (both counting multiplicity)
of $f$ and $g$ respectively.
\pause
Then
\begin{equation*}
N_f - P_f = 
N_g - P_g.
\end{equation*}
\end{theorem}

\pause

\begin{corollary}[Rouch\'e]
Let $U$, $\Gamma$ and $V$ be as in the theorem.
Suppose $f \colon U \to \C$ and $g \colon U \to \C$
are holomorphic such that
$\sabs{f(z)-g(z)} < \sabs{f(z)}+\sabs{g(z)}$
for all $z \in \Gamma$.  Then $f$ and $g$ have the same number of zeros
(counting multiplicity) in $V$.
\end{corollary}
\end{frame}

\begin{frame}
Classical statement of the theorem uses the weaker inequality
\[
\sabs{f(z)-g(z)} < \sabs{f(z)}
\]

\pause

It has a nice geometric interpretation:

\begin{center}
\scalebox{0.9}{
\subimport*{../figures/}{dogandtree.pdf_t}
}
\end{center}

\end{frame}

\begin{frame}
\textbf{Proof:}
\begin{equation*}
\abs{\frac{f(z)}{g(z)} - 1} <
\abs{\frac{f(z)}{g(z)}} + 1
\qquad \text{on } \Gamma .
\end{equation*}
\pause
So
$\varphi(z) = \frac{f(z)}{g(z)}$ is never negative (and it is never zero) on
$\Gamma$ (hence on a neighborhood).

\pause
\medskip

Let $\Log$ be the principal branch of log on $\C \setminus (-\infty,0]$.

\pause
\medskip

The function
$\frac{\varphi'}{\varphi}$ has an antiderivative $\Log \circ \varphi$ on a neighborhood of $\Gamma$.

\pause
\medskip

By Cauchy's theorem for derivatives,
together with the argument principle:

\medskip

$
\displaystyle
\qquad
0
= \frac{1}{2\pi i} \int_{\Gamma} \frac{\varphi'(z)}{\varphi(z)} \, dz
\pause
=
\frac{1}{2\pi i} \int_{\Gamma}
\left( \frac{f'(z)}{f(z)} - \frac{g'(z)}{g(z)} \right) \, dz 
$

\medskip
\pause

$\displaystyle
\qquad\qquad
=
\frac{1}{2\pi i} \int_{\Gamma}
\frac{f'(z)}{f(z)} \, dz 
-
\frac{1}{2\pi i} \int_{\Gamma}
\frac{g'(z)}{g(z)} \, dz 
\pause
=
(N_f - P_f) - (N_g - P_g) .
\quad
$
\qed
\end{frame}

\begin{frame}
The classical hypothesis $\sabs{f(z)-g(z)} < \sabs{f(z)}$ is often
sufficient.

\medskip
\pause

\textbf{Example:}
Consider $P(z) = z^n + 1$.
Let us use Rouch\'e to show that all the zeros are on $\partial \D$.

\medskip
\pause

On $\partial \Delta_{1-\epsilon}(0)$,
\[
\abs{P(z) - 1} = \abs{z}^n < 1 = \sabs{1}.
\]
\pause
By Rouch\'e %($P(z)$ is the dog and $1$ is the master)
$P(z)$ and $1$ have the same number of zeros in 
$\Delta_{1-\epsilon}(0)$.

\medskip
\pause

On $\partial \Delta_{1+\epsilon}(0)$,
\begin{equation*}
\abs{P(z) - z^n} = 1 < \abs{z^n} .
\end{equation*}
By Rouch\'e,
$P(z)$ and $z^n$ have the same number of zeros in 
$\Delta_{1+\epsilon}(0)$.
\end{frame}

\begin{frame}
\textbf{Example:}
Consider $P(z) = z^4+12z^3+24z^2+4z+6$.
\pause
On $\partial \D$,
\begin{multline*}
\abs{P(z)-(z^4+24z^2)} =
\abs{12z^3+4z+6} \leq
\abs{12z^3}+\abs{4z}+\abs{6} \\
= 22 < 23 = \abs{\sabs{24 z^2} - \sabs{z^4}}
\leq \sabs{z^4+24z^2} .
\end{multline*}
\pause
$z^4+24z^2$ has zeros at $\pm \sqrt{24} i$ (outside $\D$)
and two zeros at the origin (inside $\D$).  So
$P(z)$ also has two zeros in $\D$.

\pause
\medskip

If $\sabs{z} = 46+\epsilon$, then
\[
\abs{P(z)-z^4} = \abs{12z^3+24z^2+4z+6}
\leq 46 \sabs{z}^3 < {\sabs{z}}^4 = \abs{z^4}.
\]
\pause
So all four zeros satisfy $\sabs{z} < 46+\epsilon$, that is,
$\sabs{z} \leq 46$.

\pause
\medskip

(Actually the largest zero of $P$ has modulus less than 10).
\end{frame}

\end{document}
