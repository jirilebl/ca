\documentclass[10pt,aspectratio=169]{beamer}

% All the boilerplate is in ccaslides.sty
% Note that this also pulls in a custom vogtwidebar.sty
\usepackage{ccaslides}

\author{Ji\v{r}\'i Lebl}

\institute[OSU]{%
Departemento pri Matematiko de Oklahoma {\^S}tata Universitato}

\title{Cultivating Complex Analysis:\\%
Harmonic functions\\%
Identity and the maximum principle (7.1.2)}

\date{}

\begin{document}

\begin{frame}
\titlepage
\end{frame}

\begin{frame}
A zero set of a harmonic $f$ can have limit points: E.g., $\Re z$.
But no open sets.

\pause

\begin{theorem}[Identity]
Let $U \subset \C$ be a domain and $f \colon U \to \R$ a harmonic function.
Suppose $V \subset U$ is a nonempty open subset and $f = 0$ on $V$.  Then $f
\equiv 0$.
\end{theorem}

\pause
\textbf{Proof:}
Let $Z_f = f^{-1}(0)$.
\pause
\quad
Let $Z$ be the closure of the interior of $Z_f$ (subspace topology).

\medskip
\pause

Suppose $Z$ is nonempty and $p \in Z$.
\pause

If $\Delta_r(p) \subset U$, $f$ is
zero on some open subset of $\Delta_r(p)$.

\pause
\medskip

$\exists$ a holomorphic $h \colon \Delta_r(p) \to \C$ such that $f = \Re h$
on $\Delta_r(p)$.

\pause
\thus\quad
$h$ is holomorphic and purely imaginary on an open subset of
$\Delta_r(p)$.

\pause
\thus\quad
$h$ is constant on an open subset of $\Delta_r(p)$.

\pause
\thus\quad
$h$ is constant on $\Delta_r(p)$.

\pause
\thus\quad
$f$ is constant on $\Delta_r(p)$.

\pause
\thus\quad
$f$ is zero on $\Delta_r(p)$.

\pause
\thus\quad
$Z$ is open.

\pause
\medskip

$Z$ is also closed and $U$ is connected \wthus $Z=U$.
\qed
\end{frame}

\begin{frame}
\begin{theorem}[Maximum principle]
Suppose $U \subset \C$ is a domain and $f \colon U \to \R$
is harmonic.
If $f$ attains a local maximum (or a local minimum) in $U$, then $f$ is constant.
\end{theorem}

\pause
\textbf{Proof:}
WLOG assume $f$ has a local maximum at $p \in U$ (otherwise consider $-f$).

\medskip
\pause

It is the global maximum on some $\Delta_r(p) \subset U$.

\medskip
\pause

There exists a holomorphic $h \colon \Delta_r(p) \to \C$ such that $f = \Re h$.

\medskip
\pause

$h$ takes
$\Delta_r(p)$ to $X = \bigl\{ w \in \C : \Re w \leq f(p) \bigr\}$.

\medskip
\pause

$h(p)$ is on the boundary of $X$ (as $\Re h(p)= f(p)$).

\medskip
\pause

$h\bigl(\Delta_r(p)\bigr)$ is not open
\pause
\wthus $h$ is constant (open mapping theorem).

\medskip
\pause

So $f$ is constant on $\Delta_r(p)$
\pause
\wthus $f$ is constant on $U$ by identity.
\qed
\end{frame}

\begin{frame}
\textbf{Exercise:}
Prove that
the maximum principle for harmonic functions implies the maximum
modulus principle for holomorphic functions.
Hint: Consider $\log \sabs{f(z)}$.

\pause
\medskip

\textbf{Exercise:}
Prove the second version of the maximum principle: If $U \subset \C$
is a bounded domain and
$f \colon \widebar{U} \to \R$ is continuous and harmonic on $U$,
then $f$ achieves both its maximum and its minimum
on the boundary $\partial U$.

\pause
\medskip

\textbf{Exercise:}
Suppose $U \subset \C$ is a domain and
$f \colon U \to \R$ is harmonic.
Prove that $f(U)$ is an open interval or a single point.
\end{frame}

\end{document}
