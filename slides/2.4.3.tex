\documentclass[10pt,aspectratio=169]{beamer}

% All the boilerplate is in ccaslides.sty
% Note that this also pulls in a custom vogtwidebar.sty
\usepackage{ccaslides}

\author{Ji\v{r}\'i Lebl}

\institute[OSU]{%
Departemento pri Matematiko de Oklahoma {\^S}tata Universitato}

\title{Cultivating Complex Analysis:\\%
The exponential (as power series) (2.4.3)}

\date{}

\begin{document}

\begin{frame}
\titlepage
\end{frame}

\begin{frame}
Previously we defined
\[
e^{x+iy}
= e^x (\cos y + i \sin y)
\]
and we showed that $e^z$ was holomorphic. \pause Let's define it a different
way.

\medskip
\pause

\begin{proposition}
The power series
\begin{equation*}
f(z) = \sum_{n=0}^\infty \frac{1}{n!} z^n ,
\end{equation*}
is the unique convergent power series at the origin
such that $f(0)=1$ and $f'=f$.
\pause

Moreover, the series converges on $\C$ and
$f(z) = e^z$.
\end{proposition}

\pause

\textbf{Proof:}
Consider a convergent series $f$ satisfying $f(0)=1$ and $f'=f$:
\[
f(z) = \sum_{n=0}^\infty c_n z^n .
\]
\pause
$f(0)=1$ implies that $c_0 = 1$.

\end{frame}

\begin{frame}
$f' = f$, so  \pause
\[
f'(z) =
\sum_{n=1}^\infty n c_n z^{n-1} =
\sum_{n=0}^\infty (n+1) c_{n+1} z^{n} 
\pause
=
f(z) = \sum_{n=0}^\infty c_n z^n .
\]
\pause
Coefficients of power series are unique  so $c_n = (n+1) c_{n+1}$.

\pause
By induction, $c_n = \frac{1}{n!}$ so
\(\displaystyle f(z) = \sum_{n=0}^\infty \frac{1}{n!} z^n \)

\pause
And this series converges in $\C$ (exercise).

\pause
\medskip

Both $f$ and the exponential are holomorphic and equal to their derivatives:
\pause
\[
\frac{d}{dz} \left[ \frac{f(z)}{\exp(z)} \right]
=
\frac{f'(z)\exp(z) - f(z) \exp'(z)}{{\bigl(\exp(z)\bigr)}^2}
\pause
=
\frac{f(z)\exp(z) - f(z) \exp(z)}{{\bigl(\exp(z)\bigr)}^2}
= 0.
\]
\pause
Hence, $f(z) = C \exp(z)$ for some constant $C$.
\pause
As $f(0) = \exp(0) = 1$, conclude $C=1$.
\qed
\end{frame}


\end{document}
