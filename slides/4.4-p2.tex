\documentclass[10pt,aspectratio=169]{beamer}

% All the boilerplate is in ccaslides.sty
% Note that this also pulls in a custom vogtwidebar.sty
\usepackage{ccaslides}

\author{Ji\v{r}\'i Lebl}

\institute[OSU]{%
Departemento pri Matematiko de Oklahoma {\^S}tata Universitato}

\title{Cultivating Complex Analysis:\\%
Laurent series (4.4 part 2)}

\date{}

\begin{document}

\begin{frame}
\titlepage
\end{frame}

\begin{frame}
\begin{theorem}[Existence of Laurent series]
Suppose that $0 \leq r_1 < r_2 \leq \infty$ and
$f \colon \ann(p;r_1,r_2) \to \C$ is holomorphic.
\pause
Then there exist unique numbers $c_n \in \C$ for $n \in \Z$ such that
\[
f(z) = \sum_{n=-\infty}^{\infty} c_n {(z-p)}^n ,
\]
converging uniformly absolutely on compact subsets of
$\ann(p;r_1,r_2)$.
\pause
The numbers $c_n$ are given by
\[
c_n = 
\frac{1}{2\pi i}
\int_{\gamma}
\frac{f(z)}{{(z-p)}^{n+1}}
\,
dz  ,
\]
where $\gamma$ is any circle of radius $s$, $r_1 < s < r_2$, centered at
$p$ oriented counterclockwise.
\end{theorem}
\end{frame}

\begin{frame}

\textbf{Proof of the theorem:}
Choose $s_1$ and $s_2$ such that $r_1 < s_1 < s_2 < r_2$.

\medskip
\pause

Define the cycle
\qquad $\Gamma = \partial \Delta_{s_2}(p) - \partial \Delta_{s_1}(p)$.

\vspace*{-0.2in}
\hspace*{3.7in}%
\subimport*{../../ca/figures/}{twoannuli.pdf_t}

\vspace*{-1.45in}

\pause
If $q \in \C \setminus \ann(p;r_1,r_2)$, then $n(\Gamma;q) = 0$:

\medskip
\pause

If $q$ is in the hole,
$n\bigl(\partial \Delta_{s_j}(p);q\bigr) = 1$ for $j=1,2$.


\medskip
\pause

If $q$ is on the outside,
$n\bigl(\partial \Delta_{s_j}(p);q\bigr) = 0$ for $j=1,2$.

\medskip
\pause

So $\Gamma$ is homologous to zero in $\ann(p;r_1,r_2)$.


\medskip
\pause

However, when $z \in \ann(p;s_1,s_2)$, then $n(\Gamma;q) = 1$.

\medskip
\pause

By Cauchy's formula for $z \in \ann(p;s_1,s_2)$,

\medskip
$\displaystyle
\quad
f(z) = 
\frac{1}{2\pi i}
\int_{\Gamma} \frac{f(\zeta)}{\zeta-z} \, d\zeta 
\pause
=
\frac{1}{2\pi i}
\int_{\partial \Delta_{s_2}(p)} \frac{f(\zeta)}{\zeta-z} \, d\zeta 
-
\frac{1}{2\pi i}
\int_{\partial \Delta_{s_1}(p)} \frac{f(\zeta)}{\zeta-z} \, d\zeta  .
$
\medskip
\pause

We will expand the two integrals separately.
\end{frame}

\begin{frame}
First, if $\zeta \in \partial \Delta_{s_2}$, then
$\babs{\frac{z-p}{\zeta-p}} = \frac{\sabs{z-p}}{s_2} < 1$.

\medskip
\pause

We follow the same reasoning as in the proof of the existence
of power series.

\medskip
\pause

$\displaystyle
\quad
\frac{1}{2\pi i}
\int_{\partial \Delta_{s_2}(p)} \frac{f(\zeta)}{\zeta-z} \, d\zeta 
=
\frac{1}{2\pi i}
\int_{\partial \Delta_{s_2}(p)} \frac{f(\zeta)}{\zeta-p}
\frac{1}{1-\frac{z-p}{\zeta-p}} \, d\zeta
$

\medskip
\pause

$\displaystyle
\qquad \qquad \qquad
\qquad \qquad
=
\frac{1}{2\pi i}
\int_{\partial \Delta_{s_2}(p)} \frac{f(\zeta)}{\zeta-p}
\sum_{n=0}^\infty
{\left(\frac{z-p}{\zeta-p}\right)}^n \, d\zeta
$

\medskip
\pause

$\displaystyle
\qquad \qquad \qquad
\qquad \qquad
=
\sum_{n=0}^\infty
\underbrace{
\left(
\frac{1}{2\pi i}
\int_{\partial \Delta_{s_2}(p)} \frac{f(\zeta)}{{(\zeta-p)}^{n+1}}
 \, d\zeta
\right)
}_{c_n}
{(z-p)}^n .
$

\medskip
\pause

We can swap the series
limit with the integral as the convergence is uniform on the circle.
\end{frame}

\begin{frame}
Similarly, 
if $\zeta \in \partial \Delta_{s_1}$, then
$\babs{\frac{\zeta-p}{z-p}} = \frac{s_1}{\sabs{z-p}} < 1$, so

\medskip
\medskip
\pause

$\displaystyle
\quad
-\frac{1}{2\pi i}
\int_{\partial \Delta_{s_1}(p)} \frac{f(\zeta)}{\zeta-z} \, d\zeta 
= 
\frac{1}{2\pi i}
\int_{\partial \Delta_{s_1}(p)} \frac{f(\zeta)}{z-p}
\frac{1}{1-\frac{\zeta-p}{z-p}} \, d\zeta
$

\medskip
\pause

$\displaystyle
\qquad \qquad \qquad
\qquad \qquad ~
=
\frac{1}{2\pi i}
\int_{\partial \Delta_{s_1}(p)} \frac{f(\zeta)}{z-p}
\sum_{m=0}^\infty
{\left(\frac{\zeta-p}{z-p}\right)}^m \, d\zeta
$

\medskip
\pause

$\displaystyle
\qquad \qquad \qquad
\qquad \qquad ~
=
\sum_{m=0}^\infty
\left(
\frac{1}{2\pi i}
\int_{\partial \Delta_{s_1}(p)} f(\zeta){(\zeta-p)}^{m}
 \, d\zeta
\right)
{(z-p)}^{-m-1}
$

\medskip
\pause

$\displaystyle
\qquad \qquad \qquad
\qquad \qquad ~
=
\sum_{n=-\infty}^{-1}
\underbrace{
\left(
\frac{1}{2\pi i}
\int_{\partial \Delta_{s_1}(p)} \frac{f(\zeta)}{{(\zeta-p)}^{n+1}}
 \, d\zeta
\right)
}_{c_n}
{(z-p)}^{n} .
$
\end{frame}

\begin{frame}
We get the series although perhaps not the formula for $c_n$ for
circle of any radius $s$.

\medskip
\pause

Let $s$ be such that
$r_1 < s < r_2$.

\medskip
\pause

The cycle
$\partial \Delta_{s}(p) - \partial \Delta_{s_1}(p)$ is homologous to zero in
$\ann(p;r_1,r_2)$, and 
$z \mapsto \frac{f(z)}{{(z-p)}^{n+1}}$ is holomorphic in 
$\ann(p;r_1,r_2)$.

\medskip
\pause

Cauchy's theorem says
\[
0 = \int_{\partial \Delta_{s}(p) - \partial \Delta_{s_1}(p)}
\frac{f(\zeta)}{{(\zeta-p)}^{n+1}} \, d\zeta
\pause
=
\int_{\partial \Delta_{s}(p)}
\frac{f(\zeta)}{{(\zeta-p)}^{n+1}} \, d\zeta
-
\int_{\partial \Delta_{s_1}(p)}
\frac{f(\zeta)}{{(\zeta-p)}^{n+1}} \, d\zeta .
\]
\pause
Similarly for $s_2$ and so
\[
c_n = \frac{1}{2\pi i}
\int_{\partial \Delta_{s}(p)} \frac{f(\zeta)}{{(\zeta-p)}^{n+1}}
 \, d\zeta .
\]
\end{frame}

\begin{frame}
Next, convergence.

\medskip
\pause
For any $\epsilon > 0$,
the geometric series used for the first part converges uniformly
absolutely when $\babs{\frac{z-p}{\zeta-p}} = \frac{\sabs{z-p}}{s_2} \leq
1-\epsilon$.

\medskip

So the resulting series converges uniformly absolutely on compact
subsets of $\Delta_{s_2}(p)$.


\medskip
\pause
The geometric series used for the second part converges uniformly
absolutely when $\babs{\frac{\zeta-p}{z-p}} = \frac{s_1}{\sabs{z-p}} \leq
1-\epsilon$.

\medskip

So the resulting series converges uniformly absolutely on compact
subsets of $\C \setminus \overline{\Delta_{s_1}(p)}$.


\medskip
\pause

Any compact subset of $\ann(p;r_1,r_2)$ is a compact
inside both of these for some $s_1$, $s_2$.

\medskip
\pause

So the full series converges uniformly absolutely on compact subsets of
$\ann(p;r_1,r_2)$.
\end{frame}

\begin{frame}
Finally, uniqueness.

\medskip
\pause
Suppose
\[
f(z)
=
\sum_{n=-\infty}^{\infty} d_n {(z-p)}^{n} ,
\]
converging uniformly absolutely on compact subsets of $\ann(p;r_1,r_2)$.

\medskip
\pause

Then

\medskip
\pause

$\displaystyle
\quad
c_m
= \frac{1}{2\pi i}
\int_{\partial \Delta_{s}(p)} \frac{f(\zeta)}{{(\zeta-p)}^{m+1}}
 \, d\zeta 
\pause
=
\frac{1}{2\pi i}
\int_{\partial \Delta_{s}(p)}
\left(\sum_{n=-\infty}^{\infty} d_n {(\zeta-p)}^{n} \right)
\frac{1}{{(\zeta-p)}^{m+1}}
 \, d\zeta 
$
\medskip
\pause

$\displaystyle
\quad \quad \,\,
=
\frac{1}{2\pi i}
\sum_{n=-\infty}^{\infty}
d_n
\int_{\partial \Delta_{s}(p)}
{(\zeta-p)}^{n-m-1}
 \, d\zeta 
\pause
=
d_m .
$

\medskip
\pause

Note that
$\int_{\partial \Delta_{s}(p)}
{(\zeta-p)}^{n-m-1}
 \, d\zeta \not= 0$ only when $n=m$.
\qed
\end{frame}

\begin{frame}
We differentiate/antidifferentiate formally
(except for antidifferentiating the $c_{-1}{(z-p)}^{-1}$).

\pause

\begin{proposition}
Suppose $p \in \C$, $0 \leq r_1 < r_2 \leq \infty$, and
$f \colon \ann(p;r_1,r_2) \to \C$ is defined by
\[
f(z) = \sum_{n=-\infty}^\infty c_n {(z-p)}^n ,
\quad
\text{converging uniformly on compact subsets of } \ann(p;r_1,r_2).
\]
\pause
Then $f$ is holomorphic and its 
derivative is defined by
\[
f'(z) = \sum_{n=-\infty}^\infty n c_n {(z-p)}^{n-1} ,
\quad
\text{converging uniformly on compact subsets of } \ann(p;r_1,r_2).
\]
\pause
Moreover, if $c_{-1} = 0$, then $f = F'$, where
\[
F(z) = \sum_{n=-\infty,n\not=-1}^\infty \frac{c_n}{n+1} {(z-p)}^{n+1},
\quad
\text{converging uniformly on compact subsets of } \ann(p;r_1,r_2).
\]
\end{proposition}

\pause

\textbf{Proof:} Exercise.
\end{frame}

\begin{frame}
Some more useful exercises:

\medskip
\pause

\textbf{Exercise:}
Suppose $f$ and $g$ are holomorphic functions defined on
$\ann(p;r_1,r_2)$.  Let $a_n$ be the coefficients in the Laurent series for
$f$ and $b_n$ be the coefficients in the Laurent series for $g$.  Suppose
that $\alpha,\beta \in \C$.  Show that the Laurent series for the function
$\alpha f + \beta g$ has coefficients $\alpha a_n + \beta b_n$.

\medskip
\pause

\textbf{Exercise:}
Expand the function $f(z) = \frac{1}{(z-1)(z-2)}$ in the
sets
$\ann(0;0,1)$, $\ann(0;1,2)$, and $\ann(0;2,\infty)$.


\end{frame}

\end{document}
