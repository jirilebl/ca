\documentclass[10pt,aspectratio=169]{beamer}

% All the boilerplate is in ccaslides.sty
% Note that this also pulls in a custom vogtwidebar.sty
\usepackage{ccaslides}

\author{Ji\v{r}\'i Lebl}

\institute[OSU]{%
Departemento pri Matematiko de Oklahoma {\^S}tata Universitato}

\title{Cultivating Complex Analysis:\\%
Power series (2.3 part 3)}

\date{}

\begin{document}

\begin{frame}
\titlepage
\end{frame}

\begin{frame}
Let $K \subset \C$ be a set.
A power series $\sum c_n {(z-p)}^n$
\emph{converges uniformly absolutely}
for $z \in K$ when $\sum \sabs{c_n} \sabs{z-p}^n$
converges uniformly for $z \in K$.

\medskip
\pause

Suppose a series converges uniformly absolutely.

\medskip
\pause

It converges absolutely, so it converges, \pause and 
\[
\abs{
\sum_{n=0}^\infty c_n {(z-p)}^n
-
\sum_{n=0}^{m} c_n {(z-p)}^n
}
=
\abs{\sum_{n=m+1}^\infty c_n {(z-p)}^n} \leq
\sum_{n=m+1}^\infty \sabs{c_n} \sabs{z-p}^n .
\]
\pause
The RHS $\to 0$ uniformly in $z \in K$ as $m \to \infty$.

\medskip
\pause

So a uniformly absolutely convergent series converges uniformly.
\end{frame}

\begin{frame}

\begin{proposition}
Let $\sum c_n {(z-p)}^n$ be a power series with radius of convergence $R
> 0$.  If $0 < r < R$, then the power series converges uniformly absolutely
in $\overline{\Delta_r(p)}$.
\pause
Furthermore, let $U = \Delta_R(p)$ if $R < \infty$ and
$U = \C$ if $R=\infty$, and let $K \subset U$ be compact.
Then the series converges uniformly absolutely on $K$.
\end{proposition}

\pause

\textbf{Proof:}
WLOG suppose $R < \infty$ and let $0 < r < R$.

\medskip
\pause

$\sum c_n {(z-p)}^n$ converges absolutely
on $\Delta_R(p)$ \quad $\Rightarrow$ \quad
$\sum \sabs{c_n} r^n$ converges (and so does any tail).

\medskip
\pause

So for $z \in \overline{\Delta_r(p)}$,
\[
\abs{
\sum_{n=0}^\infty \sabs{c_n} \sabs{z-p}^n
-
\sum_{n=0}^{m} \sabs{c_n} \sabs{z-p}^n
}
\leq
\sum_{n=m+1}^\infty \sabs{c_n} r^n .
\]
\pause
The RHS, which does not depend on $z$,
goes to zero as $m \to \infty$.

\pause

Hence $\sum \sabs{c_n} \sabs{z-p}^n$ converges uniformly in $\overline{\Delta_r(p)}$.

\medskip
\pause

If $K \subset \Delta_{R}(p)$ is compact, $\exists$ $r < R$ such
that $K \subset \Delta_r(p)$ \pause

(consider an open cover of $K$ by discs $\Delta_r(p)$ for all $r < R$).

\pause
\medskip

The result follows.
\qed
\end{frame}

\begin{frame}
Let us mention a couple of useful results as exercises.

\medskip
\pause

\textbf{Exercise:}
(Weierstrass $M$-test)
Let $X$ be a set and $f_n \colon X \to \C$ is
a sequence of functions such that
$\abs{f_n(x)} \leq M_n$ for all $x \in X$ and $n \in \N$.

\medskip
\pause

If $\sum M_n < \infty$, then $\sum f_n(x)$ converges uniformly absolutely on $X$.

\medskip
\pause

\textbf{Exercise:}
Suppose
$\sum_{n=0}^\infty a_n z^{n}$ and $\sum_{n=0}^\infty b_n z^{n}$
have a radius of convergence at least $r > 0$.  Show that
$\sum_{n=0}^\infty (a_n+b_n) z^{n}$ has a radius of convergence at least
$r$ and converges to the sum of the two series.
\end{frame}

\end{document}

