\documentclass[10pt,aspectratio=169]{beamer}

% All the boilerplate is in ccaslides.sty
% Note that this also pulls in a custom vogtwidebar.sty
\usepackage{ccaslides}

\author{Ji\v{r}\'i Lebl}

\institute[OSU]{%
Departemento pri Matematiko de Oklahoma {\^S}tata Universitato}

\title{Cultivating Complex Analysis:\\%
The argument principle (5.4.1)}

\date{}

\begin{document}

\begin{frame}
\titlepage
\end{frame}

\begin{frame}
\emph{zeros/poles counted with multiplicity}:
$f(z) = z^2{(z-1)}^3$ has the zeros $z_1,z_2,z_3,z_4,z_5 = 0,0,1,1,1$.

\pause

\begin{theorem}[Argument principle]\index{argument principle}\label{thm:argprinc}
Suppose $U \subset \C$ is open and $\Gamma$ is
a cycle in $U$
homologous to zero in $U$.
Suppose $f \colon U \to \C_\infty$ is a meromorphic function with no zeros
or poles on $\Gamma$.
Let $z_1,\ldots,z_n$ denote the 
zeros of $f$ counted with multiplicity,
and let $p_1,\ldots,p_\ell$ denote the poles of $f$ counted with multiplicity.
Then
\begin{equation*}
\frac{1}{2\pi i}
\int_\Gamma \frac{f'(z)}{f(z)} \, dz
=
\sum_{k=1}^n n(\Gamma;z_k)
\quad
-
\quad
\sum_{k=1}^\ell n(\Gamma;p_k) .
\end{equation*}
\pause
Furthermore, if $h \colon U \to \C$ is holomorphic, then
\begin{equation*}
\frac{1}{2\pi i}
\int_\Gamma h(z) \frac{f'(z)}{f(z)} \, dz
=
\sum_{k=1}^n n(\Gamma;z_k)h(z_k) 
\quad
-
\quad
\sum_{k=1}^\ell n(\Gamma;p_k)h(p_k) .
\end{equation*}
\end{theorem}

\pause

\medskip

Poles/zero normally countable, but can assume finite above.

\end{frame}

\begin{frame}

Suppose $n(\Gamma;z) = 1$ or $0$ for all $z \in U$.

\medskip

The ``inside of
$\Gamma$'' are the points where $n(\Gamma;z)=1$.

\medskip

If there are
$n$ zeros and $\ell$ poles (counting multiplicity) inside $\Gamma$, then
\begin{equation*}
\frac{1}{2\pi i}
\int_\Gamma \frac{f'(z)}{f(z)} \, dz
= n - \ell .
\end{equation*}

\bigskip
\pause 

The integral $\int_{\Gamma} \frac{f'(z)}{f(z)} \, dz$
gives $i$ times the change in argument of $f$ as we traverse $\Gamma$,
since the ``antiderivative'' of 
$\frac{f'(z)}{f(z)}$ is $\log f(z) = \log \sabs{f(z)} + i \arg f(z)$.

\bigskip
\pause

Another interpretation:
\begin{equation*}
\frac{1}{2\pi i}
\int_{\gamma} \frac{f'(z)}{f(z)} \, dz = 
\frac{1}{2\pi i}
\int_{f \circ \gamma} \frac{1}{\zeta} \, d\zeta 
=
n(f \circ \gamma ; 0 )
.
\end{equation*}

\end{frame}

\begin{frame}
\textbf{Proof:}
$h(z) \frac{f'(z)}{f(z)}$ has isolated singularities
at the zeros and poles of $f$.
\pause
Let $S$ be the set of zeros and poles of $f$.
\pause
By residue theorem
\begin{equation*}
\frac{1}{2\pi i}
\int_\Gamma h(z) \frac{f'(z)}{f(z)} \, dz
=
\sum_{p \in S} n(\Gamma;p)\operatorname{Res}\left(h \, \frac{f'}{f};p\right) .
\end{equation*}
\pause
Consider a zero of $f$ of multiplicity $m$ or pole of
order $-m$.

WLOG suppose it is the origin.

\medskip
\pause

Write $f(z)  = z^m F(z)$ where $F(0) \not=0$
and $h(z) = h(0) + z H(z)$.
\pause
\[
h(z) \frac{f'(z)}{f(z)}
=
\bigl( h(0) + z H(z) \bigr)
\frac{m z^{m-1} F(z) + z^m F'(z)}{z^m F(z)}
\pause
%=
%\bigl( h(0) + z H(z) \bigr)
%\frac{m F(z) + z F'(z)}{z F(z)}
%\\
=
m\, h(0) 
\frac{1}{z}
+
h(0) 
\frac{F'(z)}{F(z)}
+
H(z)
\frac{m F(z) + z F'(z)}{F(z)} .
\]
\pause
Everything except $m\, h(0) \frac{1}{z}$ is holomorphic.
So
\[
\operatorname{Res}\left(h \, \frac{f'}{f};0\right) = m\, h(0)
\qed
\]
\end{frame}

\begin{frame}
Application:
Locate zeros of holomorphic $f$ (e.g.\ polynomials) by computing

(even numerically)
\[
\frac{1}{2\pi i}
\int_\Gamma \frac{f'(z)}{f(z)} \, dz .
\]

\pause

Related application:

If $z_1,\ldots,z_n$ are zeros of $f$ inside $\Gamma$ (going around them once), then
\begin{equation*}
\frac{1}{2\pi i}
\int_\Gamma z^k \, \frac{f'(z)}{f(z)} \, dz
=
z_1^k + \cdots + z_n^k .
\end{equation*}

\pause

If there is one simple zero $z_0$ of
$f$ within $\Gamma$, then
\begin{equation*}
\frac{1}{2\pi i}
\int_\Gamma z \, \frac{f'(z)}{f(z)} \, dz
=
z_0 .
\end{equation*}

\end{frame}

\end{document}
