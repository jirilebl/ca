\documentclass[10pt,aspectratio=169]{beamer}

% All the boilerplate is in ccaslides.sty
% Note that this also pulls in a custom vogtwidebar.sty
\usepackage{ccaslides}

\author{Ji\v{r}\'i Lebl}

\institute[OSU]{%
Departemento pri Matematiko de Oklahoma {\^S}tata Universitato}

\title{Cultivating Complex Analysis:\\%
Linear fractional transformations (1.4 part 2)}

\date{}

\begin{document}

\begin{frame}
\titlepage
\end{frame}

\begin{frame}
An LFT, $\frac{a z + b}{c z + d}$, can be viewed as a $2 \times 2$ complex matrix.

\medskip
\pause

We need to view the Riemann sphere as the so-called (one-dimensional)
complex projective space.

\medskip
\pause

Define the equivalence relation $\sim$ on $\C^2 \setminus \{ 0 \}$ by
$u \sim v$ $\Leftrightarrow$ $u = \lambda v$ for some $\lambda \in \C$.
Define
\[
\C \bP^1
\overset{\text{def}}{=}
\faktor{\C^2 \setminus \{ 0\}}{\sim} \ .
\]

\pause

$\C \bP^1$ is the set of ``complex lines through the
origin'' in $\C^2$,

or the set of one-dimensional vector subspaces of $\C^2$.

\medskip
\pause

Denote by $[z:w] \in \C \bP^1$ the equivalence class containing
$(z,w) \in \C^2 \setminus \{0\}$.

\medskip
\pause

Define the bijection $\Psi \colon \C_\infty \to \C \bP^1$ as
\begin{equation*}
\Psi(z) =
\begin{cases}
[z:1] & \text{if } z \in \C , \\
[1:0] & \text{if } z=\infty .
\end{cases}
\end{equation*}

\end{frame}

\begin{frame}
We claim an LFT corrseponds to an invertible $2 \times 2$ matrix:

\[
f(z) = \frac{a z + b}{c z + d}
\qquad
\leftrightarrow
\qquad
M =
\begin{bmatrix}
a & b \\
c & d
\end{bmatrix} .
\]

\pause

An invertible $M$ takes 1-dim subspaces to 1-dim subspaces,
$ad\not=bc$ means $M$ is invertible.

\medskip
\pause

For $z \in \C \setminus \bigl\{ \nicefrac{-d}{c} \bigr\}$ (or $z \in \C$ if $c=0$)
\[
\Psi \circ f(z) =
\left[\frac{a z + b}{c z + d}: 1 \right]
\pause
=
\left[a z + b: c z + d \right] .
\]
\pause
When $z = \nicefrac{-d}{c}$, then $cz+d = 0$, and
$\Psi\circ f(z) = \Psi(\infty) = [1:0] = [az+b : cz+d]$ as well.

\medskip
\pause

%Let us consider $\Psi \circ f \circ \Psi^{-1}$.
If $w \not= 0$, then $[z:w] = [ \nicefrac{z}{w}:1 ]$.
\pause
So
\[
\Psi \circ f \circ \Psi^{-1} \bigl([z:w]\bigr)
\pause
=
\Psi \circ f \left(\frac{z}{w}\right)
\pause
=
\left[a \frac{z}{w} + b : c \frac{z}{w} + d \right]
\pause
=
\left[a z + b w: c z + d w \right] .
\]
\pause
The same equality holds if $w=0$.

\medskip
\pause

As
$M \left[ \begin{smallmatrix} z \\ w \end{smallmatrix} \right]
= \left[ \begin{smallmatrix} 
a z + b w \\ c z + d w
\end{smallmatrix} \right]$,
the function $f$ corresponds to the linear map $v \mapsto Mv$ on $\C^2$.

\end{frame}

\begin{frame}[fragile]
Note that any scalar multiple of $M$ gives the same $f$.

\medskip
\pause

An invertible $M$ is a map from
$\C^2 \setminus \{ 0 \}$ to $\C^2 \setminus \{ 0 \}$.

\pause
\medskip

Let $\pi \colon \C^2 \setminus \{ 0 \} \to \C \bP^1$ be the map
$\pi \bigl( (z,w) \bigr) =
[z:w]$. 

\medskip
\pause

We have the commutative diagram:
\begin{equation*}
\begin{tikzcd}
\C^2 \setminus \{ 0 \} \arrow[r,"M"] \arrow[d,"\pi"] &
\C^2 \setminus \{ 0 \} \arrow[d,"\pi"] \\
\C \bP^1 \arrow[r,"\Psi \circ f \circ \Psi^{-1}"] \arrow[d,shift left,"\Psi^{-1}"] &
\C \bP^1 \arrow[d,shift left,"\Psi^{-1}"]
\\
\C_\infty \arrow[r,"f"]\arrow[u,shift left,"\Psi"] &
\C_\infty \arrow[u,shift left,"\Psi"]
\end{tikzcd}
\end{equation*}

\end{frame}

\begin{frame}
The inverse of an LFT is an LFT, simply invert the matrix.

\medskip
\pause

The formula $M^{-1} = \frac{1}{\det M}
\left[ \begin{smallmatrix} 
d & -b \\ -c & a
\end{smallmatrix} \right]$
gives a handy formula for the inverse
of an LFT.  You can also just forget about dividing by the determinant
since everything is up to a multiple.

\medskip
\pause

So LFTs form a group under composition,
the \emph{M\"obius group}.

\medskip
\pause

The group is generated by 
$T_a$, $D_a$, and $I$ for $a\in \C$.
\end{frame}

\begin{frame}
\textbf{Example:}
\emph{Cayley map}
\begin{equation*}
C(z)
=
\frac{z - i}{z + i} .
\end{equation*}
\pause
This map takes $\bH = \{ z \in \C : \Im z > 0 \}$ to $\D$.

\medskip
\pause

Why?  $C(z)$ is in the unit disc if
\begin{equation*}
1 > \abs{\frac{z - i}{z + i} }
=
\frac{\sabs{z - i}}{\sabs{z + i}} ,
\qquad \text{in other words if} \qquad
\sabs{z+i} > \sabs{z - i} .
\end{equation*}

\pause

\begin{center}
\subimport*{../figures/}{cayleydisc.pdf_t}
\end{center}
\end{frame}


\end{document}
