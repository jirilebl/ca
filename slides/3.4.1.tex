\documentclass[10pt,aspectratio=169]{beamer}

% All the boilerplate is in ccaslides.sty
% Note that this also pulls in a custom vogtwidebar.sty
\usepackage{ccaslides}

\author{Ji\v{r}\'i Lebl}

\institute[OSU]{%
Departemento pri Matematiko de Oklahoma {\^S}tata Universitato}

\title{Cultivating Complex Analysis:\\%
Holomorphic functions via integrals (3.4.1)}

\date{}

\begin{document}

\begin{frame}
\titlepage
\end{frame}

\begin{frame}
\begin{lemma}
Suppose $U \subset \C$ is open, and
$\psi \colon U \times [0,1] \to \C$ is a continuous function such that
for each fixed $t \in [0,1]$, the function $z \mapsto \psi(z,t)$ is
holomorphic.  \pause Then
\[
h(z) =
\int_0^1 \psi(z,t) \, dt
\qquad
\text{is a holomorphic function on } U.
\]
\end{lemma}

\pause

There are two common proofs of this kind of result.

\medskip
\pause

\textbf{Proof A:}
Morera together with Fubini and Cauchy--Goursat.

\medskip
\pause

Let $T \subset U$ be a triangle (solid triangle as usual).  \pause Then
\[
\int_{\partial T}
h(z)
\, dz
\pause
=
\int_{\partial T}
\int_0^1 \psi(z,t) \, dt
\, dz
\pause
=
\int_0^1
\int_{\partial T}
\psi(z,t)
\, dz
\, dt
\pause
= \int_0^1 0 \, dt = 0.
\]
\pause

Fubini applies as the integrand is continuous
if we think of each leg of $\partial
T$ separately.

\pause
\medskip

$h(z)$ is holomorphic by Morera.
\qed
\end{frame}

\begin{frame}

\textbf{Proof B:}
Apply Wirtinger derivatives and differentiate under the integral:
\pause
\[
\frac{\partial}{\partial \bar{z}}
\bigl[
h(z)
\bigr]
\pause
=
\frac{\partial}{\partial \bar{z}}
\int_0^1 \psi(z,t) \, dt
\pause
=
\int_0^1
\frac{\partial}{\partial \bar{z}}
\bigl[
\psi(z,t)
\bigr]
\, dt
\pause
= \int_0^1 0 \, dt = 0.
\]
\pause
We are really passing the partial derivatives in $x$
and $y$ under the integral via the Leibniz integral rule (*).

\medskip
\pause

There is a technicality!  Could we apply Leibniz?

\medskip
\pause

We only assumed $z \mapsto \psi(z,t)$ is holomorphic for all $t$.

\pause
\medskip

We did \emph{not} assume
$\frac{\partial \psi}{\partial x}$ and $\frac{\partial \psi}{\partial y}$
(components of 
$\frac{\partial \psi}{\partial \bar{z}}$)
were continuous functions on $U \times [0,1]$.

If we did, we would be done (that's what the standard Leibniz needs).

\medskip
\pause

By an exercise we mentioned previously
(using Cauchy's integral formula for derivatives):

If
$z \mapsto \psi(z,t)$ is holomorphic for all $t$ (and $\psi$ continuous),
then
$\frac{\partial \psi}{\partial x}$ and $\frac{\partial \psi}{\partial y}$
are continuous.

\medskip
\pause

OK, now we are done.
\qed

\end{frame}

\begin{frame}
\begin{corollary}
Suppose $U \subset \C$ is open, $\Gamma$ is a 
chain,
and
$\psi \colon U \times \Gamma \to \C$ is a continuous function such that
for each fixed $w \in \Gamma$, the function $z \mapsto \psi(z,w)$ is
holomorphic.  Then
\[
h(z) =
\int_\Gamma \psi(z,w) \, dw
\qquad \text{is a holomorphic function on } U.
\]
\end{corollary}
\pause

For a continuous $f \colon \partial \Delta_r(p) \to \C$, define
the \emph{Cauchy transform} $Cf \colon \Delta_r(p) \to \C$ by
\begin{equation*}
Cf(z)
\overset{\text{def}}{=}
\frac{1}{2\pi i}
\int_{\partial \Delta_r(p)}
\frac{f(\zeta)}{\zeta-z}\, d\zeta .
\end{equation*}
\pause

\begin{corollary}
For a continuous $f \colon \partial \Delta_r(p) \to \C$,
the Cauchy transform $Cf \colon \Delta_r(p) \to \C$ is holomorphic.
\end{corollary}

\pause

For a random continuous $f$, $Cf$ may not tend to $f$ as we approach
$\partial \Delta_r(p)$.
\end{frame}

\begin{frame}
The corollary gives a converse to Cauchy's formula:

\medskip
\pause

If $f \colon \overline{\Delta_r(p)} \to \C$ is continuous and
\begin{equation*}
f(z) =
\frac{1}{2\pi i}
\int_{\partial \Delta_r(p)} \frac{f(\zeta)}{\zeta-z} \, d\zeta
\qquad \text{for all } z \in \Delta_r(p),
\end{equation*}
\pause
then $f|_{\Delta_r(p)}$ is holomorphic.

\end{frame}




\end{document}
