\documentclass[10pt,aspectratio=169]{beamer}

% All the boilerplate is in ccaslides.sty
% Note that this also pulls in a custom vogtwidebar.sty
\usepackage{ccaslides}

\author{Ji\v{r}\'i Lebl}

\institute[OSU]{%
Departemento pri Matematiko de Oklahoma {\^S}tata Universitato}

\title{Cultivating Complex Analysis:\\%
Cauchy's formula in a disc (3.2.4)}

\date{}

\begin{document}

\begin{frame}
\titlepage
\end{frame}

\begin{frame}

We get to perhaps the most fundamental theorem in complex analysis:

The Cauchy integral formula.

\pause
\medskip

We prove a version in a disc.

\pause
\medskip

\begin{theorem}[Cauchy integral formula in a disc]
Suppose $U \subset \C$ is open, $f \colon U \to \C$ is holomorphic,
$\overline{\Delta_r(p)} \subset U$.
Then for all $z \in \Delta_r(p)$,
\begin{equation*}
f(z)
=
\frac{1}{2\pi i}
\int_{\partial \Delta_r(p)}
\frac{f(\zeta)}{\zeta-z}
\,
d \zeta .
\end{equation*}
\end{theorem}

\medskip
\pause

It should be surprising:

\pause
\medskip

Values inside (large set)

\pause
are given in terms of values on the boundary (small set).

\medskip
\pause

A quick (but hardly only) application is to compute integrals of expressions
such as $\frac{\cos(z^2)}{z(z-1)}$
that blow up somewhere inside the cycle.

\end{frame}

\begin{frame}
\textbf{Proof:}
Fix $z \in \Delta_r(p)$, and write $\gamma = \partial \Delta_r(p)$ oriented counterclockwise.

\medskip
\pause

Let $\Delta_s(z)$ be a small disc where $\overline{\Delta_s(z)} \subset
\Delta_r(p)$.
\pause
Write $\alpha = \partial \Delta_s(z)$.

\pause
\medskip

Connect $\alpha$ to $\gamma$
via two straight lines

to give two closed paths
$c_1$ and $c_2$

\vspace*{-0.35in}
\hspace*{2.6in}%
\subimport*{../figures/}{disccauchy.pdf_t}

\pause
\vspace*{-1.8in}

As chains, $c_1+c_2 = \gamma - \alpha$.

\medskip
\pause

Each $c_j$ lies in a star-like domain

where $\displaystyle \zeta \mapsto \frac{f(\zeta)}{\zeta-z}$ is holomorphic.
\pause
So

\medskip

$
\displaystyle
\int_{\gamma} \frac{f(\zeta)}{\zeta-z} \, d\zeta -
\int_{\alpha} \frac{f(\zeta)}{\zeta-z} \, d\zeta =
$
\medskip

\hspace*{0.6in}
$
\displaystyle
\int_{c_1} \frac{f(\zeta)}{\zeta-z} \, d\zeta + 
\int_{c_2} \frac{f(\zeta)}{\zeta-z} \, d\zeta = 0 + 0 .
$

\medskip

\hspace*{1.4in}
(by Cauchy's theorem)


\end{frame}

\begin{frame}
So
\[
\frac{1}{2\pi i}
\int_{\gamma} \frac{f(\zeta)}{\zeta-z} \, d\zeta =
\frac{1}{2\pi i}
\int_{\alpha} \frac{f(\zeta)}{\zeta-z} \, d\zeta .
\]
\pause

Parametrize $\alpha$ as $\alpha(t) = z+s e^{i t}$.

\pause

\[
\frac{1}{2\pi i}
\int_{\alpha}
\frac{f(\zeta)}{\zeta-z}
\,
d \zeta
\pause
=
\frac{1}{2\pi i}
\int_0^{2\pi} \frac{f(z+se^{it})}{z + se^{it} - z} s i e^{it} \, dt
\pause
=
\frac{1}{2\pi}
\int_0^{2\pi} f(z+se^{it}) \, dt .
\]

\pause

The integral over $\gamma$ (independent of $s$)
equals the integral over $\alpha$ for all $s > 0$ small enough,

\pause

so we can take the limit as $s \to 0$.

\pause
\medskip

\begin{equation*}
\frac{1}{2\pi i}
\int_{\gamma}
\frac{f(\zeta)}{\zeta-z}
\,
d \zeta
\pause
=
\lim_{s \downarrow 0}
\frac{1}{2\pi i}
\int_{\alpha}
\frac{f(\zeta)}{\zeta-z}
\,
d \zeta
\pause
=
\lim_{s \downarrow 0}
\frac{1}{2\pi}
\int_0^{2\pi} f(z+se^{it}) \, dt
\pause
=
f(z) .
\end{equation*}
\hspace*{3.8in}(by continuity of $f$ at $z$) \qed

\end{frame}

\begin{frame}
There is a very useful consequence left as an easy exercise:

\medskip
\pause

\textbf{Exercise:}
Suppose $f$ is holomorphic in an open neighborhood of
$\overline{\Delta_r(p)}$.
Show that $f$ at $p$ is the average of the values on $\partial \Delta_r(p)$.
That is, show
\[
f(p) = \frac{1}{2\pi} \int_0^{2\pi} f(p + r e^{it}) \, dt .
\]




\end{frame}

\end{document}
