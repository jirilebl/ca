\documentclass[10pt,aspectratio=169]{beamer}

% All the boilerplate is in ccaslides.sty
% Note that this also pulls in a custom vogtwidebar.sty
\usepackage{ccaslides}

\author{Ji\v{r}\'i Lebl}

\institute[OSU]{%
Departemento pri Matematiko de Oklahoma {\^S}tata Universitato}

\title{Cultivating Complex Analysis:\\%
Inverse function theorem and automorphisms (2.2.3)}

\date{}

\begin{document}

\begin{frame}
\titlepage
\end{frame}

\begin{frame}
Let $U, V \subset \C$ be open.

\medskip
\pause

If $f \colon U \to V$ is holomorphic and bijective and
$f^{-1}$ is holomorphic, then $f$ is 
called a \emph{biholomorphism}, and say $U$ and $V$
are \emph{biholomorphic}.

\pause
(surprisingly, we will later see that $f^{-1}$ is automatically holomorphic).

\medskip
\pause

If $U = V$, then $f$ is called an \emph{automorphism}.

\pause
(Remark that the word ``automorphism'' appears in many different contexts.)

\medskip
\pause

Denote the set of automorphism by $\Aut(U)$, which is a group under
composition (exercise).

\medskip
\pause

\textbf{Remark:} Traditionally, a biholomorphism is called a
\emph{conformal mapping} and biholomorphic $U$ and $V$ are said to be
\emph{conformally equivalent}.

\medskip
\pause

\textbf{Example:}
The Cayley map
$C(z)
=
\frac{z - i}{z + i}$
takes $\bH = \{ z \in \C : \Im z > 0 \}$ to $\D$, and has an inverse.

That is, $C|_{\bH} \colon \bH \to \D$ is a biholomorphism.

\medskip
\pause

\textbf{Example:}
For any $a,b \in \C$, $a \not= 0$, the function $a z + b$ is an automorphism of $\C$.
\end{frame}

\begin{frame}
Suppose $f$ is a biholomorphism.

\pause
\medskip

Differentiate
$f^{-1}\bigl(f(z)\bigr) = z$ to find $(f^{-1})' \bigl(f(z)\bigr) f'(z) = 1$.

\pause
\medskip

So $f'(z) \not= 0$ for all $z$, \pause and
if $w = f(z)$, then
\[
(f^{-1})'(w) = \frac{1}{f'(z)}
\qquad \text{or} \qquad
f'(z) =
\frac{1}{(f^{-1})'(w)} .
\]

\medskip
\pause

Consider a holomorphic 
$f = u+iv$.  \pause Its real derivative is (as usual $z=x+iy$)
\[
Df =
\begin{bmatrix}
\frac{\partial u}{\partial x} & \frac{\partial u}{\partial y} \\[5pt]
\frac{\partial v}{\partial x} & \frac{\partial v}{\partial y}
\end{bmatrix} .
\]
\pause
Using the Cauchy--Riemann equations, we compute the Jacobian determinant,
\[
\det Df \pause
=
\frac{\partial u}{\partial x}
\frac{\partial v}{\partial y} -
\frac{\partial u}{\partial y} 
\frac{\partial v}{\partial x}
\pause
=
{\left(\frac{\partial u}{\partial x}\right)}^2
+
{\left(\frac{\partial v}{\partial x}\right)}^2
\pause
=
\babs{f'(z)}^2 .
\]
\pause
$\det Df$ is nonzero (positive) and $Df$ is invertible $\Leftrightarrow$
$f'(z) \not= 0$.

\medskip
\pause

Note that we also found that a holomorphic $f$ preserves
orientation (positive $\det Df$).
\end{frame}

\begin{frame}
The real inverse function theorem
for continuously differentiable
functions of $\R^2$ to $\R^2$
says that if $Df$ is invertible at $p$, then $f$ takes a
neighborhood $V$ of $p$ bijectively to a neighborhood $f(V)$
of $f(p)$ and the inverse on that neighborhood is continuously
differentiable with $D(f^{-1})|_{f(p)} = (Df|_p)^{-1}$.

\medskip
\pause

If a $2 \times 2$ matrix represents a complex number, its inverse
represents the reciprocal.

\medskip
\pause

So in the theorem, if $f$ is holomorphic, then so is $f^{-1}$.
\pause
We proved:

\begin{theorem}[Inverse function theorem for holomorphic functions]
Suppose $U \subset \C$ is open, $f \colon U \to \C$ is holomorphic,
$p \in U$, and $f'(p) \not= 0$.  Suppose further that $f$ is continuously
differentiable.
\pause
Then there exist open sets $V, W \subset \C$ such that
$p \in V \subset U$, $f(V) = W$, the restriction $f|_V$ is injective
(one-to-one),
and hence a $g \colon W \to V$ exists such that
$g(w) = (f|_V)^{-1}(w)$ for all $w \in W$.
\pause
Furthermore, $g$ is holomorphic and
\begin{equation*}
g'(w) = \frac{1}{f'\bigl(g(w)\bigr)} \qquad \text{for all $w \in W$}.
\end{equation*}
\end{theorem}

\pause

\textbf{Remark:} We will prove later a holomorphic $f$ is always
continuously differentiable.
\end{frame}

\begin{frame}
Just because $f'$ is never zero doesn't mean that 
$f$ is (globally) bijective.

\medskip
\pause

$e^z$ is never zero (so neither is its derivative),
but $e^{z} = e^{z+2\pi i}$.

\medskip
\pause

But locally we can invert $e^z$ and we will obtain
(locally) the complex logarithm.

\medskip
\pause

\textbf{Remark:}
Locally, there are lots of local biholomorphisms
(there are lots of holomorphic functions whose derivative is nonzero at a point).

\medskip
\pause

However, for a fixed $U$ and $V$, there are very few biholomorphisms of $U$
and $V$.

\medskip
\pause

E.g., if $f \in \Aut(\D)$, then $f(z) = e^{i\theta} \frac{z-a}{1-\bar{a}z}$,
\pause
if $f \in \Aut(\C)$, then $f(z) = a z + b$.

\medskip
\pause

And comparatively, $\D$ and $\C$ have more automorphisms
than most. 

\medskip
\pause

E.g., a disc minus three generically placed points only has the identity 
automorphism.

\medskip
\pause

(Although we don't yet have enough machinery to prove these statements.)
\end{frame}

\end{document}
