\documentclass[10pt,aspectratio=169]{beamer}

% All the boilerplate is in ccaslides.sty
% Note that this also pulls in a custom vogtwidebar.sty
\usepackage{ccaslides}

\author{Ji\v{r}\'i Lebl}

\institute[OSU]{%
Departemento pri Matematiko de Oklahoma {\^S}tata Universitato}

\title{Cultivating Complex Analysis:\\%
Cauchy for star-like sets (3.2.3)}

\date{}

\begin{document}

\begin{frame}
\titlepage
\end{frame}
\begin{frame}
\begin{definition}
A set $U \subset \C$ is called \emph{{star-like}} (or
\emph{star-like with respect to $z_0$}) if there exists a
point $z_0 \in U$ such that the segment $[z_0,z] \subset U$ for every
$z \in U$.
\end{definition}

\medskip
\pause

\begin{center}
\subimport*{../figures/}{starshaped.pdf_t}
\end{center}

\medskip
\pause

A convex set is star-like, but not vice versa.

\end{frame}

\begin{frame}

\begin{proposition} \label{prop:primitiveinstarlike1}
Suppose $U \subset \C$ is open and star-like,
$f \colon U \to \C$ is continuous, and
\begin{equation*}
\int_{\partial T} f(z) \, dz = 0
\qquad \text{for every triangle $T \subset U$.}
\end{equation*}
Then $f$ has a primitive:
There exists a holomorphic $F \colon U \to \C$
such that $F' = f$.
\end{proposition}

\pause

\textbf{Proof:}
Suppose $U$ is star-like with respect to $z_0 \in U$.
\pause
Define
\begin{equation*}
F(z) = \int_{[z_0,z]} f(\zeta) \, d\zeta .
\end{equation*}
\pause

\vspace*{-0.2in}
Consider a disc $\Delta_r(z) \subset U$,

and $\sabs{h} < r$ so that $z+h \in \Delta_r(z)$.

\pause
\medskip

$U$ is star-like w.r.t. $z_0$ $\Rightarrow$

the entire triangle with vertices

$z_0$, $z$, and $z+h$ is in $U$.

\vspace*{-1.2in}

\hspace*{2.2in}%
\subimport*{../figures/}{triangantidef.pdf_t}
\end{frame}

\begin{frame}

By hypothesis $\displaystyle \int_{[z_0,z]+[z,z+h]-[z_0,z+h]} f(\zeta) \,
d\zeta = 0 $.
\pause
So
\[
\frac{F(z+h)-F(z)}{h}
\pause
=
\frac{1}{h}
\int_{[z_0,z+h]-[z_0,z]} f(\zeta) \, d\zeta
\pause
=
\frac{1}{h}
\int_{[z,z+h]} f(\zeta) \, d\zeta
\hspace*{2.0in}
\]
\[
\pause
\hspace*{1in}
=
\frac{1}{h}
\int_0^1 f(z+th) h \, dt
\pause
=
\int_0^1 f(z+th) \, dt .
\]
\pause
\vspace*{-0.1in}
In other words,
\vspace*{-0.1in}
\[
\abs{
\frac{F(z+h)-F(z)}{h} 
-
f(z)
}
=
\abs{
\int_0^1 f(z+th) \, dt
-
\int_0^1 f(z) \, dt
}
\]
\pause
\vspace*{-0.05in}
\[
\hspace*{1.08in}
 \leq
\int_0^1 \abs{f(z+th)-f(z)} \, dt .
\]
\pause
By continuity of $f$ at $z$,
\begin{equation*}
\lim_{h \to 0}
\frac{F(z+h)-F(z)}{h} 
=
f(z) . \qed
\end{equation*}

\end{frame}

\begin{frame}

Cauchy--Goursat (the integral around
triangles is zero if $f$ is holomorphic) implies

\pause

\begin{corollary} \label{cor:primitiveinstarlike}
Suppose $U \subset \C$ is open and star-like
and $f \colon U \to \C$ is holomorphic.
Then $f$ has a primitive.
%There exists a holomorphic $F \colon U \to \C$
%such that $F' = f$.
\end{corollary}

\pause

\begin{theorem}[Cauchy's theorem for star-like domains]
Suppose $U \subset \C$ is open and star-like, $f \colon U \to \C$ is holomorphic,
and $\Gamma$ is
a cycle
in $U$.  Then
\begin{equation*}
\int_{\Gamma} f(z) \, dz = 0 .
\end{equation*}
\end{theorem}

\pause

\textbf{Proof:}
The Corollary above says there is a primitive
$F \colon U \to \C$.

\pause

By Cauchy's theorem for derivatives, the integral is zero. \qed

\pause

\medskip

\textbf{Remark:}
$\C$-valued function gives a vector-field on $\R^2$.

The corollary is a special case of a theorem from vector calculus:

\medskip

\emph{In a star-like domain $U \subset \R^2$, if a
$C^1$ vector field $(u,v) \colon U \to \R^2$
satisfies $\frac{\partial u}{\partial y} = \frac{\partial v}{\partial x}$
(\emph{irrotational}),
then there exists a real-valued $f \colon \R^2 \to \R$ such that
$\nabla f = (u,v)$ (conservative vector field).}

%More concisely, \emph{an irrotational vector field
%in a star-like domain is conservative.}

\end{frame}




\end{document}
