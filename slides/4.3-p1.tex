\documentclass[10pt,aspectratio=169]{beamer}

% All the boilerplate is in ccaslides.sty
% Note that this also pulls in a custom vogtwidebar.sty
\usepackage{ccaslides}

\author{Ji\v{r}\'i Lebl}

\institute[OSU]{%
Departemento pri Matematiko de Oklahoma {\^S}tata Universitato}

\title{Cultivating Complex Analysis:\\%
Simply connected domains (4.3 part 1)}

\date{}

\begin{document}

\begin{frame}
\titlepage
\end{frame}

\begin{frame}
\begin{definition}
A domain $U \subset \C$ is \emph{simply connected}
if every cycle in $U$
is homologous to zero in $U$.
\end{definition}

\pause
That is, $U$ is simply connected if $n(\Gamma;p) = 0$ for every cycle $\Gamma$ in $U$ and
every $p \in \C \setminus U$.

\pause
\medskip
\textbf{Exercise:}

a) star-like domains (e.g., $\C$, $\D$, and $\bH$) in $\C$ are simply connected.

\pause
b) $\C \setminus \{ 0 \}$ is not simply connected.

\pause
\medskip

A few remarks are in order:

\medskip
\pause

\textbf{Remark:}
This is the ``wrong'' definition.  It happens to work in the setting of
domains in $\C$.  It is wrong for two reasons:

\pause
\medskip

1) It is in terms of homology and not homotopy (an optional
section in the book).

\medskip
\pause

One could say \emph{simply connected in the sense of homology} to emphasize.

\medskip
\pause
2) We defined cycles as ``piecewise-$C^1$'' instead of ``continuous.''

\medskip
\pause

\textbf{Remark:}
Can a disconnected set be simply connected?
We remain neutral on this.

\end{frame}

\begin{frame}
Simply connected domains satisfy Cauchy's theorem.

\begin{theorem}[Cauchy's theorem (simply connected version)]
Let $U \subset \C$ be a simply connected domain and $f \colon U \to \C$
holomorphic. \pause  If $\Gamma$ is a cycle in $U$, then
\[
\int_\Gamma f(z) \, dz = 0 .
\]
\end{theorem}

\pause

\textbf{Proof:} Since $U$ is simply connected, $n(\Gamma;p) = 0$ for every
$p \in \C \setminus U$, so Cauchy applies. \qed

\medskip
\pause

If we have Cauchy's theorem we expect primitives:

\begin{theorem}
Let $U \subset \C$ be a simply connected domain and
$f \colon U \to \C$ holomorphic. \pause  Then $f$ has a
primitive in $U$.
\end{theorem}

\pause

\textbf{Proof:}
Fix $p \in U$ and note that $U$ is path connected.

\pause
For every $z \in U$, pick
some piecewise-$C^1$ path $\gamma$ from $p$ to $z$, and

\pause
\medskip

Define \qquad
$\displaystyle
F(z) = \int_\gamma f(\zeta) \, d\zeta .
$
\end{frame}

\begin{frame}
If $\alpha$ is another path from $p$ to $z$, then by Cauchy

\medskip

$\displaystyle \quad
\int_\gamma f(\zeta) \, d\zeta -
\int_\alpha f(\zeta) \, d\zeta 
=
\int_{\gamma-\alpha} f(\zeta) \, d\zeta  =  0
$ 
\pause
\quad $\Rightarrow$ \quad
$F(z) = \int_\gamma f(\zeta) \, d\zeta$ does not depend on $\gamma$.

%Let us reduce the proof to the proof for 
%star-like domains (propref{prop:primitiveinstarlike1} and
%corref{cor:primitiveinstarlike}).

\medskip
\pause

Consider $q \in U$, $\gamma$ a path from $p$ to $q$, and $\Delta_r(q) \subset U$.
\pause
For $z \in \Delta_r(q)$ %\quad (as $F$ does not depend on the path),

\medskip

$\displaystyle
\quad
F(z) =
\int_{\gamma+[q,z]} f(\zeta) \, d\zeta
$

\vspace*{-0.4in}
\hspace*{2.2in}\subimport*{../figures/}{anyantidef.pdf_t}

\vspace*{-0.85in}
\pause

$\displaystyle
\phantom{\quad F(z)}
=
\int_{\gamma} f(\zeta) \, d\zeta
+
\int_{[q,z]} f(\zeta) \, d\zeta .
$

\medskip
\pause

The first term is a constant.

\medskip
\pause

The second term is how we defined

a primitive in a star-like domain ($\Delta_r(q)$).

See Proposition 3.2.11.
\qed

\end{frame}

\begin{frame}
\begin{corollary}
Let $U \subset \C$ be a simply connected domain and
$f \colon U \to \C$ nowhere zero and holomorphic.

\pause
Then there exists a holomorphic $g \colon U \to \C$
such that
\[
e^{g(z)} = f(z) .
\]
\end{corollary}

\pause

\textbf{Example:}
If $U \subset \C \setminus \{ 0 \}$ is a simply connected domain, then
$\exists$ a holomorphic
$L \colon U \to \C$ such that
$e^{L(z)} = z$ (a branch of the log).

\medskip
\pause

\textbf{Proof:}
$\dfrac{f'(z)}{f(z)}$ is holomorphic on $U$.  \pause Find a primitive $g(z)$.
\pause
Then
\[
\frac{d}{dz} \left[ \frac{e^{g(z)}}{f(z)} \right]
\pause =
\frac{ e^{g(z)} g'(z) f(z) - e^{g(z)} f'(z) }{{\bigl(f(z)\bigr)}^2}
\pause =
\frac{ e^{g(z)} f'(z) - e^{g(z)} f'(z) }{{\bigl(f(z)\bigr)}^2}
\pause =
0 .
\]
\pause
$\Rightarrow$ \quad $\dfrac{e^{g(z)}}{f(z)}$ is constant.

\medskip
\pause

$\Rightarrow$ \quad $\exists \, C \in \C$ such that
$e^{g(z) + C} = f(z)$.
\qed

\end{frame}

\begin{frame}

If we have the logarithm, we can take roots.

\pause

\begin{corollary}
Let $U \subset \C$ be a simply connected domain,
$f \colon U \to \C$ nowhere zero and holomorphic,
and $k \in \N$.

\pause
Then there exists a holomorphic $g \colon U \to \C$
such that
\[
{\bigl(g(z)\bigr)}^k = f(z) .
\]
\end{corollary}

\pause

\textbf{Proof:}
Find a $\psi \colon U \to \C$ such that $e^{\psi(z)} = f(z)$. 

\medskip
\pause

Let $g(z) = e^{\frac{1}{k} \psi(z)}$.

\medskip
\pause

Check:
\qquad
$\displaystyle
{\bigl(g(z)\bigr)}^k
\pause
=
{\left( e^{\frac{1}{k} \psi(z)} \right)}^k
\pause
=
e^{\psi(z)} \pause = f(z)$. \qed

\end{frame}

\end{document}
