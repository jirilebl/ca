\documentclass[10pt,aspectratio=169]{beamer}

% All the boilerplate is in ccaslides.sty
% Note that this also pulls in a custom vogtwidebar.sty
\usepackage{ccaslides}

\author{Ji\v{r}\'i Lebl}

\institute[OSU]{%
Departemento pri Matematiko de Oklahoma {\^S}tata Universitato}

\title{Cultivating Complex Analysis:\\%
Basic calculus (2.2.1)}

\date{}

\begin{document}

\begin{frame}
\titlepage
\end{frame}

\begin{frame}
Let's go through some of the very basic calculus on holomorphic functions.

\medskip
\pause

First, let's solve a differential equation.

\pause

\begin{proposition}
Let $U \subset \C$ be a domain (open and connected),
and $f \colon U \to \C$ be holomorphic, and $f'(z) = 0$ for all $z \in U$.
Then $f$ is a constant.
\end{proposition}

\pause

The proof is just the standard real result, since $f'(z)=0$ implies that
the real derivative is also zero (a zero $2 \times 2$ matrix).
\end{frame}

\begin{frame}
\begin{proposition}[Chain rule]
Let $U \subset \C$ and $V \subset \C$ be open, $f \colon U \to V$
complex differentiable at $z \in U$, and $g \colon V \to \C$ complex differentiable
at $f(z)$.  Then the composition $g \circ f$
is complex differentiable at $z$ and $(g \circ f)'(z) = g'\bigl(f(z)\bigr) f'(z)$.
\end{proposition}

\pause

We give two proofs.  One is an adaptation of the proof of the one-variable result
from real analysis, and the other uses the real result for functions of $\R^2$
to $\R^2$.

\medskip
\pause

\textbf{Proof A:}
Let $h \not= 0$, and let $k = f(z+h) -f(z)$.
Assume first $k\not= 0$.

\medskip
\pause

$\displaystyle
\qquad
\frac{(g \circ f)(z+h) - (g \circ f)(z)}{h}
\pause
 =
\frac{g \bigl( f(z+h) \bigr) - g\bigl( f(z) \bigr)}{f(z+h)-f(z)}
\frac{f(z+h)-f(z)}{h}
$

\pause
\medskip

\phantom{
$\displaystyle
\qquad
\frac{(g \circ f)(z+h) - (g \circ f)(z)}{h}
$}%
$\displaystyle
=
\frac{g \bigl( f(z) + k \bigr) - g\bigl( f(z) \bigr)}{k}
\frac{f(z+h)-f(z)}{h} .
$

\medskip
\pause

A differentiable function is continuous, so $k \to 0$ as $h \to 0$.

\pause
\medskip

If $k=0$, the difference quotient is zero, but $k=0$ only happens (for small $h$)
if $f'(z)=0$.

\pause
\medskip

Multiplication is continuous, so take the limit $h \to 0$ to finish.
\qed

\end{frame}

\begin{frame}

\textbf{Proof B:}
Complex differentiable functions are real differentiable:
Apply the real chain rule.

\medskip
\pause

For $w = f(z) \in V$,
\[
D(g \circ f)|_z = Dg|_w Df|_z .
\]
\pause
The $2 \times 2$ matrices $Dg|_w$ and $Df|_z$ correspond to complex
numbers $g'(w)$ and $f'(z)$.

\medskip
\pause

The product $Dg|_w Df|_z$ of two such matrices again corresponds to a
complex number:

the product of the two numbers, $g'(w) f'(z)$.

\medskip
\pause

So $D(g \circ f)|_z$ corresponds to the 
pertinent complex number.
\qed
\end{frame}

\begin{frame}
Chain rule of the same sort holds if we plug in a real differentiable function of
one variable.

\medskip
\pause

If $\gamma \colon (a,b) \to \C$ is (real) differentiable, where
$\gamma = \alpha + i \beta$, then

\medskip
\pause

write $\gamma' = \alpha' + i \beta'$; could be interpreted as 
a $2 \times 1$ matrix (column vector)
$\left[\begin{smallmatrix}\alpha'\\\beta'\end{smallmatrix}\right]$.

\pause

\begin{proposition}[Chain rule]
Let $U \subset \C$ be open,
$\gamma \colon (a,b) \to U$ (real) differentiable at $t \in (a,b)$,
and $f \colon U \to \C$ complex differentiable at $\gamma(t)$.
Then the composition $f \circ \gamma$ is (real) differentiable
at $t$ and $(f \circ \gamma)'(t) = f'\bigl(\gamma(t)\bigr) \gamma'(t)$.
\end{proposition}

\pause
\textbf{Proof:}
The first proof just works as is, let's see the second proof.

\medskip
\pause

Let $z = \gamma(t)$.  \pause Then 
\[
D(f \circ \gamma)|_{t} =
Df|_z D\gamma|_{t} .
\]
\pause
$Df|_z$ corresponds
to multiplication by $f'(z)$, and $D\gamma|_{t}$
is the $2 \times 1$ matrix (column vector) represented by $\gamma'(t)$.
\pause
The result follows.
\qed
\end{frame}

\begin{frame}
\begin{proposition}
Let $U \subset \C$ be open, and $f \colon U \to \C$ and
$g \colon U \to \C$ holomorphic.
\pause
\begin{enumerate}[(i)]
\item
$f+g$ is holomorphic and $\frac{d}{dz}\bigl[ f(z)+g(z) \bigr] = f'(z) + g'(z)$.
\pause
\item
$fg$ is holomorphic and $\frac{d}{dz}\bigl[f(z) g(z) \bigr] = f'(z)g(z) + f(z)g'(z)$.
\pause
\item
$\nicefrac{1}{g}$ is holomorphic on $\bigl\{ z \in U : g(z) \not= 0 \bigr\}$ and
$\frac{d}{dz}\bigl[\frac{1}{g(z)}\bigr] = \frac{-g'(z)}{{\bigl(g(z)\bigr)}^2}$.
\end{enumerate}
\end{proposition}

\medskip
\pause

\textbf{Proof:} Exercise.  Just adapt the one real variable proof.

\medskip
\pause

\textbf{Remark:} A holomorphic function is continuous so
$\bigl\{ z \in U : g(z) \not= 0 \bigr\}$ is open.
\end{frame}

\begin{frame}
\begin{proposition}[Power rule and its consequences]
\begin{enumerate}[(i)]
\item
\pause
For every integer $n$, the function $z \mapsto z^n$ is holomorphic
where defined (outside the origin if $n$ negative) and
$\frac{d}{dz}\bigl[ z^n \bigr] = n z^{n-1}$
if $n\not=0$ and $\frac{d}{dz}\bigl[ z^0 \bigr] = 0$.
\item
\pause
A polynomial $P(z) = \sum_{n=0}^d c_n z^n$ is
holomorphic and
$P'(z) = \sum_{n=0}^{d-1} (n+1) c_{n+1} z^n$.
\item
\pause
Rational functions $\frac{P(z)}{Q(z)}$
are holomorphic on the set where $Q$ is not zero.
\end{enumerate}
\end{proposition}

\pause

\textbf{Proof:} Again exercise.
\end{frame}

\begin{frame}
Let's mention
an exercise that is easy to do now and relates to an earlier side remark.

\medskip
\pause

Define $f \colon \C \to \C$ by $f(0)=0$ and
$f(z) = e^{-z^{-4}}$ for $z \not=0$.

\medskip
\pause

It is an exercise that
$\frac{\partial u}{\partial x}$,
$\frac{\partial v}{\partial x}$,
$\frac{\partial u}{\partial y}$, and
$\frac{\partial v}{\partial y}$ exist at all points (including
the origin) and satisfy the Cauchy--Riemann equations,
but $f$ is not even continuous at the origin.

\medskip
\pause

The key is of course that $f$ is not differentiable (neither real nor complex)
at the origin.
\end{frame}

\end{document}
