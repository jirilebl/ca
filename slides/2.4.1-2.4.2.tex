\documentclass[10pt,aspectratio=169]{beamer}

% All the boilerplate is in ccaslides.sty
% Note that this also pulls in a custom vogtwidebar.sty
\usepackage{ccaslides}

\author{Ji\v{r}\'i Lebl}

\institute[OSU]{%
Departemento pri Matematiko de Oklahoma {\^S}tata Universitato}

\title{Cultivating Complex Analysis:\\%
Definition (of analytic functions) (2.4.1)\\%
Analytic functions are holomorphic (2.4.2)}

\date{}

\begin{document}

\begin{frame}
\titlepage
\end{frame}


\begin{frame}

Analytic functions are functions equal to
a convergent power series near every point.

\begin{definition}
Let $U \subset \C$ be open.  A function $f \colon U \to \C$
is \emph{analytic}
if for every $p \in U$, there exists 
an $r > 0$ and a
power series $\sum c_n {(z-p)}^n$ converging to $f$ on $\Delta_r(p) \subset
U$.
\end{definition}

\pause

Eventually, we will see that function is holomorphic if and only if it is
analytic.

\pause
\medskip

But that is not so easy.  Today we will prove that analytic $\Rightarrow$
holomorphic.

\pause
\medskip

\textbf{Remark:}
A subtle point is that it is not immediate that a convergent power series is 
analytic.

\end{frame}

\begin{frame}
\begin{proposition}
Let $f \colon \Delta_R(p) \to \C$ be defined by
\begin{equation*}
f(z) = \sum_{n=0}^\infty c_n {(z-p)}^n ,
\qquad \text{converging in } \Delta_R(p) .
\end{equation*}
\pause
Then $f$ is complex differentiable at every $z \in \Delta_R(p)$, and
\begin{equation*}
f'(z) = \sum_{n=1}^\infty n c_n {(z-p)}^{n-1} ,
\qquad \text{converging in } \Delta_R(p) .
\end{equation*}
\end{proposition}

\pause

\textbf{Proof:}
WLOG $p=0$.

\medskip
\pause

Consider the difference quotient of $z^n$ at $z_0$
\[
\frac{z^n-z_0^n}{z-z_0}
=
\sum_{k=0}^{n-1}
z^k z_0^{n-1-k} 
\pause
\quad
\underset{\text{as } z \to z_0}{\to}
\quad n z_0^{n-1} .
\]

\end{frame}

\begin{frame}

Suppose $z_0,z \in \Delta_R(0)$,
\[
\frac{f(z) - f(z_0)}{z-z_0}
\pause
=
\sum_{n=1}^\infty c_n \frac{z^n-z_0^n}{z-z_0}
\pause
=
\sum_{n=1}^\infty c_n \sum_{k=0}^{n-1} z^k z_0^{n-1-k} .
\]
\pause
We need to show that the expression on the right is a continuous
function of $z$ at $z_0$.

\pause

It is continuous if the series
converges uniformly for $z$ in a
neighborhood of $z_0$.

\pause

\hspace*{\fill}
\subimport*{../figures/}{threediscs.pdf_t}

\vspace*{-1.47in}
Suppose $0 < s < r < R$ and $z_0,z \in \Delta_s(0)$.

\pause

\[
\abs{c_n \sum_{k=0}^{n-1} z^k z_0^{n-1-k}}
\leq
\sum_{k=0}^{n-1} 
\sabs{c_n} s^{n-1}
=
n
\sabs{c_n} s^{n-1}
=
n
\sabs{c_n} r^{n-1} {\left(\frac{s}{r}\right)}^{n-1}
\qquad
\qquad
\qquad
\qquad
\]

\pause

$\sabs{c_n} r^{n-1}$ is bounded by some $M > 0$
($f$ converges in
$\Delta_{R}(0)$ and $r < R$)
\pause

\[
\sqrt[n]{
n \sabs{c_n} s^{n-1}
}
=
\sqrt[n]{
n \sabs{c_n} r^{n-1} {\left(\frac{s}{r}\right)}^{n-1}
}
\leq
\sqrt[n]{
n M {\left(\frac{s}{r}\right)}^{n-1}
}
\quad\underset{\text{as } n \to \infty}{\to}\quad
\frac{s}{r} < 1
\qquad
\qquad
\qquad
\qquad
\]
\pause
so
$\sum n \sabs{c_n} s^{n-1}$
converges (root test).

\pause

So
\(
\sum_{n=1}^\infty c_n \sum_{k=0}^{n-1} z^k z_0^{n-1-k} ,
\)
converges uniformly in $z$ on $\Delta_s(p)$.

\end{frame}

\begin{frame}

So we can swap the limits:
\[
\lim_{z \to z_0}
\frac{f(z)-f(z_0)}{z-z_0}
\pause
=
\sum_{n=1}^\infty c_n \sum_{k=0}^{n-1} z_0^k z_0^{n-1-k}
=
\sum_{n=1}^\infty n c_n z_0^{n-1} .
\qquad
\qquad
\qquad
\]
\pause
As $s$ and $r$ were arbitrary we get the result at any $z_0 \in \Delta_R(0)$.
\qed

%\medskip
\pause

\begin{corollary}
Let $f \colon \Delta_R(p) \to \C$ be defined by
\(\displaystyle
f(z) = \sum_{n=0}^\infty c_n {(z-p)}^n
\),
converging in $\Delta_R(p)$.

\pause
Then $f$ is infinitely complex differentiable in $\Delta_R(p)$,
and the $k$\textsuperscript{th} derivative is given by
\begin{equation*}
f^{(k)}(z) = \sum_{n=k}^\infty n(n-1)\cdots(n-k+1) c_n {(z-p)}^{n-k} ,
\qquad \text{converging in } \Delta_R(p) .
\end{equation*}
\pause
Furthermore,
\quad
$\displaystyle
c_n =
\frac{f^{(n)}(p)}{n!}
$.
\end{corollary}

\end{frame}

\begin{frame}

A consequence is that the power series is unique since
the coefficients are unique
\[
c_n =
\frac{f^{(n)}(p)}{n!} .
\]

\medskip
\pause

In fact, $c_n$ depends only on values of $f$ in arbitrarily small
neighborhood of $p$.

\medskip
\pause

Applied to the analytic functions we get:

\begin{corollary}
An analytic function is infinitely complex differentiable, and each
derivative is analytic.
\end{corollary}

\end{frame}

\end{document}
