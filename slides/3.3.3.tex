\documentclass[10pt,aspectratio=169]{beamer}

% All the boilerplate is in ccaslides.sty
% Note that this also pulls in a custom vogtwidebar.sty
\usepackage{ccaslides}

\author{Ji\v{r}\'i Lebl}

\institute[OSU]{%
Departemento pri Matematiko de Oklahoma {\^S}tata Universitato}

\title{Cultivating Complex Analysis:\\%
The maximum modulus principle (3.3.3)}

\date{}

\begin{document}

\begin{frame}
\titlepage
\end{frame}

\begin{frame}
A useful consequence of the Cauchy's integral formula is the
\emph{maximum modulus principle}

(or just \emph{maximum principle}):

\pause

\begin{theorem}[Maximum modulus principle]
Suppose $U \subset \C$ is a domain and
$f \colon U \to \C$ is holomorphic.
\pause
If $\sabs{f(z)}$ achieves a local maximum on $U$, then $f$ is constant.
\end{theorem}

\pause

The basic idea:
Cauchy's integral formula says that $f(z)$ is the average of $f$ on a
small circle centered at $z$.
The average can't be bigger than the numbers we're averaging.

\pause
\medskip

\textbf{Proof:}
Suppose $\sabs{f(z)}$ achieves a local maximum at $p \in U$.  WLOG $p=0$.

\medskip
\pause

Also WLOG suppose $f(0) \geq 0$ (otherwise multiply by some $e^{i\theta}$).

\medskip
\pause

Take a closed disc $\overline{\Delta_r(0)} \subset U$,

where $r$ is small enough so that
$\sabs{f(z)} \leq \sabs{f(0)} = f(0)$ whenever $\sabs{z} \leq r$.

\end{frame}

\begin{frame}
Cauchy's formula says

\medskip
$\displaystyle
\qquad
f(0) = \sabs{f(0)} =
\abs{\frac{1}{2\pi i}
\int_{\partial \Delta_r(0)}
\frac{f(z)}{z} \, dz
}
\pause
 =
\abs{
\frac{1}{2\pi i}
\int_0^{2\pi}
\frac{f(re^{it})}{re^{it}} \, ri e^{it} \, dt
}
$

\pause
$\displaystyle
\qquad\qquad \leq
\frac{1}{2\pi}
\int_0^{2\pi}
\sabs{f(re^{it})}\, dt
\pause
\leq
\frac{1}{2\pi}
\int_0^{2\pi}
f(0)\, dt \pause = f(0) .
$

\pause
\medskip

$\Rightarrow$ all the inequalities above are equalities.

\medskip
\pause

In addition, $f(0)-\sabs{f(re^{it})} \geq 0$ for all $t$, and
\[
\int_0^{2\pi}\Bigl( f(0)-\sabs{f(re^{it})} \Bigr) \ dt = 0 ,
\qquad \pause \Rightarrow \qquad \sabs{f(re^{it})} = f(0) \text{ for all } t.
\]

\pause

Applying Cauchy's formula
again:
\begin{equation*}
\frac{1}{2\pi}
\int_0^{2\pi}
\sabs{f(re^{it})}\, dt
=
f(0)
=
\frac{1}{2\pi}
\int_0^{2\pi}
f(re^{it})\, dt
\end{equation*}
\pause
or
\begin{equation*}
0 =
\Re \int_0^{2\pi}
\Bigl(\sabs{f(re^{it})}-f(re^{it})\Bigr)\, dt
=
\int_0^{2\pi}
\Bigl(\sabs{f(re^{it})}-\Re f(re^{it})\Bigr)\, dt .
\end{equation*}

\end{frame}

\begin{frame}
$\sabs{w} - \Re w \geq 0$ holds for all $w \in \C$.

\medskip
\pause

So $\sabs{f(re^{it})}-\Re f(re^{it}) \geq 0$ for all $t$.

\medskip
\pause

$\displaystyle \int_0^{2\pi}
\Bigl(\sabs{f(re^{it})}-\Re f(re^{it})\Bigr)\, dt = 0$
\qquad $\Rightarrow$ \qquad
$\sabs{f(re^{it})}=\Re f(re^{it})$ for all $t$

\medskip
\pause

$\Rightarrow$ \qquad $\Im f(re^{it}) = 0$ \pause \qquad $\Rightarrow$ \qquad
$f(re^{it})=\sabs{f(re^{it})} = f(0)$ for all $t$.

\medskip
\pause
The above is true for all small enough $r > 0$.

\medskip
\pause

So the set where $f(z) = f(0)$ contains a disc,
and $f$ is constant by the identity theorem.
\qed

\pause


\begin{corollary}[Maximum modulus principle, part deux]
Suppose $U \subset \C$ is nonempty, open, and bounded (so $\widebar{U}$ is compact).
\pause
If $f \colon \widebar{U} \to \C$ is continuous and the restriction $f|_{U}$
is holomorphic, then $\sabs{f(z)}$ achieves a maximum on $\partial U$.
\pause
In
other words,
\begin{equation*}
\sup_{z \in U} \sabs{f(z)} \leq
\sup_{z \in \partial U} \sabs{f(z)} .
\end{equation*}
\end{corollary}

\pause

Proof is an exercise.

\end{frame}

\begin{frame}
There's a version for a miminum if you avoid zeros:

\qquad

\textbf{Exercise:}
(Minimum modulus principle)
Suppose $U \subset \C$ is a domain and
$f \colon U \to \C$ is holomorphic.
If $\sabs{f(z)}$ achieves a local minimum at $p \in U$ and $f(p) \not=0$,
then $f$ is constant.
\end{frame}




\end{document}
