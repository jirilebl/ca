\documentclass[10pt,aspectratio=169]{beamer}

% All the boilerplate is in ccaslides.sty
% Note that this also pulls in a custom vogtwidebar.sty
\usepackage{ccaslides}

\author{Ji\v{r}\'i Lebl}

\institute[OSU]{%
Departemento pri Matematiko de Oklahoma {\^S}tata Universitato}

\title{Cultivating Complex Analysis:\\%
Cauchy--Riemann equations (2.1.2)}

\date{}

\begin{document}

\begin{frame}
\titlepage
\end{frame}

\begin{frame}
Let $U \subset \C$ be open.  Think of $\C$ as $\R^2$ for a moment.

\pause
\medskip

Recall that $f \colon U \to \C$ is (real) differentiable at $z_0 \in U$ if
there exists a real-linear

($2\times 2$ matrix) $Df|_{z_0}$ such that
\[
\lim_{h \to 0} \frac{\sabs{f(z_0+h) - f(z_0) - (Df|_{z_0}) h}}{\sabs{h}} = 0 
\qquad \text{(again $h\in \R^2=\C$).}
\]

\pause

\[
Df|_{z_0} =
\begin{bmatrix}
\frac{\partial u}{\partial x}\big|_{z_0} & \frac{\partial u}{\partial
y}\big|_{z_0} \\[5pt]
\frac{\partial v}{\partial x}\big|_{z_0} & \frac{\partial v}{\partial y}\big|_{z_0}
\end{bmatrix} ,
\qquad \text{where }
f= u + iv
\quad \text{and} \quad
z=x+iy. 
\]
\pause
Question: When does $(Df|_{z_0})h$ correspond to $\xi h$ for some $\xi \in \C$?

\medskip
\pause

Answer: When $Df|_{z_0}$ is of the form
$\left[ \begin{smallmatrix}
a & -b \\ b & a
\end{smallmatrix} \right]$.

\medskip
\pause

I.e., when
\[
\frac{\partial u}{\partial x}\Big|_{z_0} =
\frac{\partial v}{\partial y}\Big|_{z_0}
, \qquad
\frac{\partial v}{\partial x}\Big|_{z_0} =
-\frac{\partial u}{\partial y}\Big|_{z_0} .
\]

\pause
Then
$Df|_{z_0}$ corresponds to multiplication by 
$\xi = \frac{\partial u}{\partial x}\big|_{z_0} + i \frac{\partial v}{\partial
x}\big|_{z_0} = \frac{\partial v}{\partial y}\big|_{z_0} - i \frac{\partial u}{\partial
y}\big|_{z_0}$.

\end{frame}

\begin{frame}
Consequently,
\begin{equation*}
0 = \lim_{h \to 0} \frac{\sabs{f(z_0+h) - f(z_0) - \xi h}}{\sabs{h}} =
\lim_{h \to 0} \abs{\frac{f(z_0+h) - f(z_0)}{h} - \xi} ,
\end{equation*}
\pause
or
\begin{equation*}
\lim_{h \to 0} \frac{f(z_0+h) - f(z_0)}{h} = \xi .
\end{equation*}
\pause
So $f$ is complex differentiable at $z_0$ and $f'(z_0) = \xi$.

\medskip
\pause

We proved that: If $f$ is (real) differentiable at $z_0$, with
$\frac{\partial u}{\partial x}\big|_{z_0} = \frac{\partial v}{\partial
y}\big|_{z_0}$ and $\frac{\partial v}{\partial x}\big|_{z_0} = -\frac{\partial
u}{\partial y}\big|_{z_0}$,

then $f$ is complex differentiable at $z_0$.

\end{frame}

\begin{frame}

Conversely,
if $f$ is complex differentiable at $z_0$, then it is real
differentiable at $z_0$

(the complex derivative $f'(z_0)$ gives the $Df|_{z_0}$).

\medskip
\pause

Further,
$\frac{\partial u}{\partial x}\big|_{z_0} = \frac{\partial v}{\partial
y}\big|_{z_0}$ and $\frac{\partial v}{\partial x}\big|_{z_0} = -\frac{\partial
u}{\partial y}\big|_{z_0}$
must also hold.

\pause

\begin{proposition}
Let $U \subset \C$ be open and $f = u+iv \colon U \to \C$ be a function.
Then
$f$ is complex differentiable at $z_0 \in U$
if and only if
$f$ (real) differentiable at $z_0 \in U$
with
$\frac{\partial u}{\partial x}\big|_{z_0} =
\frac{\partial v}{\partial y}\big|_{z_0}$
and
$\frac{\partial v}{\partial x}\big|_{z_0} =
-\frac{\partial u}{\partial y}\big|_{z_0}$.

\pause
In this case,
$f'(z_0) = 
\frac{\partial u}{\partial x}\big|_{z_0} + i \frac{\partial v}{\partial
x}\big|_{z_0} = \frac{\partial v}{\partial y}\big|_{z_0} - i \frac{\partial
u}{\partial y}\big|_{z_0}$.
\end{proposition}

\end{frame}

\begin{frame}

If the partial derivatives exist and are continuous, then $f$ is real
differentiable, so:

\pause

\begin{corollary}
Let $U \subset \C$ be open and let $f = u+iv \colon U \to \C$ be a function
such that $\frac{\partial u}{\partial x}$, $\frac{\partial u}{\partial y}$, $\frac{\partial
v}{\partial x}$, and $\frac{\partial v}{\partial y}$ exist and are continuous (that is,
$f$ is continuously differentiable).
\pause
Then
\begin{equation} \label{eq:CReqreal}
\frac{\partial u}{\partial x} = \frac{\partial v}{\partial y} , \qquad
\frac{\partial v}{\partial x} = -\frac{\partial u}{\partial y}
\end{equation}
if and only if $f$ is holomorphic (complex differentiable at all $z \in U$),
\pause
or in other words,
\begin{equation*}
f'(z) =
\lim_{h \to 0} \frac{f(z+h) - f(z)}{h}
\qquad
\text{exists for all $z \in U$.}
\end{equation*}
\end{corollary}

\pause

The equations \eqref{eq:CReqreal} are called the
\emph{Cauchy--Riemann equations}.

\medskip
\pause

Complex analysis is the study of their solutions.

\medskip
\pause

\textbf{Remark:} If only the partial derivatives exist but aren't
continuous, the function may fail to be differentiable (or even continuous)
and may not be holomorphic even if it satisfies
\eqref{eq:CReqreal}.

\end{frame}

\begin{frame}
\textbf{Exercise:}
Show that $e^z$ is holomorphic and its complex derivative is $e^z$,

and hence $\sin z$ and $\cos z$ are also holomorphic.

\medskip
\pause

Hint: Note that $e^{x+iy} = e^x \cos y + i e^x \sin y$ and use the
Corollary from previous slide.
\end{frame}

\end{document}
