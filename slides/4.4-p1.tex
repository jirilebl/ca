\documentclass[10pt,aspectratio=169]{beamer}

% All the boilerplate is in ccaslides.sty
% Note that this also pulls in a custom vogtwidebar.sty
\usepackage{ccaslides}

\author{Ji\v{r}\'i Lebl}

\institute[OSU]{%
Departemento pri Matematiko de Oklahoma {\^S}tata Universitato}

\title{Cultivating Complex Analysis:\\%
Laurent series (4.4 part 1)}

\date{}

\begin{document}

\begin{frame}
\titlepage
\end{frame}

\begin{frame}
Laurent series is an expansion for a holomorphic function around a
hole (or a singularity).

\pause
\medskip

Given $0 \leq r_1 < r_2 \leq \infty$ and $p \in \C$, define
\begin{equation*}
\ann(p;r_1,r_2)
\overset{\text{def}}{=}
\{ z \in \C : r_1 < \sabs{z - p} < r_2 \} .
\end{equation*}

\pause

When $0 < r_1 < r_2 < \infty$ we call this set an \emph{annulus}.

\medskip
\pause

When $r_1=0$ or $r_2=\infty$, it's not really what one would call an annulus:

\medskip
\pause

$\ann(p;0,r) = \Delta_r(p) \setminus \{ p \}$ \quad (punctured disc)

\medskip
\pause

$\ann(p;r,\infty) = \C \setminus \overline{\Delta_r(p)}$

\medskip
\pause

$\ann(p;0,\infty) = \C \setminus \{ p \}$ \quad (punctured plane)
\end{frame}

\begin{frame}
Laurent series is series of the form
\[
\sum_{n=-\infty}^{\infty} c_n {(z-p)}^n .
\]

\pause

Note that a Laurent series is a power series if $c_n=0$ for all $n < 0$.

\medskip
\pause

Convergence of a double series such as
\[
\sum_{n=-\infty}^{\infty} a_n
\]
means
\[
\sum_{n=-\infty}^{\infty} a_n
=
\lim_{N\to -\infty}
\sum_{n=N}^{-1} a_n
+
\lim_{M\to \infty}
\sum_{n=0}^{M} a_n .
\]
\pause
For Laurent series we generally have absolute convergence and
the limit can be taken in any way, but it is still useful to split
the series like this.
\end{frame}

\begin{frame}
Write a Laurent series as 
\[
\sum_{n=-\infty}^{\infty} c_n {(z-p)}^n
=
\sum_{n=0}^{\infty} c_n {(z-p)}^n
+
\sum_{n=-\infty}^{-1} c_n {(z-p)}^n
\pause
=
\sum_{n=0}^{\infty} c_n {(z-p)}^n
+
\sum_{n=1}^{\infty} c_{-n} {\left(\frac{1}{z-p}\right)}^n .
\]
\pause
So the Laurent series behaves like two power series:
\pause
One series in $z-p$ and one in $\frac{1}{z-p}$.

\medskip
\pause

E.g., the first part converges in $\Delta_{r_2}(p)$, and the second
in $\C \setminus \overline{\Delta_{r_1}(p)}$, so the full series converges
(uniformly absolutely on compact subsets) in $\ann(p;r_1,r_2)$ if $r_1 < r_2$.

\medskip
\pause

\textbf{Example:}
\[
e^{1/z}
\pause
=
\sum_{n=0}^{\infty} \frac{1}{n!} {\left(\frac{1}{z}\right)}^n
\pause
=
\sum_{n=-\infty}^0 \frac{1}{(-n)!} z^n ,
\pause
\]
converging uniformly absolutely on compact subsets of $\C \setminus \{ 0 \}$.

\medskip
\pause

\textbf{Example:}
\[
\frac{1}{1-z}
\pause
=
\frac{-1}{z}
\frac{1}{1-\frac{1}{z}}
\pause
=
\frac{-1}{z}
\sum_{n=0}^\infty
{\left(\frac{1}{z}\right)}^n
\pause
=
\sum_{n=-\infty}^{-1}
- z^{n} ,
\]
converging uniformly absolutely on compact subsets of  $\ann(0;1,\infty) = \C \setminus \overline{\D}$.
\end{frame}

\begin{frame}
We will prove that the Laurent series is unique, and so
as for power series,

it does not matter how we obtain it.

\medskip
\pause

We computed some examples in an ad hoc way,

but those are the unique Laurent series for those functions.

\medskip
\pause

We will prove that the coefficients can be computed via
an integral.

\medskip
\pause

However, computation of Laurent series is often done by other means
than by computation of the integral; it is often done as we did above.

 %(e.g., addition, multiplying
%by constants, composition and multiplication by powers of $(z-p)$, etc.).

\medskip
\pause

One of the main applications of complex analysis in engineering
is to compute integrals by computing certain coefficients
of the Laurent series by other means than integration.
\end{frame}

\end{document}
