\documentclass[10pt,aspectratio=169]{beamer}

% All the boilerplate is in ccaslides.sty
% Note that this also pulls in a custom vogtwidebar.sty
\usepackage{ccaslides}

\author{Ji\v{r}\'i Lebl}

\institute[OSU]{%
Departemento pri Matematiko de Oklahoma {\^S}tata Universitato}

\title{Cultivating Complex Analysis: Chapter 1\\%
Linear fractional transformations (1.4 part 1)}

\date{}

\begin{document}

\begin{frame}
\titlepage
\end{frame}

\begin{frame}
$\displaystyle
f(z) = \frac{a z + b}{c z + d}
\qquad \text{where} \quad ad \not= bc
$
\medskip

is a \emph{linear fractional transformation} (LFT), (or \emph{M\"obius
transformation}).

\medskip
\pause

The condition on $a$, $b$, $c$, $d$ guarantees that the ratio does not
simplify.

\medskip
\pause

If $c=0$, $f$ can be written as $f(z)=az+b$.

\medskip

Define $f(\infty) = \infty$.

\medskip
\pause

If $c\not=0$, the expression is defined on
$\C \setminus \bigl\{ \nicefrac{-d}{c} \bigr\}$.

\medskip

Define $f\bigl(\nicefrac{-d}{c}) = \infty$ and 
$f(\infty) = \frac{a}{c}$.

\medskip
\pause

So takes the Riemann sphere to Riemann sphere.

\medskip

It is an easy exercise that $f \colon \C_\infty \to \C_\infty$ is bijective and continuous.

\end{frame}

\begin{frame}
Any LFT is a composition of:

\medskip
\pause

\emph{translations}
\begin{equation*}
T_a(z) = z + a , \qquad a \in \C,
\end{equation*}

\medskip
\pause

\emph{complex dilations}
\begin{equation*}
D_a(z) = az , \qquad a \in \C \setminus \{ 0 \} ,
\end{equation*}

\medskip
\pause

and \emph{inversions}
\begin{equation*}
I(z) = \frac{1}{z}.
\end{equation*}

\medskip
\pause

Consider an LFT $f(z) = \frac{az+b}{cz+d}$.  WLOG $c=1$ or $c=0$.

\medskip
\pause

If $c=1$, then
\begin{equation*}
f(z)
=
\frac{a z + b}{z + d}
=
\frac{b-ad}{z+d}+a
=
T_a\biggr(D_{b-ad}\Bigr(I\bigl(T_d(z)\bigr)\Bigr)\biggr) .
\end{equation*}

\pause

If $c=0$, assume $d=1$ and $f(z) = az + b$:
\begin{equation*}
f(z) = az+b = T_b\bigl(D_a(z)\bigr) .
\end{equation*}
\end{frame}

\begin{frame}
A translation $T_a$ just moves everything in $\C$ in the $a$ direction and fixes
$\infty$.

\medskip
\pause

Complex dilation $D_a$ is the normal euclidean dilation (scaling) by
$\sabs{a}$ and rotation by $\arg a$.  It also fixes $\infty$.

\medskip
\pause

The inversion is the euclidean plane geometry inversion across the unit
circle and then a conjugation.

\medskip
\pause

The euclidean inversion inverts the distance:

\medskip

$\displaystyle
\qquad \qquad
\frac{1}{\sabs{z}} e^{i \arg z} = 
\frac{\sabs{z}}{\sabs{z}^2} e^{i \arg z} = \frac{1}{\bar{z}} .
$
\medskip

To get complex inversion we conjugate.

\vspace*{-0.97in}
\hspace*{3in}
\subimport*{../figures/}{inversion.pdf_t}

\pause
\vspace*{-0.15in}

A use for this decomposition (exercise):

\emph{LFTs take the set of straight lines and circles to
the set of
straight lines and circles.}

\medskip
\pause

In fact, any straight line or circle
can be taken to any other straight line or circle.

\pause
\medskip

\textbf{Remark:} A straight line is really a very large circle through $\infty$.

\end{frame}

\end{document}
