\documentclass[10pt,aspectratio=169]{beamer}

% All the boilerplate is in ccaslides.sty
% Note that this also pulls in a custom vogtwidebar.sty
\usepackage{ccaslides}

\author{Ji\v{r}\'i Lebl}

\institute[OSU]{%
Departemento pri Matematiko de Oklahoma {\^S}tata Universitato}

\title{Cultivating Complex Analysis:\\%
The open mapping theorem (5.5)}

\date{}

\begin{document}

\begin{frame}
\titlepage
\end{frame}

\begin{frame}
A continuous function from $\R^2$ to $\R^2$ can do all
sorts of things to the topology.

\medskip
\pause

$(x,y) \mapsto (x,xy)$,
takes all of $\R^2$, which is both open and closed, to the
set $\bigl\{ (x,y) : x \not= 0 \text{ or } y=0 \bigr\}$, which is neither open nor
closed.

\medskip
\pause

Holomorphic functions are always nice to your topology.

\medskip
\pause

For a continuous map, $f^{-1}(V)$ is open whenever $V$ is.

\medskip
\pause

For a holomorphic map, $f(V)$ is open whenever $V$ is, unless $f$ is
constant.

\end{frame}

\begin{frame}
\begin{theorem}[Open mapping]
Let $U \subset \C$ be a domain and $f \colon U \to \C$ be
holomorphic and nonconstant.  Then $f(V)$ is an open set for every open set
$V \subset U$.
\end{theorem}

\medskip
\pause

\textbf{Proof:}
Suppose $f$ is not constant.  As $U$ is connected, $f$ is not constant
near every point.

\medskip
\pause

Given $p \in V$, \quad $\exists$ 
$\overline{\Delta_r(p)} \subset V$ and a $\delta > 0$
such that
$\abs{f(z)-f(p)} > \delta$ for all $z \in \partial \Delta_r(p)$.

\medskip
\pause

$z \mapsto f(z)-f(p)$ has at least one zero in $\Delta_r(p)$.

\medskip
\pause

Take $w \in \Delta_{\delta}\bigl(f(p)\bigr)$.  For
all $z \in \partial \Delta_r(p)$,
\begin{equation*}
\abs{\bigl( f(z)-w \bigr) - \bigl( f(z)-f(p) \bigr)} = \abs{f(p)-w} <
\delta < \abs{f(z)-f(p)} .
\end{equation*}
\pause
By Rouch\'e, $z \mapsto f(z)-w$ has at least one
zero in $\Delta_r(p)$. \pause So
\begin{equation*}
\Delta_{\delta}\bigl(f(p)\bigr) \subset
f\bigl(\Delta_r(p)\bigr) . \qed
\end{equation*}
\end{frame}

\begin{frame}
The open mapping theorem is a stronger version of the maximum
principle.

\medskip
\pause

$f(p)$ is in the interior of $f(V)$ for any open neighborhood $V$ of $p$.

\pause
So $\sabs{f(z)}$ cannot achieve a maximum at $p$.

\medskip
\pause

The proof gives the more explicit:

\medskip

$\abs{f(z)-f(p)} > \delta$ for $z \in \partial \Delta_r(p)$,
then
$\Delta_{\delta}\bigl(f(p)\bigr) \subset f\bigl(\Delta_r(p)\bigr)$.
\end{frame}

\end{document}
