\documentclass[10pt,aspectratio=169]{beamer}

% All the boilerplate is in ccaslides.sty
% Note that this also pulls in a custom vogtwidebar.sty
\usepackage{ccaslides}

\author{Ji\v{r}\'i Lebl}

\institute[OSU]{%
Departemento pri Matematiko de Oklahoma {\^S}tata Universitato}

\title{Cultivating Complex Analysis:\\%
Residue theorem (5.3 part 1)}

\date{}

\begin{document}

\begin{frame}
\titlepage
\end{frame}

\begin{frame}
Suppose $f$ has an isolated singularity at $p$.
\pause Expand 
as a Laurent series on $\Delta_r(p) \setminus \{ p \}$:
\[
f(z) = \sum_{n=-\infty}^\infty c_n {(z-p)}^n .
\]
\pause
${(z-p)}^n$ has a primitive in 
$\Delta_r(p) \setminus \{ p \}$ for all $n$ except $n=-1$.

\medskip
\pause

So $c_{-1}$ is the only thing left after integration. \pause It's the ``residue.''

\pause

\begin{definition}
Let the \emph{residue} of $f$ at $p$ be
\begin{equation*}
\operatorname{Res}(f;p)
\overset{\text{def}}{=}
c_{-1} .
\end{equation*}
\end{definition}

\pause

For small enough $s > 0$,
\[
\operatorname{Res}(f;p) 
=
c_{-1}
=
\frac{1}{2\pi i} \int_{\partial \Delta_{s}(p)} f(z) \, dz .
\]
\pause
By Cauchy, we can compute the integral over any cycle to the residues
inside.  That's the Residue theorem.
\end{frame}

\begin{frame}

\begin{theorem}[Residue theorem]
Suppose $U \subset \C$ is open, $S \subset U$ is a finite subset,
and $\Gamma$ is
a cycle
in $U \setminus S$
homologous to zero in $U$.
Suppose $f \colon U \setminus S \to \C$ is holomorphic (isolated
singularities on $S$).
Then
\begin{equation*}
\frac{1}{2\pi i} \int_{\Gamma} f(z) \, dz = \sum_{p \in S} n(\Gamma;p) \operatorname{Res}(f;p) .
\end{equation*}
\end{theorem}

{\small
(Recall, ``homologous to zero in $U$'' means
$n(\Gamma;z) = 0$ for all $z \in \C \setminus U$.}

\medskip

\textbf{Proof:}
Write $S = \{ w_1,\ldots,w_\ell \}$.

\pause
Let $r_1,\ldots,r_\ell$ be positive
such that

$\overline{\Delta_{r_1}(w_1)},\ldots, \overline{\Delta_{r_\ell}(w_\ell)}$
are mutually disjoint
\pause

and
$\overline{\Delta_{r_j}(w_j)} \subset U$ for all $j$.

\pause
\medskip

\vspace*{-1.0in}
\hspace*{2.6in}\scalebox{0.8}{
\subimport*{../figures/}{residues.pdf_t}
}

\vspace*{-20pt}

Define the cycle

\medskip

$
\Lambda = \Gamma \,
- \, n(\Gamma;w_1) \, \partial \Delta_{r_1} (w_1)
\, -
\cdots
- \, n(\Gamma;w_\ell) \, \partial \Delta_{r_\ell} (w_\ell)$.

\pause
\medskip

In the picture \quad 
$n(\Gamma;w_1) = 1$, $n(\Gamma;w_2)=0$, and
$n(\Gamma;w_3) = 2$.

\end{frame}

\begin{frame}

{\small \hfill Recall: 
$\Lambda = \Gamma \,
- \, n(\Gamma;w_1) \, \partial \Delta_{r_1} (w_1)
\, -
\cdots
- \, n(\Gamma;w_\ell) \, \partial \Delta_{r_\ell} (w_\ell)$.}

\medskip
\pause

Winding number is an integral so it is linear:
\[
n(\Lambda;p) = n(\Gamma;p)
- n(\Gamma;w_1) \, n(\partial \Delta_{r_1} (w_1) ; p)
-
\cdots
- n(\Gamma;w_\ell) \, n(\partial \Delta_{r_\ell} (w_\ell) ; p) .
\]
\pause
If $p \notin U$ \wthus $n(\Gamma;p) = 0$ \quad (as $\Gamma$ is homologous to zero in
$U$),

\pause
and for all $j$, \quad $n\bigl( \partial \Delta_{r_j}(w_j) ; p\bigr) = 0$
\quad (as $\overline{\Delta_{r_j}(w_j)} \subset U$).

\medskip
\pause

If $p = w_k \in S$ \wthus $n\bigl( \partial \Delta_{r_j}(w_j) ; p\bigr) = 0$ if $j \not= k$,
and
$n\bigl( \partial \Delta_{r_k}(w_k) ; p\bigr) = 1$.

\pause
\medskip

\thus \quad $n(\Lambda;p) = 0$ for all $p \notin U \setminus S$.

\medskip
\pause

We can apply the Cauchy theorem in $U \setminus S$:
\begin{equation*}
0 = 
\frac{1}{2 \pi i}
\int_\Lambda f(z) \, dz
\pause
=
\frac{1}{2 \pi i}
\int_\Gamma f(z) \, dz
-
\sum_{k=1}^\ell
n(\Gamma;w_k)
\frac{1}{2 \pi i}
\int_{\partial \Delta_{r_k}(w_k)} f(z) \, dz .
\end{equation*}
\pause
Recognize the formula for $c_{-1}$ at $w_k$:
\begin{equation*}
\frac{1}{2 \pi i}
\int_{\partial \Delta_{r_k}(w_k)} f(z) \, dz
= \operatorname{Res}(f;w_k) .
\end{equation*}
\qed
\end{frame}

\end{document}
