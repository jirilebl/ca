\documentclass[10pt,aspectratio=169]{beamer}

% All the boilerplate is in ccaslides.sty
% Note that this also pulls in a custom vogtwidebar.sty
\usepackage{ccaslides}

\author{Ji\v{r}\'i Lebl}

\institute[OSU]{%
Departemento pri Matematiko de Oklahoma {\^S}tata Universitato}

\title{Cultivating Complex Analysis:\\%
Residue theorem (5.3 part 1)}

\date{}

\begin{document}

\begin{frame}
\titlepage
\end{frame}

\begin{frame}
Suppose $f$ has an isolated singularity at $p$.
\pause Expand 
as a Laurent series on $\Delta_r(p) \setminus \{ p \}$:
\[
f(z) = \sum_{n=-\infty}^\infty c_n {(z-p)}^n .
\]
\pause
${(z-p)}^n$ has a primitive in 
$\Delta_r(p) \setminus \{ p \}$ unless $n=-1$.

\begin{definition}
Let the \emph{residue} of $f$ at $p$ be
\begin{equation*}
\operatorname{Res}(f;p)
\overset{\text{def}}{=}
c_{-1} .
\end{equation*}
\end{definition}

We know how to compute $c_{-1}$: For small enough $s > 0$,
\begin{equation*}
\operatorname{Res}(f;p) 
=
\frac{1}{2\pi i} \int_{\partial \Delta_{s}(p)} f(z) \, dz .
\end{equation*}
Via Cauchy's theorem, we relate any integral around
a cycle to the residues that lie inside the cycle.
With that, we can state a theorem that is often used for computing
integrals---even integrals that do not at all seem like line integrals or
have any complex numbers in them.

\end{frame}

\begin{frame}

\begin{theorem}[Residue theorem]
Suppose $U \subset \C$ is open, $S \subset U$ is a finite subset,
and $\Gamma$ is
a cycle
in $U \setminus S$
homologous to zero in $U$.
\footnote{As usual, this statement means that
$n(\Gamma;z) = 0$ for all $z \in \C \setminus U$.}
Suppose $f \colon U \setminus S \to \C$ is holomorphic (isolated
singularities on $S$).
Then
\begin{equation*}
\frac{1}{2\pi i} \int_{\Gamma} f(z) \, dz = \sum_{p \in S} n(\Gamma;p) \operatorname{Res}(f;p) .
\end{equation*}
\end{theorem}

\textbf{Proof:}
Let $w_1,\ldots,w_\ell$ denote the elements of $S$.
Let $r_1,\ldots,r_\ell$ be positive numbers such that
the closed discs
$\overline{\Delta_{r_1}(w_1)},\ldots, \overline{\Delta_{r_\ell}(w_\ell)}$
are mutually disjoint (no pair of them intersects), and
$\overline{\Delta_{r_j}(w_j)} \subset U$ for all $j$.

\scalebox{0.9}{
\subimport*{../figures/}{residues.pdf_t}
}
\end{frame}

\begin{frame}

Proof of residue theorem by putting small discs around all
singularities.  Note that $n(\Gamma;w_1) = 1$, $n(\Gamma;w_2)=0$, and
$n(\Gamma;w_3) = 2$.

Define the cycle
\begin{equation*}
\Lambda = \Gamma \,
- \, n(\Gamma;w_1) \, \partial \Delta_{r_1} (w_1)
\, -
\cdots
- \, n(\Gamma;w_\ell) \, \partial \Delta_{r_\ell} (w_\ell) .
\end{equation*}
We claim that
\begin{equation*}
n(\Lambda;p) = 0
\end{equation*}
for all $p \notin U \setminus S$.
The winding number is defined by an integral and so
\begin{equation*}
n(\Lambda;p) = n(\Gamma;p)
- n(\Gamma;w_1) \, n(\partial \Delta_{r_1} (w_1) ; p)
-
\cdots
- n(\Gamma;w_\ell) \, n(\partial \Delta_{r_\ell} (w_\ell) ; p) .
\end{equation*}
If $p \notin U$, then $n(\Gamma;p) = 0$ as $\Gamma$ is homologous to zero in
$U$,
and as 
$\overline{\Delta_{r_j}(w_j)} \subset U$ for all $j$, we get
$n\bigl( \partial \Delta_{r_j}(w_j) ; p\bigr) = 0$, and the claim follows.
If $p = w_k \in S$, then 
$n\bigl( \partial \Delta_{r_j}(w_j) ; p\bigr) = 0$ if $j \not= k$,
and
$n\bigl( \partial \Delta_{r_k}(w_k) ; p\bigr) = 1$.
The claim again follows.

By the homology version of the Cauchy theorem, we find
\begin{equation*}
0 = 
\frac{1}{2 \pi i}
\int_\Lambda f(z) \, dz
=
\frac{1}{2 \pi i}
\int_\Gamma f(z) \, dz
-
\sum_{k=1}^\ell
n(\Gamma;w_k)
\frac{1}{2 \pi i}
\int_{\partial \Delta_{r_k}(w_k)} f(z) \, dz .
\end{equation*}
We recognize the formula for the $c_{-1}$ term of the Laurent series at
$w_k$, that is,
\begin{equation*}
\frac{1}{2 \pi i}
\int_{\partial \Delta_{r_k}(w_k)} f(z) \, dz
= \operatorname{Res}(f;w_k) .  \qedhere
\end{equation*}
\qed
\end{frame}

\end{document}
