\documentclass[10pt,aspectratio=169]{beamer}

% All the boilerplate is in ccaslides.sty
% Note that this also pulls in a custom vogtwidebar.sty
\usepackage{ccaslides}

\author{Ji\v{r}\'i Lebl}

\institute[OSU]{%
Departemento pri Matematiko de Oklahoma {\^S}tata Universitato}

\title{Cultivating Complex Analysis:\\%
Line integrals (3.1 part 2)}

\date{}

\begin{document}

\begin{frame}
\titlepage
\end{frame}

\begin{frame}
The value of an integral does not depend on the parametrization,
\pause
except for orientation.

\medskip
\pause

Suppose $\gamma \colon [a,b] \to \C$ is $C^1$ and 
$h \colon [c,d] \to [a,b]$ is $C^1$,

$h' > 0$ (increasing),
$h(c)=a$, and $h(d) = b$.

\medskip
\pause

Then $\gamma \circ h$ is a new $C^1$ path (different parametrization of
$\gamma$).

\medskip
\pause

Change of variable $t=h(s)$ says
\[
\int_{\gamma} f(z) \, dz
\pause
=
\int_a^b f\bigl( \gamma(t) \bigr) \gamma'(t) \, dt
\pause
 =
\int_c^d f\Bigl( \gamma\bigl(h(s)\bigr) \Bigr) \gamma'\bigl(h(s)\bigr) h'(s) \, ds
\pause
=
\int_{\gamma \circ h} f(z) \, dz.
\]
\pause
If $h' < 0$, $h(c)=b$ and $h(d)=a$, \pause then
\[
\int_{\gamma} f(z) \, dz =
- \int_{\gamma \circ h} f(z) \, dz.
\]
\end{frame}

\begin{frame}
The general version is more difficult and left as an exercise
(we only morally need it).

\pause

\begin{proposition}[Reparametrization]
Suppose $\gamma \colon [a,b] \to \C$ and $\alpha \colon [c,d] \to \C$ are
piecewise-$C^1$ paths such that
$\gamma\bigl([a,b]\bigr) = \alpha\bigl([c,d]\bigr)$.
Suppose either
\begin{enumerate}[(i)]
\item
$\gamma$ and $\alpha$ are injective, or
\item
$\gamma|_{(a,b]}$ and
$\alpha|_{(c,d]}$ are injective and 
$\gamma(a)=\alpha(c)=\gamma(b)=\alpha(d)$ (simple closed paths).
\end{enumerate}
\pause
Then there exists a strictly monotone continuous $h \colon [c,d] \to [a,b]$ such
that $\gamma\bigl(h(t)\bigr) = \alpha(t)$ for all $t \in [c,d]$.
\pause
Furthermore:
\begin{enumerate}[(i)]
\item
If $h$ is increasing, then for every $f$ continuous on the path,
$\displaystyle
\int_\gamma f(z) \, dz = \int_{\alpha} f(z) \, dz .
$
\item
If $h$ is decreasing, then for every $f$ continuous on the path,
$\displaystyle
\int_\gamma f(z) \, dz = - \int_{\alpha} f(z) \, dz .
$
\end{enumerate}
\end{proposition}
\end{frame}

\begin{frame}
Hence, we often write down the
boundary of some open set as our path

(as long as it's piecewise-$C^1$).

\medskip
\pause

We consider any parametrization going counterclockwise around the interior.

\medskip
\pause

For instance, given a disc $\Delta_r(p)$, we parametrize
the boundary $\partial \Delta_r(p)$ by

$\gamma  \colon [0,2\pi] \to \C$ given by $\gamma(t) = p + re^{it}$,

\medskip
\pause

and then 
\begin{equation*}
\int_{\partial \Delta_r(p)} f(z) \, dz
=
\int_{\gamma} f(z) \, dz .
\end{equation*}
\end{frame}

\begin{frame}
There is also the ``$u$-substitution'' from calculus.
\pause

\begin{proposition}
Let $U,V \subset \C$ be open, $\gamma \colon [a,b] \to V$ 
piecewise-$C^1$ path, $g \colon V \to U$ holomorphic, and $f \colon U \to \C$
continuous.
\pause
Then $g \circ \gamma$ is a piecewise-$C^1$ path in $V$
(possibly with vanishing derivative, however, if $g'$ is zero on $\gamma$) and
\pause
\begin{equation*}
\int_{\gamma} f\bigl(g(z)\bigr) g'(z) \, dz
=
\int_{g \circ \gamma} f(w) \, dw .
\end{equation*}
\end{proposition}

\pause
\textbf{Proof:}
Clearly $g \circ \gamma$ is a piecewise-$C^1$ path (except with perhaps
vanishing derivative).

\medskip
\pause
Apply the chain rule,
$(g \circ \gamma)'(t) = g'\bigl(\gamma(t)\bigr) \gamma'(t)$:
\pause
\[
\int_{\gamma} f\bigl(g(z)\bigr) g'(z) \, dz
\pause
=
\int_a^b
f\Bigl(g\bigl(\gamma(t)\bigr)\Bigr) g'\bigl(\gamma(t)\bigr) \gamma'(t) \, dt
\pause
=
\int_a^b
f\bigl((g \circ \gamma)(t)\bigr) (g \circ \gamma)'(t) \, dt
\pause
=
\int_{g \circ \gamma} f(w) \, dw .
\]
\qed
\end{frame}

\begin{frame}
%\subsection{Arclength integral}
We also integrate with respect to arclength (the $ds$ from calculus).

\medskip
\pause

For an $f$ continuous on a piecewise-$C^1$ path $\gamma \colon [a,b] \to \C$,
we define
\[
\int_\gamma f(z) \, \sabs{dz}
\overset{\text{def}}{=}
\int_a^b f\bigl( \gamma(t) \bigr) \sabs{\gamma'(t)} \, dt .
\]

\pause

For example,
\[
\int_\gamma \sabs{dz} =
\int_a^b \sabs{\gamma'(t)} \, dt
\qquad
\left(
=
\int_\gamma ds \right)
\]
is the length of $\gamma$.
\end{frame}

\begin{frame}
\begin{proposition}[Triangle inequality for line integrals]
Suppose $\gamma \colon [a,b] \to \C$ is 
a piecewise-$C^1$ path and $f$ is a continuous function on
$\gamma$.  Then
\begin{equation*}
\abs{\int_\gamma f(z) \, dz} \leq \int_\gamma \sabs{f(z)} \, \sabs{dz} .
\end{equation*}
\pause
In particular, if $\sabs{f(z)} \leq M$ on $\gamma$ and $\ell = \int_{\gamma} \sabs{dz}$, then
\begin{equation*}
\abs{\int_\gamma f(z) \, dz} \leq M \ell .
\end{equation*}
\end{proposition} 
\pause

\textbf{Proof:}
We estimate
\[
\abs{\int_\gamma f(z) \, dz}
\pause
=
\abs{\int_a^b f\bigl( \gamma(t) \bigr) \gamma'(t) \, dt}
\pause
\leq
\int_a^b \abs{ f\bigl( \gamma(t) \bigr) }  \sabs{ \gamma'(t) } \, dt
\pause
\leq
\pause
M
\int_a^b \sabs{ \gamma'(t) } \, dt .
\]
\qed
\end{frame}

\begin{frame}
Note that while uniform convergence of the functions, $f_n \to f$ passes under the
integral:
\[
\lim_{n\to\infty} \int_{\gamma} f_n(z) \, dz
=
\int_{\gamma} f(z) \, dz
\qquad
\lim_{n\to\infty} \int_{\gamma} f_n(z) \, \sabs{dz}
=
\int_{\gamma} f(z) \, \sabs{dz}
\]
\pause
Uniform convergence of the paths $\gamma_n \to \gamma$ does not (for either
integral):

\pause
\medskip
For instance, there exists (exercise) uniformly convergent $\gamma_n \to 0$ such that
\[
\lim_{n\to\infty} \int_{\gamma_n} \sabs{dz} = \infty
\]

\pause

So arclength is not preserved under uniform convergence of paths.

\medskip
\pause

For this you would also need $\gamma_n'$ to also converge.
\end{frame}

\end{document}
