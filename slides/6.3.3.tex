\documentclass[10pt,aspectratio=169]{beamer}

% All the boilerplate is in ccaslides.sty
% Note that this also pulls in a custom vogtwidebar.sty
\usepackage{ccaslides}

\author{Ji\v{r}\'i Lebl}

\institute[OSU]{%
Departemento pri Matematiko de Oklahoma {\^S}tata Universitato}

\title{Cultivating Complex Analysis:\\%
Cycles around compacts (6.3.3)}

\date{}

\begin{document}

\begin{frame}
\titlepage
\end{frame}

\begin{frame}
Given a compact $K \subset U$, we want a $\Gamma$ in $U \setminus K$
homologous to zero in $U$ that goes around $K$.

\medskip
\pause

\begin{lemma}\label{lemma:patharoundK}
Let $U \subset \C$ be open and suppose that $K \subset U$ is compact and
nonempty.
Then there exists a cycle
$\Gamma$ in $U \setminus K$ such that $n(\Gamma;z) = 1$ for all $z \in K$ and
$n(\Gamma;z) = 0$ for all $z \in \C \setminus U$
 and such that $n(\Gamma;z)$ is $0$ or $1$ for all $z \notin
\Gamma$.
\end{lemma}

\pause
Lots of ideas how to do it, but
proof always involves checking many details.

\medskip
\pause

We will map to the disk, but with a twist.  We'll take $\C_\infty \setminus
K$ 
to the disc, and go around the ``outside'' in the opposite direction:

\begin{center}
\scalebox{0.9}{
\subimport*{../figures/}{cyclearoundcompact.pdf_t}
}
\end{center}

\end{frame}

\begin{frame}
\textbf{Proof:}
$K$ could have infinitely many components.
\pause
For a small $r > 0$, $\exists$ closed discs such that
\[
K \subset K' = \overline{\Delta_r(z_1)} \cup \cdots \cup
\overline{\Delta_r(z_m)} \subset U .
\]
$K'$ is compact and has only finitely many components.
\pause
A $\Gamma$ around $K'$ suffices as $K \subset K'$.

\medskip
\pause

Let $K_1,\ldots,K_n$ be the components of $K'$.
\quad
\pause
$K_1$ and
$K_2 \cup \cdots \cup K_n$ are closed.
\pause

\medskip
If we prove the lemma for $K_1$ and
$U \setminus ( K_2 \cup \cdots \cup K_n )$ to find a cycle
$\Gamma_1$, then we are done:

\pause
\medskip
Repeat the procedure
for each $K_j$ to find $\Gamma_j$ and let $\Gamma = \Gamma_1 + \cdots +
\Gamma_n$.

\medskip
\pause
$n(\Gamma_j;z) = 1$ for all $z \in K_j$
\pause
and
$n(\Gamma_j;z) = 0$ for all $z \in K_\ell$ if $\ell \not= j$.
\pause
So $\Gamma$ works.

\medskip
\pause

So without loss of generality, assume that $K$ is connected.  
\end{frame}


\begin{frame}
Assume $0 \in K$. \pause
Assume $K$ has more than one point.

\medskip
\pause

Let $\varphi \colon \C_\infty \to \C_\infty$, be $\varphi(z) = \frac{1}{z}$ for $z \in \C \setminus \{ 0 \}$,
$\varphi(0) = \infty$ and $\varphi(\infty) = 0$. \pause
Let
\[
V = \varphi(\C_\infty \setminus K) .
\]
\pause
$\infty \notin V$,
\pause
$0 \in V$,
\pause
$V \not= \C$,
\pause
and $\C_\infty \setminus V = \varphi(K)$ is connected.

\pause
So components of $V$ are simply connected (exercise).

\pause
$K$ a union of discs \wthus
$\C_\infty \setminus K$ and thus $V$ has finitely many components $V_1,\ldots,V_m$.

\medskip
\pause

By RMT, $\forall j$, $\exists$
a biholomorphic map from $V_j$ to $\Delta_1(q_j)$ (disjoint).

\pause
Write \[D = \Delta_1(q_1) \cup \cdots \cup \Delta_1(q_m).\]

\pause
So $\exists$ biholomorphic $\psi \colon V \to D$, \pause $q_1=0$ and
$\psi(0)=0=q_1$.

\pause
$\C_\infty \setminus U$ is compact
\pause 
~\thus~
$\varphi(\C_\infty \setminus U) \subset V$ is compact
\pause
~\thus~
$S = \psi\bigl(\varphi(\C_\infty \setminus U)\bigr) \subset D$
is compact.

\pause
$\exists$ $r < 1$ such that
\[
S \subset \Delta_r(q_1) \cup \cdots \cup \Delta_r(q_m)
\]

\pause
Let $\gamma_j(t) = q_j + r e^{-it}$ for $t \in [0,2\pi]$ \quad ($\gamma_j = -\partial
\Delta_r(q_j)$).

\pause
Let $\Gamma_j = \varphi^{-1} \circ \psi^{-1} \circ \gamma_j$, and $\Gamma = \Gamma_1 +
\cdots + \Gamma_m$.
\end{frame}

\begin{frame}
Suppose $p \notin \Gamma$.
\[
n(\Gamma;p) = 
\sum_{j=1}^m
\frac{1}{2\pi i}
\int_{\varphi^{-1} \circ \psi^{-1} \circ \gamma_j}
\frac{1}{z-p} \, dz
\pause
=
\sum_{j=1}^m
\frac{1}{2\pi i}
\int_{\psi^{-1} \circ \gamma_j}
\frac{-1}{(1-\zeta p) \zeta} \, d\zeta
\hspace*{1in}
\]
\pause
\[
\hspace*{1in}
=
\sum_{j=1}^m
\frac{1}{2\pi i}
\int_{\gamma_j}
\frac{-1}{ \bigl( 1-\psi^{-1}(\xi)p \bigr)
\,
\psi^{-1}(\xi)
\,
\psi' \bigl(\psi^{-1}(\xi)\bigr)} \, d \xi .
\]

\pause
Suppose $p \in \C \setminus U$.
\pause
\[
h(\xi) =
\frac{-1}{ \bigl( 1-\psi^{-1}(\xi)p \bigr)
\,
\psi^{-1}(\xi)
\,
\psi' \bigl(\psi^{-1}(\xi)\bigr)}
\]
has two (simple) poles: one at $\psi\bigl(\frac{1}{p}\bigr)$ and one
at $q_1=0$.  \quad (third factor never zero)
\pause
\begin{equation*}
\operatorname{Res}(h;0) = 
\frac{-1}{ \bigl( 1-\psi^{-1}(0)p \bigr)
\,
\psi' \bigl(\psi^{-1}(0)\bigr)}
\,
\frac{1}{\frac{1}{\psi'(\psi^{-1}(0))}}
=
-1
\end{equation*}
\pause
\begin{equation*}
\operatorname{Res}\bigl(h;\psi(1/p)\bigr)
=
\frac{-1}{
\psi^{-1}\bigl(\psi(1/p)\bigr)
\,
\psi' \bigl(\psi^{-1}(\psi(1/p))\bigr)
}
\,
\frac{1}{
\frac{-1}{\psi'(\psi^{-1}(\psi(1/p)))}
\, p
}
=
1 .
\end{equation*}
\pause
$\gamma_1$ goes around $0$, \pause
some $\gamma_j$ goes around 
$\psi\bigl(\frac{1}{p}\bigr) \in S$ (as $r < 1$ is large enough)
\pause
\wthus
$n(\Gamma;p) = 0$
\end{frame}

\begin{frame}
Suppose $p \in K$.
\pause
\medskip

$p \in K$ \wthus $\psi^{-1}(\xi) \not= \frac{1}{p}$ for all $\xi \in D$,

\quad
\pause

\thus
\quad
$\displaystyle
h(\xi) =
\frac{-1}{ \bigl( 1-\psi^{-1}(\xi)p \bigr)
\,
\psi^{-1}(\xi)
\,
\psi' \bigl(\psi^{-1}(\xi)\bigr)}
$
\quad
has only one pole $0$.

\medskip

\pause
\begin{equation*}
n(\Gamma;p) = 
\sum_{j=1}^m
\frac{1}{2\pi i}
\int_{\gamma_j} h(\xi) \, d\xi 
\pause
=
\frac{1}{2\pi i}
\int_{\gamma_1} h(\xi) \, d\xi 
\pause
= - \operatorname{Res}(h;0) = 1 .  \qed
\end{equation*}
($\gamma_1$ traverses the circle backwards)
\end{frame}

\begin{frame}
\begin{theorem}
Let $U \subset \C$ be a domain.  Then
$\C_\infty \setminus U$ is connected if and only if $U$ is simply connected.
\end{theorem}

\pause

\textbf{Proof:}
Forward direction is done (we've just used it above).

\medskip
\pause

Suppose $\C_\infty \setminus U$ is disconnected.

\medskip
\pause

Write $S \cup K = \C_\infty \setminus U$ where
$S$ and $K$ are nonempty, closed, and disjoint.

\medskip
\pause
Assume $\infty \in S$.

\medskip
\pause
$U' = U \cup K$ is open as $S$ is closed,
\quad
\pause
$U' \subset \C$,
\quad
\pause
$K \subset U'$ is compact.

\medskip
\pause

Apply lemma to find a cycle
$\Gamma$ in $U = U' \setminus K$ such that $n(\Gamma;z) = 1$ for all $z \in
K$.

\medskip
\pause

In other words, $\Gamma$ is not homologous to zero in $U$. \qed
\end{frame}

\begin{frame}
\textbf{Exercise:}
Suppose $\{ f_n \}$ is a sequence of holomorphic functions on an open set
$U \subset \C$ that converges uniformly on compact subsets to
a nonconstant
$f \colon U \to \C$.  Let $K \subset U$ be a compact set.  Prove that
for every open neighborhood $V$ of $K$ in $U$ (so $K \subset V \subset U$) there exists
a smaller open neighborhood $W$ (so $K \subset W \subset V$) and an $N \in \N$
such that $f$ and $f_n$ have the same number of zeros in $W$ for all
$n \geq N$.

\end{frame}

\end{document}
