\documentclass[10pt,aspectratio=169]{beamer}

% All the boilerplate is in ccaslides.sty
% Note that this also pulls in a custom vogtwidebar.sty
\usepackage{ccaslides}

\author{Ji\v{r}\'i Lebl}

\institute[OSU]{%
Departemento pri Matematiko de Oklahoma {\^S}tata Universitato}

\title{Cultivating Complex Analysis:\\%
Singularities and the Laurent series (5.2.2)}

\date{}

\begin{document}

\begin{frame}
\titlepage
\end{frame}

\begin{frame}
$\dfrac{1}{z}$ has an isolated singularity at $z=0$.

\medskip
\pause

It is a pole of order $1$.

\medskip
\pause

The Laurent series at $z=0$ is just $\nicefrac{1}{z}$,

and all
coefficients of order less than $-1$ are zero.

\end{frame}

\begin{frame}
OK, we could make this more complicated:

\medskip

$\displaystyle
\frac{1}{z} + \frac{1}{1-z}
=
\sum_{n=-1}^\infty z^n
$

\medskip
\pause

Again, pole of order $1$ at $z=0$,

and all coefficients in the series
of order less than $-1$ are zero.

\end{frame}

\begin{frame}
Or even more complicated:

\medskip

$\displaystyle
\frac{1}{z^3} + \frac{1}{z^2} + \frac{1}{z} + \frac{1}{1-z}
=
\sum_{n=-3}^\infty z^n
$

\medskip
\pause

A pole of order $3$,

and all coefficients in the series
of order less than $-3$ are zero.
\end{frame}

\begin{frame}
Or even more complicated:

\medskip

$\displaystyle
\frac{1}{z^3} + \frac{1}{z^2} + \frac{1}{z} + \frac{1}{1-z}
=
\sum_{n=-3}^\infty z^n
$

\medskip
\pause

A pole of order $3$,

and all coefficients in the series
of order less than $-3$ are zero.
\end{frame}

\begin{frame}
$\displaystyle
e^{1/z}
=
\sum_{n=-\infty}^0 \frac{1}{(-n)!} z^n
$

\medskip

has an essential singularity at $z=0$,

and has nonzero coefficients of all negative orders.
\end{frame}

\begin{frame}
It is not difficult to prove the general statement:

\begin{proposition}
Suppose $f \colon \Delta_r(p) \setminus \{p\} \to \C$ is holomorphic,
and
\begin{equation*}
f(z) = \sum_{n=-\infty}^\infty c_n {(z-p)}^n
\end{equation*}
is the corresponding Laurent series.
\pause
The singularity at $p$ is
\begin{enumerate}[(i)]
\item
\emph{removable} if and only if $c_n = 0$ for all $n < 0$,
\item
\pause
a \emph{pole} of order $k \in \N$ if and only if $c_n = 0$ for all $n < -k$ and
$c_{-k}
\not= 0$,
\item
\pause
\emph{essential} if and only if $c_n \not= 0$ for infinitely many negative $n$.
\end{enumerate}
\end{proposition}

\pause

The proof is left as an exercise.

\pause
Hint: Laurent series is unique, and for a removable singularity
equals the power series.

\end{frame}

\begin{frame}

\begin{definition}
At an isolated singularity, the negative part of the Laurent series
\begin{equation*}
\sum_{n=-\infty}^{-1} c_n {(z-p)}^n 
\end{equation*}
is called the \emph{\myindex{principal part}}.
\end{definition}

\pause

\emph{Observation:}
If $P(z)$ is the principal part of $f(z)$ at $p$, then
$f(z)-P(z)$ has a removable singularity at $p$.
\end{frame}

\begin{frame}
The example $e^{1/z}$ motivates the following concept.

\medskip
\pause

Given an entire $f \colon \C \to \C$, we talk about its
\emph{singularity at infinity.}

\medskip
\pause

$\C \subset \C_{\infty}$ and $\nicefrac{1}{z}$ is a self mapping of
$\C_{\infty}$ that swaps $\infty$ and $0$.

\medskip
\pause

$z \mapsto f(\nicefrac{1}{z})$ has a removable singularity
at $0$, and that's the ``singularity of $f$ at $\infty$.''

\medskip
\pause

$e^z$ has an essential singularity at infinity,

because
$e^{1/z}$ has an essential singularity at $0$.

\end{frame}

\begin{frame}
\textbf{Exercise:}
Prove that if $f$ has a pole at the origin and $g$ has an essential
singularity at the origin, then $f+g$ has an essential singularity at the
origin.

\pause
\medskip

\textbf{Exercise:}
If $f$ has a pole at $p$, then $e^{f(z)}$ has an essential singularity at
$p$.

Hint: First do it for a simple pole.

\pause
\medskip

\textbf{Exercise:}
Show that an entire holomorphic $f \colon \C \to \C$ has a pole at infinity
if and only if it is a nonconstant polynomial.
The order of the pole is the degree of the polynomial.

\pause
\medskip

\textbf{Exercise:}
Show that if $f \colon \C \to \C$ is an automorphism, then
$f(z) = az + b$ for
some constants $a \not= 0$ and $b$.  Hint: Show that $f$ has a simple pole at
infinity.
\end{frame}

\end{document}
