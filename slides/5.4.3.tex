\documentclass[10pt,aspectratio=169]{beamer}

% All the boilerplate is in ccaslides.sty
% Note that this also pulls in a custom vogtwidebar.sty
\usepackage{ccaslides}

\author{Ji\v{r}\'i Lebl}

\institute[OSU]{%
Departemento pri Matematiko de Oklahoma {\^S}tata Universitato}

\title{Cultivating Complex Analysis:\\%
Hurwitz's theorem (5.4.3)}

\date{}

\begin{document}

\begin{frame}
\titlepage
\end{frame}

\begin{frame}
\begin{theorem}[Hurwitz]
Let $U \subset \C$ be open and $f_n \colon U \to \C$ a sequence of
holomorphic functions converging uniformly on compact subsets
to a holomorphic $f \colon U \to \C$.  Suppose $\Gamma$ is
a cycle in $U$ homologous to zero in $U$,
such that $n(\Gamma;z)$ is $0$ or $1$ for all $z \notin \Gamma$.
Suppose $f$ has no zeros on $\Gamma$ and $k$ zeros (counting
multiplicity) in $V = \{ z \in U : n(\Gamma;z) = 1 \}$.
Then there is an $N$ such that for all $n \geq N$,
$f_n$ has $k$ zeros (counting multiplicity) in $V$.
\end{theorem}

\medskip
\pause

\textbf{Proof:}
$\Gamma$ is compact \wthus
there is a $\delta > 0$ such that $\delta < \sabs{f(z)}$
for all $z \in \Gamma$.

\medskip
\pause

$\{ f_n \}$
converges uniformly to $f$ on $\Gamma$.

\medskip
\pause

For $n$ large enough,
\begin{equation*}
\sabs{f(z)-f_n(z)} < \delta < \sabs{f(z)}
\qquad \forall z \in \Gamma
\end{equation*}
\pause
Rouch\'e's theorem
\wthus
$f$ and $f_n$ have the same number of zeros in $V$.  \qed

\medskip
\pause

``No zeros on $\Gamma$'' is necessary:
\quad $\Gamma = \partial \D$,
\quad $f(z) = z-1$, \quad $f_n(z) = z+(1-\frac{1}{n})$.
\end{frame}

\begin{frame}
\textbf{Example:}
For every integer $k > 0$, $\exists$ $N$ such that
$\forall$ $d \geq N$,
\begin{equation*}
P_d(z) = \sum_{n=0}^d \frac{{(-1)}^n}{(2n)!}z^{2n}
\end{equation*}
has exactly $2k$ zeros in $\Delta_{\pi k}(0)$.

\medskip
\pause

Proof:

$P_d$ are the partial sums of the power series of
$\cos(z)$, which has exactly $2k$ zeros in $\Delta_{\pi k}(0)$.
\end{frame}

\begin{frame}
The usual application is for a small disc:

\medskip
\pause

Suppose $\{ f_n \}$ is a sequence of holomorphic
functions on $U$ converging uniformly
on compact sets to $f$.
\pause
Suppose $z_0$ is a zero of
$f$ of order $k$.

\medskip
\pause
Then for a small enough $\Delta_r(z_0)$,
$\exists$ $N$ such that
$\forall$ $n \geq N$,

$f_n$ has $k$ zeros counting multiplicity in
$\Delta_r(z_0)$.

\bigskip
\pause

Hurwitz theorem does not work for real functions.

\medskip
\pause

\textbf{Example:}

$f(x) = x^2$, \quad $f_n(x) = x^2+\frac{1}{n}$.  \quad $f_n \to f$ uniformly, but
$f_n$ never zero.

\medskip
\pause

$f(z) = z^2$, \quad $f_n(z) = z^2+\frac{1}{n}$.  \quad $f_n \to f$ uniformly.

\pause
\medskip

For any $\epsilon > 0$, ~$z^2+\frac{1}{n}$~ has
two zeros in $\Delta_\epsilon(0)$, for large enough $n$:
\quad
$\pm i \sqrt{\frac{1}{n}}$.
\end{frame}

\begin{frame}
Injective holomorphic
mappings are called \emph{\myindex{univalent}}.

\pause

\begin{corollary}
Suppose $U \subset \C$ is a domain and $f_n \colon U \to \C$ are
univalent holomorphic functions that converge uniformly on compact sets
to $f \colon U \to \C$.  Then $f$ is either univalent or constant.
\end{corollary}

\pause
\textbf{Proof:}
Suppose $f$ is nonconstant.
\pause
Suppose $\exists$ distinct $z_1$ and $z_2$ in $U$ such that $f(z_1) =
f(z_2) = w$.

\pause
\medskip

$f-w$ has isolated zeros at $z_1$ and $z_2$.

\medskip
\pause
Consider $\Delta_r(z_1),\Delta_r(z_2) \subset U$,

\pause
$\overline{\Delta_r(z_1)} \cap \overline{\Delta_r(z_2)} = \emptyset$,

\pause 
such that $f-w$ is not zero on
$\overline{\Delta_r(z_1)} \setminus \{ z_1 \}$ or
$\overline{\Delta_r(z_2)} \setminus \{ z_2 \}$.

\medskip
\pause

Hurwitz \wthus for large enough $n$, ~$f_n-w$ has the same number of
zeros in $\Delta_r(z_1)$ as $f-w$. \pause Same for $\Delta_r(z_2)$.

\pause
\medskip
\thus \quad $\exists$ $z_1' \in \Delta_r(z_1)$ and
$z_2' \in \Delta_r(z_2)$ such that $f_n(z_1')=f_n(z_2')=w$.

\pause
\medskip

\thus \quad
$f_n$ not univalent. \qed
\end{frame}

\end{document}
