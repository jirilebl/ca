\documentclass[10pt,aspectratio=169]{beamer}

% All the boilerplate is in ccaslides.sty
% Note that this also pulls in a custom vogtwidebar.sty
\usepackage{ccaslides}

\author{Ji\v{r}\'i Lebl}

\institute[OSU]{%
Departemento pri Matematiko de Oklahoma {\^S}tata Universitato}

\title{Cultivating Complex Analysis\\Complex Numbers (1.1)}

\date{}

\begin{document}

\begin{frame}
\titlepage
\end{frame}

\begin{frame}
\emph{Modern mathematics} takes a false statement:

\medskip

\emph{All polynomials have a root.}

\medskip
\pause

and redefines what it means to have a \emph{root}
to make the statement true.

\medskip
\pause

Examples:

Start with $\N = \{ 1,2,3,\ldots \}$,
and arrive at $\Z$ to solve something like $n+2=1$.

\medskip
\pause

Start with $\Z$ and arrive at $\Q$ to solve
$2x=1$.

\medskip
\pause

Start with $\Q$ and arrive at $\R$ to be able to take limits.

\medskip
\pause

Start with $\R$ and arrive at $\C$ to be able to solve $z^2+1=0$.
\end{frame}

\begin{frame}
Interestingly, analysis with $\C$ is very different to analysis with $\R$.

\medskip
\pause

Mainly once we ask for the complex derivative we get lots of odd statements:

\medskip

\emph{If you can differentiate once, you can differentiate twice.}

\pause
\emph{Every function acts sort of like a linear function.}

\pause
\emph{If all derivatives are zero at a point, the function is constant.}

\pause
etc.

\medskip
\pause

A graduate real analysis course crushes the student's dreams.

\medskip
\pause

A graduate complex analysis course fills the student with unrealistic
optimism.

\end{frame}

\begin{frame}
Definition of the complex numbers (complex field, complex plane):

\[
\C \overset{\text{def}}{=} \R^2
\]

with multiplication:
\begin{align*}
(a,b) + (c,d) & \overset{\text{def}}{=} (a+b,c+d) , \\
(a,b) (c,d) & \overset{\text{def}}{=} (ac-bd,bc+da) .
\end{align*}

\pause

This makes $\C$ into a field.

\pause
\medskip

$\R \subset \C$ by identifying $x \in \R$ with $(x,0)$.

\pause
\medskip

\emph{Imaginary unit:}
\[
i \overset{\text{def}}{=} (0,1)
\]

\pause
\medskip

$
i^2 = -1
$
\qquad so
$z^2+1=0$ has the two solutions $i$ and $-i$.
\end{frame}

\begin{frame}

$(x,y) = x+iy$, so use $x+iy$ from now on (\emph{cartesian form}).

\medskip
\pause

The evil twin of $z=x+iy$ is the \emph{complex conjugate}
\[
\bar{z} \overset{\text{def}}{=} x-iy.
\]

\medskip
\pause

Let's give names to the $x$ and $y$:

\[
\Re z = 
\Re (x+iy)
\overset{\text{def}}{=}
\frac{z+\bar{z}}{2}
= x
\qquad \text{\emph{real part} of } z .
\]

\pause

\[
\Im z = 
\Im (x+iy)
\overset{\text{def}}{=}
\frac{z-\bar{z}}{2i}
= y
\qquad \text{\emph{imaginary part} of } z .
\]

\pause

An expression in $x,y$ can be written in terms of $z,\bar{z}$
and vice versa:

\[
x^3 + y^3 + 3ixy
=
{\left( \frac{z+ \bar{z}}{2} \right)}^3 + 
{\left( \frac{z- \bar{z}}{2i} \right)}^3 + 
3i {\left( \frac{z+ \bar{z}}{2} \right)} 
{\left( \frac{z- \bar{z}}{2i} \right)} ,
\]
or
\[
z^2 - i \bar{z}^2 + z \bar{z}
=
{(x+iy)}^2 - i {(x-iy)}^2 + 
(x+iy)(x-iy) .
\]
\pause

Almost looks as if $z$ and $\bar{z}$ were independent variables.
\end{frame}

\begin{frame}
We measure distance with the \emph{modulus} (euclidean distance from zero):
\[
\sabs{z} \overset{\text{def}}{=} \sqrt{z \bar{z}} = \sqrt{x^2+y^2} .
\]
\pause
The (euclidean) distance 
\(
\sabs{z-w}
\)
makes $\C$ into a complete metric space.

\medskip
\pause

%${\sabs{z}}^2 = z\bar{z}$.
%
%\medskip
%\pause
%
%$\sabs{z} \geq 0$, and $\sabs{z} = 0$ if and only if $z=0$.
%
%\end{frame}
%
%\begin{frame}
\begin{proposition}[Cauchy--Schwarz and the triangle inequality]
If $z,w \in \C$, then
\begin{enumerate}[(i)]
\item
$\sabs{\Re z\bar{w}} \leq \sabs{z} \sabs{w}$ \quad (Cauchy--Schwarz inequality, note: $\Re z
\bar{w}$ is the real dot product),\index{Cauchy--Schwarz}
\item
$\sabs{z+w} \leq \sabs{z} + \sabs{w}$ \quad (Triangle inequality).%
\index{triangle inequality!complex numbers}
\end{enumerate}
\end{proposition}

\textbf{Proof:}
Cachy--Schwarz:

\medskip

\(
\qquad
0  \leq {\sabs{z\bar{w}-\bar{z}w}}^2 \pause
   =    (z\bar{w}-\bar{z}w)(\bar{z}w-z\bar{w})  \pause
   =    2z\bar{z}w\bar{w} - z^2\bar{w}^2 - \bar{z}^2w^2 \pause
\)

\( \qquad \qquad
   =    4z\bar{z}w\bar{w} - {(z\bar{w}+\bar{z}w)}^2 \pause
   =    {\bigl(2\sabs{z}\sabs{w}\bigr)}^2 - {\bigl(2 \Re z\bar{w}\bigr)}^2 .
\)

\medskip
\pause

Triangle inequality:

\medskip
\(
\qquad
{\sabs{z+w}}^2  =    (z+w)(\bar{z}+\bar{w}) \pause
                =    z\bar{z} + w\bar{w} + z\bar{w} + \bar{z}w \pause
                \leq z\bar{z} + w\bar{w} + 2 \sabs{z}\sabs{w} \pause
                =    {\bigl(\sabs{z}+\sabs{w}\bigr)}^2 .
\) \qed

\end{frame}

\begin{frame}

\begin{proposition}
Complex addition, multiplication, division, and conjugation are continuous:
Suppose $\{ a_n \}$ and $\{ b_n \}$ are two convergent sequences
of complex numbers.  Then,
\begin{enumerate}[(i)]
\item
$\lim\limits_{n\to\infty} (a_n + b_n) = 
\left(\lim\limits_{n\to\infty} a_n \right) +
\left(\lim\limits_{n\to\infty} b_n \right)$,
\item
$\lim\limits_{n\to\infty} a_n b_n = 
\left(\lim\limits_{n\to\infty} a_n \right)
\left(\lim\limits_{n\to\infty} b_n \right)$,
\item
$\lim\limits_{n\to\infty} \frac{1}{a_n} = \frac{1}{\lim\limits_{n\to\infty} a_n}$,
as long as $\lim\limits_{n\to\infty} a_n \not= 0$,
\item
$\lim\limits_{n\to\infty} \bar{a}_n = 
\overline{\lim\limits_{n\to\infty} a_n}$.
\end{enumerate}
\end{proposition}

\medskip
\pause

All these operations are defined in terms of operations on the real an
imaginary parts which are continuous.  Details left as exercise.

\end{frame}

\begin{frame}

If $p \in \C$ and $r > 0$, define the \emph{disc} of radius $r$ around $p$
as
\begin{equation*}
\Delta_r(p)
\overset{\text{def}}{=}
\bigl\{ z \in \C : \sabs{z-p} < r \bigr\} .
\end{equation*}
\pause
\begin{equation*}
\D
\overset{\text{def}}{=}
\Delta_1(0)
=
\bigl\{ z \in \C : \sabs{z} < 1 \bigr\} 
\qquad \text{\emph{unit disc}}.
\end{equation*}
\pause
A useful ``version'' of $\D$ is the \emph{\myindex{upper half-plane}}:
\begin{equation*}
\bH 
\overset{\text{def}}{=}
\bigl\{
z \in \C : \Im z > 0
\bigr\} .
\end{equation*}

\medskip
\pause

\begin{definition}
An open and connected set $U \subset \C$ is called a
\emph{\myindex{domain}}.
\end{definition}

\end{frame}

\begin{frame}
If $f \colon X \to \C$, write $u = \Re f$ and $v = \Im f$:
\qquad
\(\displaystyle
f = u+iv .
\)
\pause
\medskip

If $X \subset \C$ ($z = x+iy$):
\qquad
\(
\displaystyle
\frac{\partial f}{\partial x} = 
\frac{\partial u}{\partial x} + i
\frac{\partial v}{\partial x}
\qquad\text{and}\qquad
\frac{\partial f}{\partial y} = 
\frac{\partial u}{\partial y} + i
\frac{\partial v}{\partial y} .
\)
\medskip
\pause

If $X \subset \R$:
\qquad
\(
f' = u' + iv'.
\)
\medskip
\pause

If $f \colon [a,b] \to \C$, $f$
is (Riemann) integrable if $u$ and $v$ are, and
\begin{equation*}
\int_a^b f(t) \, dt = 
\int_a^b u(t) \, dt + i \int_a^b v(t) \, dt .
\end{equation*}
\pause

\begin{proposition}
Suppose $f \colon [a,b] \to \C$ is (Riemann) integrable.  Then $\sabs{f}$ is
(Riemann) integrable and
\begin{equation*}
\bbabs{\int_a^b f(t) \, dt} \leq 
\int_a^b \abs{f(t)} \, dt .
\end{equation*}
\end{proposition}

\end{frame}

\begin{frame}
$\C$ is a 2-dimensional real vector space.

\pause
\medskip

Multiplication $z \mapsto \xi z$ is a real-linear operator
\(
\left[
\begin{smallmatrix}
a & -b \\
b & a
\end{smallmatrix}
\right]
\) if $\xi = a+ib$.

\pause
\[
\begin{bmatrix}
a & -b \\
b & a
\end{bmatrix}
\begin{bmatrix}
c \\
d 
\end{bmatrix}
=
\begin{bmatrix}
ac-bd \\
bc+ad
\end{bmatrix}
\quad
\text{just like}
\quad
(a+ib)(c+id) = (ac-bd) + i(bc+ad) .
\]
\pause
\medskip
$
1 ~~ ``{}={}'' ~~
\left[
\begin{smallmatrix}
1 & 0 \\
0 & 1
\end{smallmatrix}
\right]
\quad \text{and} \quad
i ~~ ``{}={}'' ~~ \left[
\begin{smallmatrix}
0 & -1 \\
1 & 0
\end{smallmatrix} 
\right]
\quad
$
are identity and rotation counterclockwise by $90^{\circ}$.

\pause
\medskip

$\C$ can be identified as a subring of $M_2(\R)$: Multiplying the matrices
is multiplying the numbers (exercise).

\medskip
\pause

Complex conjugation is real-linear,
$\left[ \begin{smallmatrix} 1 & 0 \\ 0 &
-1 \end{smallmatrix} \right]$,
but not a multiplication by a complex number.

\pause
\medskip

Multiplication by complex numbers and complex conjugation ``generate''
the ring $M_2(\R)$:  All real-linear operator are of the form
$z \mapsto \xi z + \zeta \bar{z}$ (exercise).

\pause
\medskip

Easy exercise:
If $M$ is $z \mapsto \xi z$, then $\det(M) = \abs{\xi}^2$.
If $M$ is $z \mapsto \xi z+ \zeta \bar{z}$, then $\det(M) =
\abs{\xi}^2-\abs{\zeta}^2$.

\end{frame}

\begin{frame}

How will we use this representation of $\C$?

\medskip
\pause

A differentiable
$f \colon \C \to \C$ is really
$f \colon \R^2 \to \R^2$.

\medskip
\pause

The derivative of $f$ is a linear operator from $\R^2$ to $\R^2$.

\medskip
\pause

Complex analysis is the study of functions (holomorphic) whose
derivative is multiplication by a complex number.

\medskip
\pause

The derivative is a matrix of partial derivatives, to be of the form
$\left[
\begin{smallmatrix}
a & -b \\
b & a
\end{smallmatrix}
\right]$ is a system of partial differential equations (PDE):
the Cauchy--Riemann equations.
\end{frame}

\end{document}
