\documentclass[10pt,aspectratio=169]{beamer}

% All the boilerplate is in ccaslides.sty
% Note that this also pulls in a custom vogtwidebar.sty
\usepackage{ccaslides}

\author{Ji\v{r}\'i Lebl}

\institute[OSU]{%
Departemento pri Matematiko de Oklahoma {\^S}tata Universitato}

\title{Cultivating Complex Analysis:\\%
Convergence of sequences of holomorphic functions (3.4.2)}

\date{}

\begin{document}

\begin{frame}
\titlepage
\end{frame}

\begin{frame}
What is the right topology for the set of holomorphic functions?

\medskip
\pause

Curiously, the right topology is the same as for continuous functions:

\begin{definition}
A sequence of functions $f_n \colon U \to \C$ converges
\emph{uniformly on compact subsets}
to $f \colon U \to \C$ if
$f_n|_K$ converges uniformly to $f|_K$
for every compact $K \subset U$.
\end{definition}

\medskip
\pause

What do we mean by the right topology?

The most natural one that preserves holomorphic functions.

\medskip
\pause

For example, the uniform topology is the right one for continuous functions:

The uniform limit of continuous functions is continuous.

\medskip
\pause

But for real differentiable functions, it's not:

\medskip
\pause

$\sabs{x}^{1+1/n}$ is $C^1$ on $\R$ and converges uniformly on compact
subsets to $\sabs{x}$.
\end{frame}

\begin{frame}

\begin{theorem}
Suppose $U \subset \C$ is open and $f_n \colon U \to \C$ is a sequence
of holomorphic functions converging uniformly on compact subsets to
$f \colon U \to \C$.
\pause
Then $f$ is holomorphic.
\pause
Moreover, for all $\ell$, the $\ell$\textsuperscript{th} derivative
$f_n^{(\ell)}$ converges uniformly on compact subsets to $f^{(\ell)}$.
\end{theorem}

\pause

\textbf{Proof:}
Fix $p \in U$.
\pause
Consider $\overline{\Delta_r(p)} \subset U$.

\medskip
\pause

For any $z \in \Delta_r(p)$,
\[
f_n(z) = \frac{1}{2\pi i}
\int_{\partial \Delta_r(p)} \frac{f_n(\zeta)}{\zeta-z} \, d\zeta .
\]
\pause
$\partial \Delta_r(p)$ is compact.
Let $\delta > 0$ be the distance of $z$ to
$\partial \Delta_r(p)$.

\medskip
\pause

For $\zeta \in \partial \Delta_r(p)$,
\begin{equation*}
\abs{\frac{f_n(\zeta)}{\zeta-z}
-
\frac{f(\zeta)}{\zeta-z}
}
=
\frac{\sabs{f_n(\zeta)-f(\zeta)}}{\sabs{\zeta-z}}
\pause
\leq
\frac{1}{\delta}
\sabs{f_n(\zeta)-f(\zeta)} .
\end{equation*}
\pause
I.e.,
$\zeta \mapsto \frac{f_n(\zeta)}{\zeta-z}$ converges uniformly to 
$\zeta \mapsto \frac{f(\zeta)}{\zeta-z}$ on $\partial \Delta_r(p)$
~(as $f_n \to f$ uniformly on $\partial \Delta_r(p)$).

\medskip
\pause

We can take the limit $n \to \infty$ underneath the integral.
\end{frame}

\begin{frame}
We get
\qquad
$
\displaystyle
f(z) = \frac{1}{2\pi i}
\int_{\partial \Delta_r(p)} \frac{f(\zeta)}{\zeta-z} \, d\zeta
$
\qquad
for all $z \in \Delta_r(p)$.

\medskip
\pause

$f|_{\partial \Delta_r(p)}$ is continuous by uniform convergence.

\medskip
\pause

$f|_{\Delta_r(p)} = C[f|_{\partial \Delta_r(p)}]$ (the Cauchy transform),
which is holomorphic.

\medskip
\pause

So $f$ is holomorphic.

\medskip
\pause

We still need to prove the ``Moreover'' bit:

That $\bigl\{ f_n^{(\ell)} \bigr\}$
also
converges uniformly on compact subsets.
\end{frame}

\begin{frame}
Suppose $K \subset U$ is compact.

\medskip
\pause

If $U \not= \C$, the distance $d$ of
$K$ and $\partial U$ is positive
(if $U=\C$, take $d > 0$ arbitrary).

\medskip
\pause

Let
$\displaystyle
\quad
K' = \bigcup_{z \in K} \overline{\Delta_{d/2}(z)} .
$

\medskip
\pause

$K \subset K' \subset U$.

\vspace*{-0.5in}
\hspace*{2.5in}%
\scalebox{0.95}{
\subimport*{../figures/}{kkprime.pdf_t}
}
\pause

\vspace*{-0.6in}

$K'$ is bounded.

\pause

If $p \notin K'$, $\exists q \in K$ such that
$\sabs{p-q}$ is

the distance of $K$ (compact) to $p$.

\pause
$\sabs{p-q} > \nicefrac{d}{2}$  \quad ($p\not\in K'$).

\pause
Every point in $\Delta_{\sabs{p-q}-d/2}(p)$
is further than $\nicefrac{d}{2}$ from $K$ (so not in $K'$).  So $K'$ is closed.

\pause

$\Rightarrow$ $K'$ is compact.

\medskip
\pause

$\{ f_n \}$ converges uniformly on $K'$:
\pause
Given $\epsilon > 0$, $\exists N$ such that
$\sabs{f_n(z)-f(z)} < \epsilon$ $\forall$ $z \in K'$, $n \geq N$.

\medskip
\pause
For any $p \in K$, by Cauchy estimates
on $f_n-f$
in $\Delta_{d/2}(p)$ (note $\partial \Delta_{d/2}(p) \subset K'$):
\pause
\begin{equation*}
\abs{
\,
f_n^{(\ell)}(p)
-
f^{(\ell)}(p)
}
\pause
\leq
\frac{\ell! \, \snorm{f_n-f}_{\partial \Delta_{d/2}(p)}}{{(d/2)}^{\ell}}
\pause
\leq
\frac{\ell!2^\ell}{d^\ell}\epsilon .
\end{equation*}
\pause
$\Rightarrow$ $\bigl\{ f_n^{(\ell)} \bigr\}$ converges uniformly to $f^{(\ell)}$ on $K$.
\qed
\end{frame}

\begin{frame}

Point is: We can write all derivatives of $f$ as integrals of $f$.

\medskip
\pause

Integration is a far nicer operation than differentiation, and for
holomorphic functions,

we can differentiate by integrating.
\end{frame}

\end{document}
