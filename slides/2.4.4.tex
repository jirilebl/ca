\documentclass[10pt,aspectratio=169]{beamer}

% All the boilerplate is in ccaslides.sty
% Note that this also pulls in a custom vogtwidebar.sty
\usepackage{ccaslides}

\author{Ji\v{r}\'i Lebl}

\institute[OSU]{%
Departemento pri Matematiko de Oklahoma {\^S}tata Universitato}

\title{Cultivating Complex Analysis:\\%
The identity theorem (2.4.4)}

\date{}

\begin{document}

\begin{frame}
\titlepage
\end{frame}

\begin{frame}
\begin{theorem}[Identity]
Suppose $U \subset \C$ is a domain, 
and $f \colon U \to \C$ analytic.
\pause
If $Z_f = \bigl\{ z \in U : f(z) = 0 \bigr\}$
has a limit point in $U$, then $f$ is identically zero.
\pause
In other words, all points of $Z_f$ are isolated unless $f \equiv 0$.
\end{theorem}

\pause

\begin{definition}
The points in the set $Z_f$ are called the \emph{zeros} of $f$.
\end{definition}

\pause
\medskip

Common application:

\pause

\emph{If the function is zero on a nonempty open subset, then $f \equiv 0$.}

\end{frame}

\begin{frame}

\textbf{Proof:}
Suppose $f$ is not identically zero.

\medskip
\pause

$Z_f$ is closed (in $U$, of course) as $f$ is continuous.

\medskip
\pause

Must show that points of $Z_f$ are isolated.

\medskip
\pause

WLOG $0 \in U$, $0 \in Z_f$, $0$ not  in the interior of $Z_f$.

\pause

\[
f(z) = \sum_{n=0}^\infty c_n z^n .
\]
\pause
As $f(0) = 0$, $c_0 =0$.
\pause
\medskip

Let $k$ be the smallest $k$ such that $c_k \not=0$ (exists since $f$
is not identically zero nearby).

\pause

\[
f(z) = z^k \sum_{n=k}^\infty c_n z^{n-k} = z^k g(z) .
\]
$g(z)$ is a convergent power series and $g(0) = c_k \not=
0$.

\pause
\medskip

$g$ is continuous and so $g(z) \not= 0$ in a whole neighborhood of $0$.

\pause
\medskip

$z^k$ is only zero at $0$ $\Rightarrow$ $0$ is an isolated zero of $f$.

\end{frame}

\begin{frame}

So points of $Z_f$ are either interior points or isolated points.

\medskip
\pause

Let $Z_f'$ be the interior points (nonisolated points).

\medskip
\pause

$Z_f'$ is closed (in $U$) ($Z_f$ is closed
and points in $Z_f'$ cannot approach isolated points).

\medskip
\pause

$Z_f'$ is open and closed and $U$ is connected $\Rightarrow$ either $U=Z_f'$ or $Z_f' = \emptyset$.
\qed

\end{frame}

\begin{frame}
A useful idea from the proof:

\medskip
\pause

If $f(z)$ is a power series at $a$ and $f(a)=0$ (and $f\not\equiv 0$),
we can factor out some power of $z-a$:
\pause
\begin{equation*}
f(z) = {(z-a)}^k g(z) ,
\end{equation*}
where $g(z)$ is a power series at $a$ such that $g(a) \not= 0$.
\end{frame}


\end{document}
