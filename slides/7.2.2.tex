\documentclass[10pt,aspectratio=169]{beamer}

% All the boilerplate is in ccaslides.sty
% Note that this also pulls in a custom vogtwidebar.sty
\usepackage{ccaslides}

\author{Ji\v{r}\'i Lebl}

\institute[OSU]{%
Departemento pri Matematiko de Oklahoma {\^S}tata Universitato}

\title{Cultivating Complex Analysis:\\%
Harmonic functions\\%
Mean-value property (7.2.2)}

\date{}

\begin{document}

\begin{frame}
\titlepage
\end{frame}

\begin{frame}
$f \colon \R \to \R$ is harmonic if
\pause
\[
\nabla^2 f = 
\frac{\partial^2}{\partial x^2} f = f'' = 0 .
\]
\pause
So $f(x) = Ax+B$ (affine linear).

\medskip
\pause

For any interval $[a,b]$, $a < b$,
\[
f\Bigl(\frac{a+b}{2}\Bigr) = \frac{f(a)+f(b)}{2} . \tag{*}
\]

\medskip
\pause

``Mean-value property'':

\medskip

\textbf{Exercise:} A continuous $f \colon \R \to \R$ is harmonic (affine linear) \wiffif (*) holds
for all $[a,b]$.
\end{frame}

\begin{frame}
\begin{theorem}[Mean-value property]
Suppose $U \subset \C$ is open.
A continuous 
$f \colon U \to \R$
is harmonic if and only if 
for every $p \in U$ there is an $R_p > 0$ such that
$\Delta_{R_p}(p) \subset U$ and
\begin{equation*}
f(p) = \frac{1}{2\pi} \int_{-\pi}^{\pi} f(p+re^{i\theta})\, d\theta
\qquad \text{for all } r < R_p .
\end{equation*}
Moreover, if $f$ is harmonic we may choose any $R_p > 0$ such that
$\Delta_{R_p}(p) \subset U$.
\end{theorem}

\pause
\textbf{Proof:}

\thus) \quad \pause
Suppose $f$ is harmonic and $\overline{\Delta_r(p)} \subset U$. 
\pause
\quad
($R_p$ as large as wanted)

\pause
\medskip

Solve the Dirichlet problem in $\Delta_r(p)$ (at $p$)
\begin{equation*}
\frac{1}{2\pi} \int_{-\pi}^{\pi}f(p + r e^{it}) \, dt =
P\bigl[f|_{\partial \Delta_r(p)}\big](p) \pause = f(p) .
\end{equation*}

\end{frame}

\begin{frame}
$\Leftarrow$) \quad \pause
Suppose $f$ is continuous and satisfies the
mean-value property.

\medskip
\pause

Suppose $\overline{\Delta_s(q)} \subset U$.
\qquad
\pause
Let $h = P\bigl[f|_{\partial \Delta_s(q)}\big]$.

\medskip
\pause
Let $\varphi = f-h$.
\quad
\pause
$\varphi$ is continuous,
\pause
identically zero on $\partial \Delta_s(q)$,
\pause
and satisfies the mean-value property on same circles as $f$.

\medskip
\pause
Suppose (for contradiction) $\varphi$
attains a maximum at $p \in \Delta_s(q)$ and
$\varphi(p) > 0$.

\medskip
\pause
$X = \bigl\{ z \in \Delta_s(q) : \varphi(z) = \varphi(p) \bigr\}$ is compact.

\medskip
\pause
Assume $p$ is the point of $X$ closest to $\partial \Delta_s(q)$.

\medskip
\pause
Suppose $\partial \Delta_r(p) \subset \Delta_s(q)$ (and $r < R_p$),

\vspace*{-0.8in}
\hfill
\subimport*{../figures/}{meanvalue.pdf_t}

\vspace*{-0.5in}

\pause
$\varphi \leq C < \varphi(p)$ (for some $C \in \R$) on an open subset of $\partial \Delta_r(p)$

\medskip
\pause
So
\quad
$\displaystyle
\frac{1}{2\pi} \int_{-\pi}^{\pi} \varphi(p+re^{i\theta})\, d\theta <
\varphi(p)$.
\pause
\quad
A contradiction!
\pause
\quad
So $\varphi \leq 0$.

\medskip
\pause

Similarly $\varphi \geq 0$.

\medskip
\pause

So $f=h$ and $f$ is harmonic. \qed
\end{frame}

\begin{frame}
For $C^2$ functions, just because $f_n \to f$ uniformly,
doesn't mean that $\nabla^2 f_n$ goes to $\nabla^2 f$. 

\medskip
\pause

But it does for harmonic functions.

\pause
\begin{theorem}[Harnack's first]
Let $U \subset \C$ be open, and let $f_n \colon U \to \R$ be a sequence of
harmonic functions converging uniformly on compact subsets to $f \colon U
\to \R$.  Then $f$ is harmonic.
\end{theorem}

\pause
\textbf{Proof:}
Firstly, $f$ is continuous.

\medskip
\pause

If $\overline{\Delta_r(p)} \subset U$,
then $\{ f_n \}$ converges uniformly on $\partial \Delta_r(p)$.  So
\pause
\begin{equation*}
f(p) =
\lim_{n\to\infty} f_n(p)
\pause
=
\lim_{n\to\infty} 
\frac{1}{2\pi} \int_{-\pi}^{\pi} f_n(p+re^{i\theta})\, d\theta
\pause
=
\frac{1}{2\pi} \int_{-\pi}^{\pi} f(p+re^{i\theta})\, d\theta .
\end{equation*}
\pause
Done by mean-value property. \qed
\end{frame}

\begin{frame}
\textbf{Exercise:}
Prove the maximum principle for harmonic functions directly from the
mean-value property.

\medskip
\pause

\textbf{Exercise:}
Let $U \subset \C$ be open.
Prove that a continuous $f \colon U \to \R$
is harmonic if and only if it satisfies the
\emph{disc mean-value property} for every $\overline{\Delta_r(p)}
\subset U$:
\[
f(p) = 
\frac{1}{\pi r^2} \int_{\overline{\Delta_r(p)}} \, f(z) \, dA.
\]

\end{frame}

\end{document}
