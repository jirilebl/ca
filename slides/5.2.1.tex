\documentclass[10pt,aspectratio=169]{beamer}

% All the boilerplate is in ccaslides.sty
% Note that this also pulls in a custom vogtwidebar.sty
\usepackage{ccaslides}

\author{Ji\v{r}\'i Lebl}

\institute[OSU]{%
Departemento pri Matematiko de Oklahoma {\^S}tata Universitato}

\title{Cultivating Complex Analysis:\\%
Types of singularities and Riemann extension (5.2.1)}

\date{}

\begin{document}

\begin{frame}
\titlepage
\end{frame}

\begin{frame}
\begin{definition}
Suppose $U \subset \C$ is open and $p \in U$.

\medskip
\pause
A holomorphic $f \colon U \setminus \{ p \} \to \C$ has
an \emph{isolated singularity} at $p$.

\medskip
\pause
An isolated singularity is \emph{removable}
if there exists a holomorphic $F \colon U \to \C$
such that $f(z) = F(z)$ for all $z \in U \setminus \{ p \}$.

\medskip
\pause
An isolated singularity $p$ is a \emph{pole} if 
$\displaystyle
\lim_{z \to p} f(z) = \infty$.

\medskip
\pause
An isolated singularity that is neither removable nor a pole is 
an \emph{essential singularity}.
\end{definition}

\pause

\textbf{Examples:} Pole: $\nicefrac{1}{z}$, essential: $e^{1/z}$.

\medskip
\pause

Interestingly, a holomorphic $f$ must blow up near a singularity for it to
not be removable.

\end{frame}

\begin{frame}

\begin{theorem}[Riemann extension theorem]
Suppose $U \subset \C$ is open, $p \in U$,
and $f \colon U \setminus \{p\} \to \C$ is holomorphic.

\pause
If $f$ is bounded (near $p$ suffices), then $p$ is a removable singularity.
\end{theorem}

\pause

\textbf{Proof:}
Let $g(z) = {(z-p)}^2 f(z)$ for $z \not= p$ and $g(p) = 0$.

\medskip
\pause

$g$ is holomorphic in $U \setminus \{ p \}$.

\medskip
\pause

Supposing $f$ is bounded,
\[
\lim_{z\to p} \frac{g(z)-g(p)}{z-p} = \lim_{z\to p} (z-p) f(z) \pause = 0 .
\]
\pause
So $g$ is complex differentiable at $p$ and so holomorphic on $U$.

\medskip
\pause

As $g(p)=0$ and $g'(p)=0$, $g$ has a zero of order $k \geq 2$.

\medskip
\pause

Write $g(z) = {(z-p)}^k h(z)$, where $h$ is holomorphic on $U$.

\medskip
\pause

Then $f(z) = {(z-p)}^{k-2}h(z)$, that is, $p$ is a removable singularity.
\qed

\end{frame}

\begin{frame}

\textbf{Exercise:}
Prove that if $f \colon \D \setminus \{ 0 \} \to \D \setminus \{ 0 \}$
is an automorphism, then $f(z) = e^{i\theta} z$ for some $\theta$.

\medskip
\pause

The Riemann extension theorem is (of course) not true for functions that are
not holomorphic.

\medskip
\pause

\textbf{Exercise:}
Prove that $\frac{xy}{x^2+y^2}$ is a bounded infinitely 
(real) differentiable function
on $\R^2 \setminus \{ (0,0) \}$ with an isolated singularity, and this
function does not
extend through the singularity even continuously.

\end{frame}

\begin{frame}
Poles are places where the function ``blows up'' to a finite order:

\pause

\begin{corollary}
Suppose $U \subset \C$ is open, $p \in U$,
and $f \colon U \setminus \{p\} \to \C$ holomorphic.
\pause
\begin{enumerate}[(i)]
\item
If $p$ is a pole, then there exists a $k \in \N$ such that
$g(z) = {(z-p)}^k f(z)$
has a removable singularity at $p$.
\pause
\item
Conversely, if there exists a $k \in \N$ such that
$g(z) = {(z-p)}^k f(z)$ is bounded near $p$ (has a removable singularity),
then $f$ has a pole or a
removable singularity at $p$.
\end{enumerate}
\end{corollary}

\pause
\textbf{Proof:}
Suppose $f$ has a pole at $p$.

\medskip
\pause

Then $f$ is not zero in some $\Delta_r(p) \setminus \{p\}$ (it goes to
infinity).

\medskip
\pause

$\nicefrac{1}{f}$ is bounded in some $\Delta_r(p) \setminus \{p\}$
(as $f \to \infty$).

\medskip
\pause

Thus, by Riemann extension, there exists an $h$ holomorphic in $\Delta_r(p)$
such that $h(z) = \nicefrac{1}{f(z)}$ for $z \not= p$.

\medskip
\pause

As $\frac{1}{f} \to 0$ as $z \to p$, we have $h(p) = 0$.

\medskip
\pause

$\Rightarrow$ ~ $h(z) = {(z-p)}^k \psi(z)$ for some holomorphic $\psi$ and $k \in \N$,
where $\psi(p) \not= 0$.

\end{frame}

\begin{frame}
\hfill (keeping track: \quad $h(z) = \nicefrac{1}{f(z)}$ \quad
$h(z) = {(z-p)}^k \psi(z)$ \quad $g(z) = {(z-p)}^k f(z)$)

\medskip
\pause

As $h\not=0$ in $\Delta_r(p) \setminus \{p\}$,
then $\psi \not= 0$ in $\Delta_r(p)$
\quad $\Rightarrow$ \quad
$\nicefrac{1}{\psi}$ is holomorphic in $\Delta_r(p)$.

\medskip
\pause

For $z \in \Delta_r(p) \setminus \{p\}$,
\[
g(z) =
{(z-p)}^k f(z)
\pause
=
{(z-p)}^k \frac{1}{{(z-p)}^k \psi(z)}
\pause
=
\frac{1}{\psi(z)} .
\]
and (i) is done.

\medskip
\pause

For the converse statement suppose
$g(z)$ has a removable singularity.

\medskip
\pause

Then $g(z) = {(z-p)}^\ell \varphi(z)$ where
$\varphi(p) \not= 0$.

\medskip
\pause
Then
\[
f(z) = {(z-p)}^{\ell-k} \varphi(z) ,
\]
\pause
$f(z)$ either goes to $\infty$ if $k > \ell$ ($f$ has a pole) or
\pause

is bounded near $p$ if $k \leq \ell$ ($f$ has a removable singularity).
\qed
\end{frame}

\begin{frame}

\begin{definition}
Given a holomorphic function $f$ with a pole at $p$, the smallest $k \in \N$
such that ${(z-p)}^k f(z)$ is bounded near $p$ is called the
\emph{order} of the pole.
\pause
A pole of order $1$ is called a \emph{simple pole}.
\end{definition}
\pause

So if $f$
has a pole at $p$, then
\begin{equation*}
f(z) = \frac{g(z)}{{(z-p)}^k}
\end{equation*}
for holomorphic $g$ nonzero at $p$, and $k$ is the order of the pole.

\medskip
\pause

Symmetry between zeros and poles:
\medskip
\pause

If $f$ has a zero of order $k$ at $p$, then $\nicefrac{1}{f}$ has a pole of
order $k$ at $p$.

\medskip
\pause

If $f$ has a pole of order $k$ at $p$, then $\nicefrac{1}{f}$
has a removable singularity, and the extended function has a zero of order
$k$ at $p$.

\medskip
\pause

More precisely, if $f$ has a pole or a removable singularity we can write
$f(z) = {(z-p)}^\ell g(z)$ for some $\ell \in \Z$ and some holomorphic $g$ such
that $g(p) \not= 0$.

\medskip
\pause

$p$ is a zero of order $\ell$ if $\ell > 0$, and $p$ is a pole of order $-\ell$ if $\ell < 0$.

\end{frame}

\begin{frame}

A useful exercise that parallels a result we proved for zeros:

\medskip
\pause

\textbf{Exercise:}
Suppose $f$ has a pole of order $k \in \N$ at $p$.
Show that there exists a holomorphic $g$ defined near $p$
such that $g(p) = 0$ and $g'(p) \not= 0$ and such that near $p$
\begin{equation*}
f(z) = \frac{1}{{\bigl(g(z)\bigr)}^k} .
\end{equation*}

\end{frame}

\end{document}
