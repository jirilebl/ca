\documentclass[10pt,aspectratio=169]{beamer}

% All the boilerplate is in ccaslides.sty
% Note that this also pulls in a custom vogtwidebar.sty
\usepackage{ccaslides}

\author{Ji\v{r}\'i Lebl}

\institute[OSU]{%
Departemento pri Matematiko de Oklahoma {\^S}tata Universitato}

\title{Cultivating Complex Analysis:\\%
Wirtinger operators (2.2.2)}

\date{}

\begin{document}

\begin{frame}
\titlepage
\end{frame}

\begin{frame}
Suppose $z=x+iy$.

\medskip
\pause

Define
the \emph{Wirtinger operators}:
\[
\frac{\partial}{\partial z}
\overset{\text{def}}{=}
\frac{1}{2}
\left(
\frac{\partial}{\partial x} - i
\frac{\partial}{\partial y}
\right),
\qquad
\frac{\partial}{\partial \bar{z}}
\overset{\text{def}}{=}
\frac{1}{2}
\left(
\frac{\partial}{\partial x} + i
\frac{\partial}{\partial y}
\right) .
\]

\pause

Not really ``partial derivatives'' despite the notation.

\medskip
\pause

The operators are really determined by wanting
\[
\frac{\partial}{\partial z} z = 1, \quad
\frac{\partial}{\partial z} \bar{z} = 0, \quad
\frac{\partial}{\partial \bar{z}} z = 0, \quad
\frac{\partial}{\partial \bar{z}} \bar{z} = 1.
\]
\end{frame}

\begin{frame}
Let $f = u+iv$.
\[
\frac{\partial f}{\partial \bar{z}} 
=
\frac{1}{2}
\left(
\frac{\partial f}{\partial x} + i
\frac{\partial f}{\partial y}
\right)
\pause
=
\frac{1}{2}
\left(
\frac{\partial u}{\partial x} 
+ i \frac{\partial v}{\partial x} 
+ i \frac{\partial u}{\partial y}
- \frac{\partial v}{\partial y}
\right) 
\pause
=
\frac{1}{2}
\left(
\frac{\partial u}{\partial x} 
- \frac{\partial v}{\partial y}
\right)
+
\frac{i}{2}
\left(
\frac{\partial v}{\partial x} 
+ \frac{\partial u}{\partial y}
\right)
\]
\pause
This expression is zero if and only if the real and imaginary
parts are zero. \pause  Namely,
\[
\frac{\partial u}{\partial x} 
- \frac{\partial v}{\partial y}
= 0,
\qquad
\text{and}
\qquad
\frac{\partial v}{\partial x} 
+ \frac{\partial u}{\partial y} = 0
.
\]
\pause
Those are the Cauchy--Riemann equations!

\pause

\begin{proposition}
Let $U \subset \C$ be open.  Then $f \colon U \to \C$ is
holomorphic if and only if
$f$ is (real) differentiable and
\begin{equation*}
\frac{\partial f}{\partial \bar{z}} \equiv 0 .
\end{equation*}
\end{proposition}
\end{frame}

\begin{frame}
On the other hand, if $f$ is holomorphic,
\[
\frac{\partial f}{\partial z} 
\pause
=
\frac{1}{2}
\left(
\frac{\partial u}{\partial x} 
+ \frac{\partial v}{\partial y}
\right)
+
\frac{i}{2}
\left( \frac{\partial v}{\partial x} - \frac{\partial u}{\partial y}
\right) 
\pause
=
\frac{\partial u}{\partial x} 
+ i \frac{\partial v}{\partial x}
\pause
 =
\frac{\partial f}{\partial x}
\pause
\qquad
=
\frac{1}{i} \left(
\frac{\partial u}{\partial y}
+ i
\frac{\partial v}{\partial y} 
\right)
\pause
 =
\frac{1}{i}
\frac{\partial f}{\partial y}
.
\]
\pause
(In the second form think of derivative in $iy$ not $y$).

\medskip
\pause

To compute $f'(z)$ we can approach from any direction:
\pause
\[
f'(z) =
\lim_{\substack{h \to 0\\h\in\C}}
\frac{f(z+h)-f(z)}{h}
\pause
=
\lim_{\substack{t \to 0\\t\in\R}}
\frac{f(z+t)-f(z)}{t}
\pause
=
\frac{\partial u}{\partial x} \Big|_z
+ i \frac{\partial v}{\partial x}\Big|_z
\pause
 =
\frac{\partial f}{\partial x} \Big|_z ,
\]
\pause
and
\[
f'(z) =
\lim_{\substack{h \to 0\\h\in\C}}
\frac{f(z+h)-f(z)}{h}
\pause
=
\lim_{\substack{t \to 0\\t\in\R}}
\frac{f(z+it)-f(z)}{it}
\pause
=
\frac{1}{i}
\left(
\frac{\partial u}{\partial y}  \Big|_z
+ i \frac{\partial v}{\partial y} \Big|_z
\right)
\pause
 =
\frac{1}{i}
\frac{\partial f}{\partial y} \Big|_z .
\]
\pause
So for a holomorphic function
\begin{equation*}
f' =
\frac{\partial f}{\partial z} .
\end{equation*}
\end{frame}

\begin{frame}
The complex derivative $f'$, sometimes written as $\frac{df}{dz}$,
only exists for holomorphic functions.

\medskip
\pause

The Wirtinger operators
$\frac{\partial f}{\partial z}$ and
$\frac{\partial f}{\partial \bar{z}}$ make sense for every real differentiable
function.

\medskip
\pause

For polynomials the operators work as if $z$ and $\bar{z}$ were separate
variables.  E.g. (exercise)
\[
\frac{\partial}{\partial z}
\left[ z^2 \bar{z}^3 + z^{10} \right]
=
2z \bar{z}^3 + 10 z^{9}
\qquad
\text{and}
\qquad
\frac{\partial}{\partial \bar{z}}
\left[ z^2 \bar{z}^3 + z^{10} \right]
=
z^2 ( 3 \bar{z}^2 ) + 0 .
\]
\pause
So a holomorphic function is ``one that does not depend on $\bar{z}$.''

\medskip
\pause
\textbf{Caution:}
Note that
$\frac{d}{dz} \left[ z^2 \bar{z}^3 + z^{10} \right]$ does not exist, while
$\frac{\partial}{\partial z} \left[ z^2 \bar{z}^3 + z^{10} \right]$ does.
\end{frame}

\begin{frame}
A few other remarks that are left as exercises.

\medskip
\pause
\textbf{Remark 1:}
If we write the real derivative $Df|_p$ as the map $h \mapsto \zeta h + \xi \bar{h}$,
\pause
then
\[
\frac{\partial f}{\partial z}\big|_p = \zeta
\qquad \text{and} \qquad
\frac{\partial f}{\partial \bar{z}}\big|_p = \xi .
\]

\medskip
\pause
\textbf{Remark 2:}
If $\bar{f}$ is the complex conjugate of $f$, then
\pause
\[
\overline{\left(\frac{\partial f}{\partial z}\right)} = 
\frac{\partial \bar{f}}{\partial \bar{z}}
\qquad \text{and} \qquad
\overline{\left(\frac{\partial f}{\partial \bar{z}}\right)} = 
\frac{\partial \bar{f}}{\partial z} .
\]

\medskip
\pause
\textbf{Remark 3:} The chain rule for real differentiable functions
can be written as
\[
\frac{\partial (g \circ f)}{\partial z}\Big|_p
=
\frac{\partial g}{\partial z}\Big|_{f(p)}
\frac{\partial f}{\partial z}\Big|_p
+
\frac{\partial g}{\partial \bar{z}}\Big|_{f(p)}
\frac{\partial \bar{f}}{\partial z}\Big|_p
\qquad
\text{and}
\qquad
\frac{\partial (g \circ f)}{\partial \bar{z}}\Big|_p
=
\frac{\partial g}{\partial z}\Big|_{f(p)}
\frac{\partial f}{\partial \bar{z}}\Big|_p
+
\frac{\partial g}{\partial \bar{z}}\Big|_{f(p)}
\frac{\partial \bar{f}}{\partial \bar{z}}\Big|_p .
\]
\end{frame}

\end{document}
