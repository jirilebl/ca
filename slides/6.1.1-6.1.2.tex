\documentclass[10pt,aspectratio=169]{beamer}

% All the boilerplate is in ccaslides.sty
% Note that this also pulls in a custom vogtwidebar.sty
\usepackage{ccaslides}

\author{Ji\v{r}\'i Lebl}

\institute[OSU]{%
Departemento pri Matematiko de Oklahoma {\^S}tata Universitato}

\title{Cultivating Complex Analysis:\\%
Convergence of subsequences (6.1.1)\\%
Equicontinuity (6.1.2)}

\date{}

\begin{document}

\begin{frame}
\titlepage
\end{frame}

\begin{frame}
\textbf{Goal:} Classify relatively compact subsets of the space of holomorphic
functions.

\medskip
\pause

We want something like Bolzano--Weierstrass:

\emph{If $\{ z_n \}$ is a
bounded sequence, then it has a convergent subsequence.}

\bigskip
\pause

\textbf{Examples:}

\medskip

$\sin(nx)$, \enspace $x \in \R$. \pause
On no interval $[a,b] \subset \R$ does there exist
a subsequence converging pointwise.  Not even almost everywhere.
\pause
(Proof requires some measure theory)

\medskip
\pause

$x^n$, \enspace $x \in [0,1]$. \pause
Converges (pointwise) to a discontinuous function.

\end{frame}

\begin{frame}
We start with boundedness.

\pause

\begin{definition}
A sequence $f_n \colon X \to \C$
is \emph{\myindex{pointwise bounded}} if for every $x \in X$, there is an $M_x \in \R$
such that
\begin{equation*}
\sabs{f_n(x)} \leq M_x \qquad \text{for all } n \in \N .
\end{equation*}
\pause
$\{ f_n \}$
is \emph{\myindex{uniformly bounded}} if there is an $M \in \R$
such that
\begin{equation*}
\sabs{f_n(x)} \leq M \qquad \text{for all } n \in \N \text{ and all } x \in X.
\end{equation*}
\end{definition}

\medskip
\pause

\textbf{Example:}

$\dfrac{n^2x}{1+n^2x^2}$ is pointwise bounded (converges pointwise)
but not uniformly bounded.


\end{frame}

\begin{frame}
\begin{proposition}
Let $X$ be a countable set and $f_n \colon X \to \C$ a pointwise bounded
sequence of functions.  Then $\{ f_n \}$ has a subsequence that converges
pointwise.
\end{proposition}

\pause

\textbf{Proof:}
Let $\{ x_n \}_{n=1}^{\infty}$ give $X$.

\pause
\medskip

$\{ f_n(x_1) \}_{n=1}^\infty$ is bounded
\wthus
$\exists$
subsequence
$\{ f_{1,k} \}_{k=1}^\infty$
of $\{ f_n \}_{n=1}^{\infty}$
s.t.
$\{ f_{1,k}(x_1) \}_{k=1}^\infty$ converges.

%\pause
%Next $\{ f_{1,k}(x_2) \}_{k=1}^\infty$ is bounded $\Rightarrow$
%$\{ f_{1,k} \}_{k=1}^\infty$ has a subsequence
%$\{ f_{2,k} \}_{k=1}^\infty$ such that
%$\{ f_{2,k}(x_2) \}_{k=1}^\infty$ converges. ($\{ f_{2,k}(x_1)
%\}_{k=1}^\infty$ converges)

\medskip
\pause

Define subsequences $\{ f_{m,k} \}_{k=1}^\infty$ as follows:

\pause
\medskip

Given a subsequence
$\{ f_{m,k} \}_{k=1}^\infty$ of
$\{ f_{m-1,k} \}_{k=1}^\infty$
that makes $\{ f_{m,k}(x_j) \}_{k=1}^\infty$ converge for all $j \leq m$,
\pause

Let $\{ f_{m+1,k} \}_{k=1}^\infty$ be a subsequence of $\{ f_{m,k} \}_{k=1}^\infty$
such that
$\{ f_{m+1,k}(x_{m+1}) \}_{k=1}^\infty$ converges

\pause
(and so 
$\{ f_{m+1,k}(x_{j}) \}_{k=1}^\infty$ converges for all
$j \leq m+1$).

\pause
\medskip

If $X$ is finite \wthus done.

\pause
\medskip

If $X$ is infinite,
pick the subsequence $\{ f_{k,k} \}_{k=1}^\infty$.

\pause
\medskip

For any $m$, the tail $\{ f_{k,k} \}_{k=m}^\infty$ is a subsequence of $\{ f_{m,k}
\}_{k=1}^\infty$
\wthus
$\{ f_{k,k}(x_m) \}_{k=1}^\infty$ converges.
\qed
\end{frame}

\begin{frame}
Continuity of the limit requires some uniformity.
\pause

\begin{definition}
Let $(X,d)$ be a metric space.
A set $S$ of functions
$f \colon X \to \C$ is
\emph{\myindex{equicontinuous}} at $x \in X$
if for every $\epsilon > 0$, there is a $\delta > 0$
such that if $y \in X$ with $d(x,y) < \delta$,
\begin{equation*}
\sabs{f(x)-f(y)} < \epsilon \qquad \text{for all } f \in S .
\end{equation*}
$S$ is \emph{equicontinuous} if it is equicontinuous at every $x \in X$.

\pause
\medskip

$S$ is 
\emph{\myindex{uniformly equicontinuous}}
if for every $\epsilon > 0$, there is a $\delta > 0$
such that if $x,y \in X$ with $d(x,y) < \delta$,
\begin{equation*}
\sabs{f(x)-f(y)} < \epsilon \qquad \text{for all } f \in S .
\end{equation*}
\end{definition}

\medskip
\pause

For finite sets $S$, same as continuity and uniform continuity.
\end{frame}

\begin{frame}

For compact sets, equicontinuity implies uniform equicontinuity (as one
would expect)
\pause

\begin{proposition}
Let $(X,d)$ be a compact metric space and
$f_n \colon X \to \C$ an equicontinuous sequence of functions.
Then the sequence $\{ f_n \}$ is uniformly equicontinuous.
\end{proposition}

\pause
\textbf{Proof:}
Suppose $\{ f_n \}$ is not
uniformly equicontinuous.
\pause

\thus~
$\exists$
$\epsilon > 0$
s.t. $\forall k \in \N$,
$\exists$
$n_k \in \N$ \&
$x_k,y_k \in X$ with $d(x_k,y_k) < \nicefrac{1}{k}$
where $\sabs{f_{n_k}(x_k)-f_{n_k}(y_k)} \geq \epsilon$.

\medskip
\pause

WLOG, by compactness and passing to a subseq., $\{x_k\}$ and $\{ y_k\}$ converge
to some $x \in X$.

\medskip
\pause

For any $\delta > 0$, take $k$ such that
$d(x,x_k) < \delta$ and
$d(x,y_k) < \delta$.
\pause Then
\begin{equation*}
\epsilon \leq 
\abs{f_{n_k}(x_k)-f_{n_k}(y_k)}
\leq
\abs{f_{n_k}(x_k)-f_{n_k}(x)} + \abs{f_{n_k}(x)-f_{n_k}(y_k)} .
\end{equation*}
\pause
\thus \quad
Either 
$\abs{f_{n_k}(x_k)-f_{n_k}(x)}$ or $\abs{f_{n_k}(x)-f_{n_k}(y_k)}$ is
$\geq \nicefrac{\epsilon}{2}$ (no matter how small $\delta$ is).

\medskip
\pause
\thus\quad $\{ f_n \}$ is not equicontinuous at $x$. \qed
\end{frame}

\begin{frame}
\textbf{Exercise:}
Suppose $(X,d)$ is a compact metric space,
and a sequence of continuous $f_n \colon X \to \C$
converges uniformly. Prove that $\{ f_n \}$ is uniformly equicontinuous.

\pause
\medskip

\textbf{Exercise:}
Suppose $S$ is a set of (real) differentiable functions $f \colon [0,1] \to \R$
such that $\sabs{f'(x)} \leq 1$ for all $x \in [0,1]$.
Prove that $S$ is uniformly equicontinuous.

\end{frame}

\end{document}
