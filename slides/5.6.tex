\documentclass[10pt,aspectratio=169]{beamer}

% All the boilerplate is in ccaslides.sty
% Note that this also pulls in a custom vogtwidebar.sty
\usepackage{ccaslides}

\author{Ji\v{r}\'i Lebl}

\institute[OSU]{%
Departemento pri Matematiko de Oklahoma {\^S}tata Universitato}

\title{Cultivating Complex Analysis:\\%
Inverses of holomorphic functions (5.6)}

\date{}

\begin{document}

\begin{frame}
\titlepage
\end{frame}

\begin{frame}
Let us restate the inverse function theorem.

\begin{theorem}[Inverse function theorem for holomorphic functions]
Suppose $U \subset \C$ is open, $f \colon U \to \C$ is holomorphic,
$p \in U$, and $f'(p) \not= 0$.  
Then there exist open sets $V, W \subset \C$ such that
$p \in V \subset U$, $f(V) = W$, the restriction $f|_V$ is injective
(one-to-one),
and hence a $g \colon W \to V$ exists such that
$g(w) = (f|_V)^{-1}(w)$ for all $w \in W$.
Furthermore, $g$ is holomorphic and
\begin{equation*}
g'(w) = \frac{1}{f'\bigl(g(w)\bigr)} \qquad \text{for all $w \in W$}.
\end{equation*}
\end{theorem}

\pause
\medskip

In other words, if $f'$ is nonzero somewhere, $f$ is injective near that
point.

\pause
\medskip

Only local: $f(z) = z^2$ maps $\C \setminus \{ 0 \}$
to itself, $f'$ does not vanish, but $f$ is $2$-to-$1$ globally.
\end{frame}

\begin{frame}
Real functions can be injective and the derivative can vanish:

$f \colon \R \to \R$, ~$f(x) = x^3$, ~is injective but $f'(0) = 0$.

\medskip
\pause

Holomorphic functions locally all behave like
$z \mapsto z^k$, and that is injective only if $k=1$.

\medskip
\pause

\begin{lemma}
If $U \subset \C$ is open and $f \colon U \to \C$ is holomorphic and injective, then
$f'$ is never zero.
\end{lemma}

\pause

\textbf{Proof:}
Suppose $f$ nonconstant and $f'(p) = 0$.

\pause
Let $\overline{\Delta_r(p)} \subset U$
be so that ~ $f' \not= 0$ on $\Delta_r(p) \setminus
\{ p \}$, ~
and $\sabs{f(z)-f(p)} > \delta > 0$ for $z \in \partial
\Delta_r(p)$.

\medskip
\pause

$z \mapsto f(z) - f(p)$ has a zero of multiplicity at least
two.

\pause
\medskip

Let $w \in \Delta_{\delta}\bigl(f(p)\bigr) \setminus \bigl\{ f(p) \bigr\}$
\pause
\wthus
$z \mapsto f(z)-w$ has at least two zeros (Rouch\'e).

\pause
\medskip

$f'\not=0$ in $\Delta_r(p) \setminus \{ p \}$
\pause
\wthus
zeros of $z \mapsto f(z)-w$ are simple

\pause
\thus \quad
$f(z)-w$ has at least two distinct zeros
\pause
\wthus
$f$ is not injective.  \qed
\end{frame}

\begin{frame}
We can actually compute the inverse:

\begin{lemma}
If $f \colon U \to \C$ is holomorphic and injective, and
$\overline{\Delta_r(p)} \subset U$.  Then for all $w \in
f\bigl(\Delta_r(p)\bigr)$,
\begin{equation*}
f^{-1}(w) = \frac{1}{2\pi i} \int_{\partial \Delta_r(p)}
\frac{f'(z)z}{f(z)-w} \, dz .
\end{equation*}
\end{lemma}

\pause

\textbf{Proof:}
Fix $w \in f\bigl(\Delta_r(p)\bigr)$ and $\zeta \in \Delta_r(p)$
such that $f(\zeta) = w$.

\medskip
\pause

$f'$ is never zero, so $z \mapsto f(z)-w$ has a simple
zero at $z=\zeta$.

\medskip
\pause

By the residue theorem
\begin{equation*}
\frac{1}{2\pi i} \int_{\partial \Delta_r(p)}
\frac{f'(z)z}{f(z)-w} \, dz
=
\operatorname{Res}\left(\frac{f'(z)z}{f(z)-w};\zeta\right)
\pause
=
\frac{f'(\zeta)\zeta}{f'(\zeta)} = \zeta = f^{-1}(w) .
\qed
\end{equation*}

\pause
\medskip

Consequently, $f^{-1}$ is holomorphic without even using the inverse function
theorem.
\end{frame}

\begin{frame}
\begin{theorem}
If $U \subset \C$ is open and $f \colon U \to \C$ is holomorphic and
injective, then $f(U)$ is open, $f'$ is never zero on $U$, and $f^{-1}
\colon f(U) \to U$ is holomorphic.
\end{theorem}

\medskip
\pause

\textbf{Proof:}
$f(U)$ is open by the open mapping theorem.

\medskip
\pause

By one of the lemmas, $f'$ is never zero on $U$.

\medskip
\pause

By the other (or IFT), $f^{-1}$ is holomorphic.
\qed
\end{frame}

\end{document}
