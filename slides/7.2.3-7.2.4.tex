\documentclass[10pt,aspectratio=169]{beamer}

% All the boilerplate is in ccaslides.sty
% Note that this also pulls in a custom vogtwidebar.sty
\usepackage{ccaslides}

\author{Ji\v{r}\'i Lebl}

\institute[OSU]{%
Departemento pri Matematiko de Oklahoma {\^S}tata Universitato}

\title{Cultivating Complex Analysis:\\%
Harmonic functions\\%
Harnack's inequality and principle (7.2.3--7.2.4)}

\date{}

\begin{document}

\begin{frame}
\titlepage
\end{frame}

\begin{frame}
\begin{theorem}[Harnack's inequality]
Suppose $f \colon \Delta_R(p) \to \R$ is harmonic and nonnegative,
and suppose $0 < r < R$.
Then for all $z \in \overline{\Delta_r(p)}$,
\begin{equation*}
\frac{R-r}{R+r} \, f(p) \leq f(z) \leq \frac{R+r}{R-r} \, f(p) .
\end{equation*}
\end{theorem}

\pause

\textbf{Proof:}
$\frac{R+r}{R-r}$ increasing in $r$ and
$\frac{R-r}{R+r}$ decreasing in $r$
\pause
\wthus
enough to prove for
$z \in \partial \Delta_r(p)$.
%$z=p+re^{i\theta}$.

\pause
Let $S$ be such that $0 < r < S < R$.
\pause
\[
f(p + re^{i\theta})
=
\int_{-\pi}^\pi f(p+Se^{it}) P_{r/S}(\theta-t) \, dt
\pause
=
\frac{1}{2\pi} 
\int_{-\pi}^\pi f(p+Se^{it})
\frac{S^2-r^2}{S^2+r^2-2Sr \cos (\theta-t)}
\, dt .
\]
\pause
\[
\frac{S-r}{S+r}
\pause
=
\frac{S^2-r^2}{S^2+r^2+2Sr}
\pause
\leq
\frac{S^2-r^2}{S^2+r^2-2Sr \cos (\theta-t)}
\pause
\leq
\frac{S^2-r^2}{S^2+r^2-2Sr}
\pause
=
\frac{S+r}{S-r} .
\]
\pause
For $z = p+re^{i\theta}$ (using $f \geq 0$),
\[
f(z)
=
\int_{-\pi}^\pi f(p+Se^{it}) P_{r/S}(\theta-t) \, dt
\pause
\leq
\frac{S+r}{S-r} 
\left(
\frac{1}{2\pi}
\int_{-\pi}^\pi f(p+Se^{it}) \, dt
\right)
\pause
\leq
\frac{S+r}{S-r} 
\,
f(p) .
\]
\pause
The lower inequality follows in the same way.

\pause
As $S<R$ was arbitrary, the theorem follows by taking a limit.
\qed
\end{frame}

\begin{frame}
The inequalities are optimal.
\pause
In $\D$, the theorem says
\begin{equation*}
\frac{1-r}{1+r} \, f(0) \leq f(z) \leq \frac{1+r}{1-r} \, f(0) .
\end{equation*}
\pause
Consider
\[f(z) = \Re \frac{1+z}{1-z}.\]
(the Poisson kernel except for
$\frac{1}{2\pi}$).

\pause
\medskip

So $f(z) > 0$ on $\D$.

\pause
\medskip

$f(0)=1$.

\medskip
\pause

$z=r$ gets equality on the right hand side.

\medskip
\pause

$z=-r$ gets equality on the left hand side.
\end{frame}


\begin{frame}
\begin{corollary}[Harnack's inequality]
Suppose $U \subset \C$ is a domain and $K \subset U$ is compact.
Then there exists a $C > 0$ such that
\begin{equation*}
\sup_{z \in K} f(z) \leq C \inf_{z\in K} f(z)
\end{equation*}
for every harmonic and nonnegative function $f$ defined on $U$.
\end{corollary}

\pause

\textbf{Proof:}
WLOG assume that $K$ is connected: \pause
(e.g.\ cover by finitely many closed discs so that $K$ has finitely many
components, then connect with paths as
$U$ is path connected).

\medskip
\pause

Let 
$r> 0$ be less than half the distance from $K$ to $\partial U$.

\pause
$\Delta_{r}(z_1),\ldots,\Delta_{r}(z_N)$ cover $K$
and
$\Delta_{2r}(z_j) \subset U$ for every $j$.

\pause
Fix $\zeta,\xi \in K$.
\pause
Assume
$\zeta \in \Delta_{r}\bigl(z_{1}\bigr)$

and
$\xi \in \Delta_{r}\bigl(z_{n}\bigr)$

(for some $n \leq N$.)
\pause

Also assume
$\Delta_{r}\bigl(z_{j}\bigr) \cap \Delta_{r}\bigl(z_{j+1}\bigr)
\not=\emptyset$

for all $j=1,\ldots,n-1$ ($K$ connected).
\pause

\vspace*{-1in}
\hfill
\subimport*{../figures/}{harnacksgeneral.pdf_t}

\end{frame}

\begin{frame}
Suppose $f \colon U \to \R$ is harmonic and nonnegative.

\medskip
\pause

$\Delta_{2r}(z_j) \subset U$ for all $j$ \pause \wthus if $w \in \Delta_r(z_j)$
\pause
\quad \thus
\[
\frac{1}{3} \, f(z_j) = 
\frac{2r-r}{2r+r}\, f(z_j)
\leq
f(w)
\leq \frac{2r+r}{2r-r}\, f(z_j)
= 3 f(z_j) .
\]
\pause
\thus\quad $f(w) \leq 3 f(z_j)$ and $f(z_j) \leq 3 f(w)$.

\pause
\medskip

Follow the discs.

\medskip
\pause

$\zeta \in \Delta_r(z_1)$ \wthus
$f(\zeta) \leq 3 f(z_1)$.

\pause
\medskip

Let 
$q$ be the midpoint between $z_{j}$ and $z_{j+1}$
\wthus
$q \in \Delta_{r}\bigl(z_{j}\bigr) \cap \Delta_{r}\bigl(z_{j+1}\bigr)$.

\pause
\medskip
\thus
\quad
$f(z_j) \leq 3 f(q) \leq 3 \bigl( 3 f(z_{j+1}) \bigr) = 3^2 f(z_{j+1})$.

\pause
\medskip

$\xi \in \Delta_r(z_n)$ \wthus
$f(z_n) \leq 3 f(\xi)$.

\pause
\medskip

All in all, \quad
$f(\zeta) \leq
3^{2n+2}
f(\xi)
\leq
3^{2N+2}
f(\xi)$.

\pause
\medskip

$N$ only depends on $K$, not on $\zeta$, $\xi$, or $f$.
As $\zeta$ and $\xi$ were arbitrary, the theorem follows. \qed

\medskip
\pause

\textbf{Remark:} We got an explicit (if not optimal) $C$.
\end{frame}

\begin{frame}
\textbf{Exercise:}
Show by example that Harnack's general inequality need not hold if $U$
is not assumed to be connected.

\medskip
\pause

\textbf{Exercise:}
Find the following counterexample of Harnack's inequality
if $f$ is not assumed to be
nonnegative.  For every $M > 0$ find
a harmonic function $f \colon \D \to \R$ such that $f(0) = 1$ and
$f(\nicefrac{1}{2}) \geq M$.

\medskip
\pause

\textbf{Exercise:}
Use Harnack's inequality to prove Liouville's theorem
for harmonic functions:

If $f \colon \C \to \R$ is harmonic
and nonnegative, then $f$ is constant.
\end{frame}

\begin{frame}

\begin{theorem}[Harnack's principle]
Let $U \subset \C$ be a domain and $\{ f_n \}$ a sequence of
harmonic functions on $U$ such that $f_1 \leq f_2 \leq f_3 \leq \cdots$.
Then either $f_n \to +\infty$ uniformly on compact subsets, or
$f_n \to f$ for a harmonic $f \colon U \to \R$ uniformly on compact subsets.
\end{theorem}

\pause

\textbf{Proof:}
WLOG assume $f_n \geq 0$ for all $n$.  (Otherwise apply to 
$f_n-f_1$).

\pause

By the monotonicity, $\{f_n\}$ converges pointwise (possibly to $+\infty$).

\pause

If $\lim f_n(p) = +\infty$ for some $p$,
\pause
let $K \subset U$ be any compact and let $K' = K \cup \{ p \}$.

\medskip
\pause
Harnack's inequality \wthus
$\displaystyle f_n(p) \leq \sup_{z \in K'} f_n(z) \leq C \inf_{z\in K'} f_n(z) \leq C \inf_{z\in K} f_n(z)$.

\medskip
\pause
\thus \quad
$f_n(z) \to +\infty$ uniformly for $z \in K$.

\medskip
\pause

Suppose $f(z) = \lim f_n(p) < +\infty$ for every $z \in U$.

\pause
Let $K \subset U$ be compact, take the $C$ from Harnack's, and take any $p \in K$.

\pause

Given $\epsilon > 0$, suppose $m > n$ are 
such that $f_m(p)-f_n(p) < \nicefrac{\epsilon}{C}$,
\pause
then
\[
\sup_{z \in K} \bigl( f_m(z)- f_n(z) \bigr)
\leq
C \inf_{z\in K} \bigl( f_m(z)-f_n(z) \bigr)
\pause
\leq
C \bigl( f_m(p)-f_n(p) \bigr)
< \epsilon .
\]
\pause
\thus \quad
$\{ f_n \}$ is uniformly Cauchy on $K$
\pause
\wthus
converges uniformly.
\pause

$f$ is harmonic by Harnack's first theorem.
\qed
\end{frame}

\end{document}
