\documentclass[10pt,aspectratio=169]{beamer}

% All the boilerplate is in ccaslides.sty
% Note that this also pulls in a custom vogtwidebar.sty
\usepackage{ccaslides}

\author{Ji\v{r}\'i Lebl}

\institute[OSU]{%
Departemento pri Matematiko de Oklahoma {\^S}tata Universitato}

\title{Cultivating Complex Analysis:\\%
Holomorphic functions are analytic (3.3.1)}

\date{}

\begin{document}

\begin{frame}
\titlepage
\end{frame}

\begin{frame}
A consequence of Cauchy's formula is
that holomorphic functions are analytic.

\medskip
\pause

We already proved
that analytic functions are holomorphic, let's prove the converse.

\pause

\begin{theorem} \label{thm:holpower}
Let $U \subset \C$ be open, $f \colon U \to \C$
holomorphic, $p \in U$, and $\Delta_R(p) \subset U$.
\pause
Then there exists a power series $\sum c_n {(z-p)}^n$
such that for all $z \in \Delta_R(p)$,
\begin{equation*}
f(z) = \sum_{n=0}^\infty c_n {(z-p)}^n .
\end{equation*}
\pause
Moreover,
\begin{equation*}
c_n = 
\frac{1}{2\pi i}
\int_{\gamma}
\frac{f(z)}{{(z-p)}^{n+1}}
\,
dz  ,
\end{equation*}
where $\gamma$ is any circle of radius $r$, $0 < r < R$, centered at
$p$ oriented counterclockwise.
\end{theorem}

\end{frame}

\begin{frame}

\textbf{Proof:}
Fix an $r$ such that $0 < r < R$.
\pause
Thus $\overline{\Delta_r(p)} \subset U$.

\pause
\medskip

Fix a $z \in \Delta_r(p)$.
\pause
For $\zeta \in \partial \Delta_r(p)$, 
\[
\abs{\frac{z-p}{\zeta-p}} =
\frac{\sabs{z-p}}{r} < 1 .
\]

\pause

So
\quad
$\displaystyle
\sum_{n=0}^\infty
{\left(\frac{z-p}{\zeta-p}\right)}^n
=
\frac{1}{1-\frac{z-p}{\zeta-p}}
=
\frac{\zeta-p}{\zeta-z}
$
\quad
converges uniformly absolutely for $\zeta \in \partial \Delta_r(p)$.

\pause
\medskip

$\displaystyle
f(z)
=
\frac{1}{2\pi i}
\int_{\partial \Delta_r(p)}
\frac{f(\zeta)}{\zeta-z}
\,
d \zeta 
\pause
=
\frac{1}{2\pi i}
\int_{\partial \Delta_r(p)}
\frac{f(\zeta)}{\zeta-p}
\frac{\zeta-p}{\zeta-z}
\,
d \zeta 
$

\medskip
\pause

\hspace*{0.5in}$\displaystyle
=
\frac{1}{2\pi i}
\int_{\partial \Delta_r(p)}
\frac{f(\zeta)}{\zeta-p}
\sum_{n=0}^\infty
{\left(\frac{z-p}{\zeta-p}\right)}^n
\,
d \zeta 
\pause
=
\sum_{n=0}^\infty
\underbrace{
\left(
\frac{1}{2\pi i}
\int_{\partial \Delta_r(p)}
\frac{f(\zeta)}{{(\zeta-p)}^{n+1}}
\,
d \zeta 
\right)
}_{c_n}
{(z-p)}^n .
$

The last equality held because the sum converges uniformly in 
$\zeta \in \partial \Delta_r(p)$

(we'll justify that on the next slide).

\end{frame}

\begin{frame}
We claim
\quad
$\displaystyle \sum_{n=0}^\infty
\frac{f(\zeta)}{\zeta-p}
{\left(\frac{z-p}{\zeta-p}\right)}^n$
\quad
converges uniformly in $\zeta \in \partial \Delta_r(p)$.

\pause
\medskip

$z$ is fixed and if $M$ is the supremum of $\abs{\frac{f(\zeta)}{\zeta-p}} =
\frac{\sabs{f(\zeta)}}{r}$ on $\partial \Delta_r(p)$ (a compact set),
\pause
then
\begin{equation*}
\abs{
\frac{f(\zeta)}{\zeta-p}
{\left(\frac{z-p}{\zeta-p}\right)}^n
}
\leq
M 
{\left(\frac{\abs{z-p}}{r}\right)}^n,
\qquad \text{and} \qquad
\frac{\abs{z-p}}{r} < 1 .
\end{equation*}
\pause
Thus, $\sum 
\abs{
\frac{f(\zeta)}{\zeta-p}
{\left(\frac{z-p}{\zeta-p}\right)}^n
}$ converges uniformly in $\zeta \in \partial \Delta_r(p)$,
\pause
\medskip

and so
$\sum 
\frac{f(\zeta)}{\zeta-p}
{\left(\frac{z-p}{\zeta-p}\right)}^n$ converges uniformly absolutely
(and hence uniformly).

\medskip
\pause

We found a power series converging to $f(z)$ for all $z \in \Delta_r(p)$.

\pause
\medskip

By uniqueness of the power series, the $c_n$ is independent of $r$.

\medskip
\pause

We get the same series for every $r$ and it converges in $\Delta_R(p)$.
\qed

\end{frame}

\begin{frame}

The key is writing the \emph{Cauchy kernel}
as
\[
\frac{1}{\zeta-z}
=
\frac{1}{\zeta-p}
\frac{\zeta-p}{\zeta-z}
\]
\pause
and using the geometric series.

\medskip

A common technique: the integral of a function against a kernel
gains properties of the kernel.

\pause
\medskip

Besides analyticity,
we also proved 

that the radius of convergence is

at least $R$, where $R$ is the

maximum $R$ such that $\Delta_R(p) \subset U$.

\vspace*{-0.85in}

\hspace*{2.5in}%
\subimport*{../figures/}{largestr.pdf_t}

\pause

\vspace*{-0.30in}
Note that the radius of convergence

gives bounds on the derivatives!

\pause

So we know about
the size of the derivatives at $p$ just from knowing how far away from $p$
is
$f$ still holomorphic.

\medskip
\pause

\textbf{Remark:}
Nothing like this is true for real-analytic functions such as $\varphi(x) =
\frac{1}{1+x^2}$ whose radius of convergence at $x=0$ is $1$, but $\varphi \colon
\R \to \R$ is (real) analytic everywhere.

\end{frame}

\begin{frame}
Let us restate the main conclusion:

\begin{corollary}
Let $U \subset \C$ be an open set.  A function $f \colon U \to \C$
is holomorphic if and only if $f$ is analytic.
\end{corollary}

\medskip
\pause

Results we proved for analytic functions are true for holomorphic
functions and vice versa.

\medskip
\pause

E.g., it is easy to show 
that the composition of
holomorphic functions is holomorphic via the chain rule.

\pause
\medskip

It is much harder to show this for analytic functions
by directly manipulating power series.

\pause
\medskip

But we don't have to, analytic functions are holomorphic.

\pause
\medskip

We have also finally proved the following:

\emph{A convergent power series defines an analytic function.}
\end{frame}

\end{document}
