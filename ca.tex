\documentclass[12pt,openany]{book}

%FIXME: for PRINT run for lulu or createspace, search for %PRINT

%\usepackage{pdf14}

\usepackage[shortlabels]{enumitem}
\usepackage{ifpdf}
\usepackage{amsmath}
\usepackage{amsfonts}
\usepackage{amssymb}
\usepackage{amsthm}
\usepackage{graphicx}
\usepackage{color}
\usepackage{chngcntr}

% Remove page numbers on chapter opens
% must come before fullpage
\usepackage{nopageno}

\usepackage[headings]{fullpage}

% smaller margins top/bottom margins
\addtolength{\textheight}{0.8in}
\addtolength{\topmargin}{-0.3in}

%PRINT
%First offset even/odd pages a bit
% not for the coil bound full letter size one
%\addtolength{\oddsidemargin}{0.2in}
%\addtolength{\evensidemargin}{-0.2in}

%PRINT
%Now cut page size a bit.  I'll run it through ghostcript anyway
%to convert to the right size, but this is good for crown quatro
%conversion, don't use for the full letter size versions
%\addtolength{\paperwidth}{-0.3in}
%\addtolength{\paperheight}{-1in}
%\addtolength{\topmargin}{-0.5in}
%\addtolength{\oddsidemargin}{-0.15in}
%\addtolength{\evensidemargin}{-0.15in}


\usepackage{url}
\usepackage{varioref}
\usepackage{imakeidx}
\PassOptionsToPackage{hyphens}{url}
\usepackage{hyperref} % do NOT set [ocgcolorlinks] here!

%If you have an older tex installation you might need
%to comment out the next line:
%PRINT (COMMENT OUT FOR PRINT)
\usepackage[ocgcolorlinks]{ocgx2} %perhaps run without for lulu/createspace

\usepackage[all]{hypcap}
\usepackage[shortalphabetic,msc-links]{amsrefs}
\usepackage{nicefrac}
\usepackage{mathdots}
\usepackage{mathtools}
\usepackage{microtype}
\usepackage{cancel}
\usepackage{framed}
\usepackage{import}

%\usepackage{draftwatermark}
%\SetWatermarkText{Draft of v3.0 as of \today. May change substantially!}
%\SetWatermarkAngle{90}
%\SetWatermarkHorCenter{0.5in}
%\SetWatermarkColor[gray]{0.7}
%\SetWatermarkScale{0.18}


\usepackage{tikz}
\usetikzlibrary{cd}
\usepackage{rotating}

\usepackage{cellspace}
\usepackage[toc,nopostdot,sort=use,nomain,automake]{glossaries}

%Palatino
\usepackage[theoremfont]{newpxtext}
\usepackage[vvarbb]{newpxmath}
\linespread{1.05}
\usepackage[scr=boondoxo]{mathalfa} % but we want the nice fancy script fonts

\usepackage[T1]{fontenc}

% Footnotes should use symbols, not numbers.  Numbered footnotes are
% evil
\usepackage[perpage,symbol*]{footmisc}

% useful
\newcommand{\ignore}[1]{}

% analysis/geometry stuff
\newcommand{\ann}{\operatorname{ann}}
\renewcommand{\Re}{\operatorname{Re}}
\renewcommand{\Im}{\operatorname{Im}}
\newcommand{\Orb}{\operatorname{Orb}}
\newcommand{\hol}{\operatorname{hol}}
\newcommand{\aut}{\operatorname{aut}}
\newcommand{\Aut}{\operatorname{Aut}}
\newcommand{\codim}{\operatorname{codim}}
\newcommand{\sing}{\operatorname{sing}}
\newcommand{\ord}{\operatorname{ord}}

% reals
\newcommand{\esssup}{\operatorname{ess~sup}}
\newcommand{\essran}{\operatorname{essran}}
\newcommand{\innprod}[2]{\langle #1 | #2 \rangle}
\newcommand{\linnprod}[2]{\langle #1 , #2 \rangle}
\newcommand{\blinnprod}[2]{\bigl\langle #1 , #2 \bigr\rangle}
\newcommand{\supp}{\operatorname{supp}}
\newcommand{\Nul}{\operatorname{Nul}}
\newcommand{\Ran}{\operatorname{Ran}}
\newcommand{\sabs}[1]{\lvert {#1} \rvert}
\newcommand{\snorm}[1]{\lVert {#1} \rVert}
\newcommand{\babs}[1]{\bigl\lvert {#1} \bigr\rvert}
\newcommand{\bnorm}[1]{\bigl\lVert {#1} \bigr\rVert}
\newcommand{\Babs}[1]{\Bigl\lvert {#1} \Bigr\rvert}
\newcommand{\Bnorm}[1]{\Bigl\lVert {#1} \Bigr\rVert}
\newcommand{\bbabs}[1]{\biggl\lvert {#1} \biggr\rvert}
\newcommand{\bbnorm}[1]{\biggl\lVert {#1} \biggr\rVert}
\newcommand{\BBabs}[1]{\Biggl\lvert {#1} \Biggr\rvert}
\newcommand{\BBnorm}[1]{\Biggl\lVert {#1} \Biggr\rVert}
\newcommand{\abs}[1]{\left\lvert {#1} \right\rvert}
\newcommand{\norm}[1]{\left\lVert {#1} \right\rVert}

% sets (some)
\newcommand{\C}{{\mathbb{C}}}
\newcommand{\R}{{\mathbb{R}}}
\newcommand{\Z}{{\mathbb{Z}}}
\newcommand{\N}{{\mathbb{N}}}
\newcommand{\Q}{{\mathbb{Q}}}
\newcommand{\D}{{\mathbb{D}}}
\newcommand{\F}{{\mathbb{F}}}

% consistent
\newcommand{\bB}{{\mathbb{B}}}
\newcommand{\bC}{{\mathbb{C}}}
\newcommand{\bR}{{\mathbb{R}}}
\newcommand{\bZ}{{\mathbb{Z}}}
\newcommand{\bN}{{\mathbb{N}}}
\newcommand{\bQ}{{\mathbb{Q}}}
\newcommand{\bD}{{\mathbb{D}}}
\newcommand{\bF}{{\mathbb{F}}}
\newcommand{\bH}{{\mathbb{H}}}
\newcommand{\bO}{{\mathbb{O}}}
\newcommand{\bP}{{\mathbb{P}}}
\newcommand{\bK}{{\mathbb{K}}}
\newcommand{\bV}{{\mathbb{V}}}
\newcommand{\CP}{{\mathbb{CP}}}
\newcommand{\RP}{{\mathbb{RP}}}
\newcommand{\HP}{{\mathbb{HP}}}
\newcommand{\OP}{{\mathbb{OP}}}
\newcommand{\sA}{{\mathscr{A}}}
\newcommand{\sB}{{\mathscr{B}}}
\newcommand{\sC}{{\mathscr{C}}}
\newcommand{\sF}{{\mathscr{F}}}
\newcommand{\sG}{{\mathscr{G}}}
\newcommand{\sH}{{\mathscr{H}}}
\newcommand{\sM}{{\mathscr{M}}}
\newcommand{\sO}{{\mathscr{O}}}
\newcommand{\sP}{{\mathscr{P}}}
\newcommand{\sQ}{{\mathscr{Q}}}
\newcommand{\sR}{{\mathscr{R}}}
\newcommand{\sS}{{\mathscr{S}}}
\newcommand{\sI}{{\mathscr{I}}}
\newcommand{\sL}{{\mathscr{L}}}
\newcommand{\sK}{{\mathscr{K}}}
\newcommand{\sU}{{\mathscr{U}}}
\newcommand{\sV}{{\mathscr{V}}}
\newcommand{\sX}{{\mathscr{X}}}
\newcommand{\sY}{{\mathscr{Y}}}
\newcommand{\sZ}{{\mathscr{Z}}}
\newcommand{\fS}{{\mathfrak{S}}}

\newcommand{\interior}{\operatorname{int}}

% Topo stuff
\newcommand{\id}{\textit{id}}
\newcommand{\im}{\operatorname{im}}
\newcommand{\rank}{\operatorname{rank}}
\newcommand{\Tor}{\operatorname{Tor}}
\newcommand{\Torsion}{\operatorname{Torsion}}
\newcommand{\Ext}{\operatorname{Ext}}
\newcommand{\Hom}{\operatorname{Hom}}

%extra thingies
%\newcommand{\mapsfrom}{\ensuremath{\text{\reflectbox{$\mapsto$}}}}
\newcommand{\from}{\ensuremath{\leftarrow}}
\newcommand{\dhat}[1]{\hat{\hat{#1}}}

\definecolor{mypersianblue}{rgb}{0.11, 0.22, 0.73}

\hypersetup{
    pdfborderstyle={/S/U/W 0.5}, %this just in case ocg isn't there
    %PRINT (for print use the below and comment out the above):
    %pdfborder={0 0 0},
    citecolor=mypersianblue,
    filecolor=mypersianblue,
    linkcolor=mypersianblue,
    urlcolor=mypersianblue,
    pdftitle={FIXME },
    pdfsubject={FIXME},
    pdfkeywords={one complex variable, complex analysis},
    pdfauthor={Jiri Lebl}
}

% Set up our index
\makeindex

% Very simple indexing
\newcommand{\myindex}[1]{#1\index{#1}}

% define this to be empty to kill notes
\newcommand{\sectionnotes}[1]{\noindent \emph{Note: #1} \medskip \par}

\author{Ji\v{r}\'i Lebl}

\title{One Complex Variable}

% Don't include subsections
\setcounter{tocdepth}{1}

\theoremstyle{plain}
\newtheorem{thm}{Theorem}[section]
\newtheorem{lemma}[thm]{Lemma}
\newtheorem{prop}[thm]{Proposition}
\newtheorem{cor}[thm]{Corollary}
\newtheorem{claim}[thm]{Claim}

\theoremstyle{remark}
\newtheorem{remark}[thm]{Remark}

\theoremstyle{definition}
\newtheorem{defn}[thm]{Definition}

\newtheoremstyle{exercise}% name
  {}% Space above
  {}% Space below
  {\itshape}% Body font
  {}% Indent amount 1
  {\bfseries \itshape}% Theorem head font
  {:}% Punctuation after theorem head
  {.5em}% Space after theorem head 2
  {}% Theorem head spec (can be left empty, meaning "normal")

\newenvironment{exbox}{%
    \def\FrameCommand{\vrule width 1pt \relax\hspace {10pt}}%
    \MakeFramed {\advance \hsize -\width \FrameRestore }%
}{%
    \endMakeFramed
}

\newenvironment{exparts}{%
    \leavevmode\begin{enumerate}[a),noitemsep,topsep=0pt,parsep=0pt,partopsep=0pt]
}{%
    \end{enumerate}
}
\newenvironment{exnumparts}{%
    \leavevmode\begin{enumerate}[1),noitemsep,topsep=0pt,parsep=0pt,partopsep=0pt]
}{%
    \end{enumerate}
}

\newenvironment{myfig}{%
    \begin{center}
}{%
    \end{center}
}

\theoremstyle{exercise}
\newtheorem{exercise}{Exercise}[section]

\newtheoremstyle{example}% name
  {}% Space above
  {}% Space below
  {}% Body font
  {}% Indent amount 1
  {\bfseries}% Theorem head font
  {:}% Punctuation after theorem head
  {.5em}% Space after theorem head 2
  {}% Theorem head spec (can be left empty, meaning "normal")

\theoremstyle{example}
\newtheorem{example}[thm]{Example}

% referencing
\newcommand{\figureref}[1]{\hyperref[#1]{Figure~\ref*{#1}}}
\newcommand{\tableref}[1]{\hyperref[#1]{Table~\ref*{#1}}}
\newcommand{\chapterref}[1]{\hyperref[#1]{chapter~\ref*{#1}}}
\newcommand{\Chapterref}[1]{\hyperref[#1]{Chapter~\ref*{#1}}}
\newcommand{\appendixref}[1]{\hyperref[#1]{appendix~\ref*{#1}}}
\newcommand{\Appendixref}[1]{\hyperref[#1]{Appendix~\ref*{#1}}}
\newcommand{\sectionref}[1]{\hyperref[#1]{section~\ref*{#1}}}
\newcommand{\exerciseref}[1]{\hyperref[#1]{Exercise~\ref*{#1}}}
\newcommand{\exampleref}[1]{\hyperref[#1]{Example~\ref*{#1}}}
\newcommand{\thmref}[1]{\hyperref[#1]{Theorem~\ref*{#1}}}
\newcommand{\propref}[1]{\hyperref[#1]{Proposition~\ref*{#1}}}
\newcommand{\lemmaref}[1]{\hyperref[#1]{Lemma~\ref*{#1}}}
\newcommand{\corref}[1]{\hyperref[#1]{Corollary~\ref*{#1}}}
\newcommand{\defnref}[1]{\hyperref[#1]{Definition~\ref*{#1}}}

% List of Symbols/Notation
\newglossary[nlg]{notation}{not}{ntn}{List of Notation}

\loadglsentries{notations}
\makeglossaries

\begin{document}

\ifpdf
  \pdfbookmark{Title Page}{title}
\fi
\newlength{\centeroffset}
\setlength{\centeroffset}{-0.5\oddsidemargin}
\addtolength{\centeroffset}{0.5\evensidemargin}
%\addtolength{\textwidth}{-\centeroffset}
\thispagestyle{empty}
\vspace*{\stretch{1}}
\noindent\hspace*{\centeroffset}\makebox[0pt][l]{\begin{minipage}{\textwidth}
\flushright
{\Huge\bfseries \sffamily One Complex Variable }
\noindent\rule[-1ex]{\textwidth}{5pt}\\[2.5ex]
\hfill\emph{\Large \sffamily A witty subtitle goes here }
\end{minipage}}

\vspace{\stretch{1}}
\noindent\hspace*{\centeroffset}\makebox[0pt][l]{\begin{minipage}{\textwidth}
\flushright
{\bfseries 
%by
Ji{\v r}\'i Lebl\\[3ex]} 
\today
\\
(version 0.0)
\end{minipage}}

%\addtolength{\textwidth}{\centeroffset}
\vspace{\stretch{2}}


\pagebreak

\vspace*{\fill}

%\begin{small} 
\noindent
Typeset in \LaTeX.

\bigskip

\noindent
Copyright \copyright 2019 Ji{\v r}\'i Lebl

%PRINT
% not for the coil version
%\noindent
%ISBN 978-0-359-64225-0

\bigskip

%\begin{floatingfigure}{1.4in}
%\vspace{-0.05in}
\noindent
\includegraphics[width=1.38in]{figures/license}
\quad
\includegraphics[width=1.38in]{figures/license2}
%\end{floatingfigure}

\bigskip

\noindent
\textbf{License:}
\\
This work is dual licensed under
the Creative Commons
Attribution-Non\-commercial-Share Alike 4.0 International License and
the Creative Commons
Attribution-Share Alike 4.0 International License.
To view a
copy of these licenses, visit
\url{https://creativecommons.org/licenses/by-nc-sa/4.0/}
or
\url{https://creativecommons.org/licenses/by-sa/4.0/}
or send a letter to
Creative Commons
PO Box 1866, Mountain View, CA 94042, USA\@.
%Creative Commons, 171 Second Street, Suite 300, San Francisco, California,
%94105, USA.

\bigskip

\noindent
You can use, print, duplicate, share this book as much as you want.  You can
base your own notes on it and reuse parts if you keep the license the
same.  You can assume the license is either the CC-BY-NC-SA or CC-BY-SA\@,
whichever is compatible with what you wish to do, your derivative works must
use at least one of the licenses.

%\bigskip
%
%\noindent
%\textbf{Acknowledgments:}
%\\
%I would like to thank FIXME
%and students in my classes for pointing out typos/errors
%and helpful suggestions. 
%
%\bigskip
%
%\noindent
%During the writing of this book, 
%the author was in part supported by NSF grant DMS-1362337.

\bigskip

\noindent
\textbf{More information:}
\\
See \url{https://www.jirka.org/ca/} for more information
(including contact information).

\medskip

\noindent
The \LaTeX\ source for the book is available
for possible modification and customization
at github: \url{https://github.com/jirilebl/ca}


% For large print do this
%\large

\microtypesetup{protrusion=false}
\tableofcontents
\microtypesetup{protrusion=true}

%\addtocontents{toc}{\protect\vspace{-2\baselineskip}}
\addtocontents{toc}{\protect\vspace{-\baselineskip}}
%\addtocontents{toc}{\protect\enlargethispage{\baselineskip}}

%%%%%%%%%%%%%%%%%%%%%%%%%%%%%%%%%%%%%%%%%%%%%%%%%%%%%%%%%%%%%%%%%%%%%%%%%%%%%%
%%%%%%%%%%%%%%%%%%%%%%%%%%%%%%%%%%%%%%%%%%%%%%%%%%%%%%%%%%%%%%%%%%%%%%%%%%%%%%
%%%%%%%%%%%%%%%%%%%%%%%%%%%%%%%%%%%%%%%%%%%%%%%%%%%%%%%%%%%%%%%%%%%%%%%%%%%%%%

\chapter*{Introduction} \label{ch:intro}
\addcontentsline{toc}{chapter}{Introduction}
\markboth{INTRODUCTION}{INTRODUCTION}

%%%%%%%%%%%%%%%%%%%%%%%%%%%%%%%%%%%%%%%%%%%%%%%%%%%%%%%%%%%%%%%%%%%%%%%%%%%%%%

FIXME

%%%%%%%%%%%%%%%%%%%%%%%%%%%%%%%%%%%%%%%%%%%%%%%%%%%%%%%%%%%%%%%%%%%%%%%%%%%%%%

\section{FIXME} \label{sec:FIXME}

FIXME

%%%%%%%%%%%%%%%%%%%%%%%%%%%%%%%%%%%%%%%%%%%%%%%%%%%%%%%%%%%%%%%%%%%%%%%%%%%%%%
%%%%%%%%%%%%%%%%%%%%%%%%%%%%%%%%%%%%%%%%%%%%%%%%%%%%%%%%%%%%%%%%%%%%%%%%%%%%%%
%%%%%%%%%%%%%%%%%%%%%%%%%%%%%%%%%%%%%%%%%%%%%%%%%%%%%%%%%%%%%%%%%%%%%%%%%%%%%%

\chapter{FIXME} \label{ch:FIXME2}

%%%%%%%%%%%%%%%%%%%%%%%%%%%%%%%%%%%%%%%%%%%%%%%%%%%%%%%%%%%%%%%%%%%%%%%%%%%%%%

\section{FIXME} \label{sec:FIXME2}

FIXME

%%%%%%%%%%%%%%%%%%%%%%%%%%%%%%%%%%%%%%%%%%%%%%%%%%%%%%%%%%%%%%%%%%%%%%%%%%%%%%
%%%%%%%%%%%%%%%%%%%%%%%%%%%%%%%%%%%%%%%%%%%%%%%%%%%%%%%%%%%%%%%%%%%%%%%%%%%%%%
%%%%%%%%%%%%%%%%%%%%%%%%%%%%%%%%%%%%%%%%%%%%%%%%%%%%%%%%%%%%%%%%%%%%%%%%%%%%%%

\appendix

% No sections in appendixes
\counterwithin{thm}{chapter}

%%%%%%%%%%%%%%%%%%%%%%%%%%%%%%%%%%%%%%%%%%%%%%%%%%%%%%%%%%%%%%%%%%%%%%%%%%%%%%
%%%%%%%%%%%%%%%%%%%%%%%%%%%%%%%%%%%%%%%%%%%%%%%%%%%%%%%%%%%%%%%%%%%%%%%%%%%%%%
%%%%%%%%%%%%%%%%%%%%%%%%%%%%%%%%%%%%%%%%%%%%%%%%%%%%%%%%%%%%%%%%%%%%%%%%%%%%%%

\chapter{Basic notation and terminology} \label{ap:basicnotation}

%%%%%%%%%%%%%%%%%%%%%%%%%%%%%%%%%%%%%%%%%%%%%%%%%%%%%%%%%%%%%%%%%%%%%%%%%%%%%%

% Is this overly too nitpicky to include?

FIXME

%%%%%%%%%%%%%%%%%%%%%%%%%%%%%%%%%%%%%%%%%%%%%%%%%%%%%%%%%%%%%%%%%%%%%%%%%%%%%%

%%%%%%%%%%%%%%%%%%%%%%%%%%%%%%%%%%%%%%%%%%%%%%%%%%%%%%%%%%%%%%%%%%%%%%%%%%%%%%
%%%%%%%%%%%%%%%%%%%%%%%%%%%%%%%%%%%%%%%%%%%%%%%%%%%%%%%%%%%%%%%%%%%%%%%%%%%%%%
%%%%%%%%%%%%%%%%%%%%%%%%%%%%%%%%%%%%%%%%%%%%%%%%%%%%%%%%%%%%%%%%%%%%%%%%%%%%%%

%FIXME: else I don't get links, weird
%\def\MR#1{\relax\ifhmode\unskip\spacefactor3000 \space\fi%
  %\href{http://www.ams.org/mathscinet-getitem?mr=#1}{MR#1}}
\def\myDOI#1{\href{http://dx.doi.org/#1}{#1}}



%FIXME
%\cleardoublepage  
\clearpage
\phantomsection
\addcontentsline{toc}{chapter}{Further reading}
\markboth{FURTHER READING}{FURTHER READING}
\begin{bibchapter}[Further Reading] \label{ch:furtherreading}

%Here we list useful books for extra reading.

\begin{biblist}[\normalsize]

%FIXME: also cite the ra book

\bib{Rudin:principles}{book}{
   author={Rudin, Walter},
   title={Principles of mathematical analysis},
   edition={3},
   note={International Series in Pure and Applied Mathematics},
   publisher={McGraw-Hill Book Co., New York-Auckland-D\"usseldorf},
   date={1976},
   pages={x+342},
   review={\MR{0385023}},
}

\end{biblist}
\end{bibchapter}

%%%%%%%%%%%%%%%%%%%%%%%%%%%%%%%%%%%%%%%%%%%%%%%%%%%%%%%%%%%%%%%%%%%%%%%%%%%%%%
%%%%%%%%%%%%%%%%%%%%%%%%%%%%%%%%%%%%%%%%%%%%%%%%%%%%%%%%%%%%%%%%%%%%%%%%%%%%%%
%%%%%%%%%%%%%%%%%%%%%%%%%%%%%%%%%%%%%%%%%%%%%%%%%%%%%%%%%%%%%%%%%%%%%%%%%%%%%%

%\cleardoublepage  
\clearpage  
\phantomsection
\addcontentsline{toc}{chapter}{\indexname}  
\microtypesetup{protrusion=false}
\printindex
\microtypesetup{protrusion=true}

%%%%%%%%%%%%%%%%%%%%%%%%%%%%%%%%%%%%%%%%%%%%%%%%%%%%%%%%%%%%%%%%%%%%%%%%%%%%%%
%%%%%%%%%%%%%%%%%%%%%%%%%%%%%%%%%%%%%%%%%%%%%%%%%%%%%%%%%%%%%%%%%%%%%%%%%%%%%%
%%%%%%%%%%%%%%%%%%%%%%%%%%%%%%%%%%%%%%%%%%%%%%%%%%%%%%%%%%%%%%%%%%%%%%%%%%%%%%

%
% automake on glossaries doesn't work if the index is before the glossary.
% That's why the List of Notation is last, no other reason.  Problem is
% that printindex does a clearpage which screws up the delayed write18
% that glossaries sets up
%

\begingroup
\renewcommand{\pagelistname}{Page}
\setglossarystyle{long3colheader}
% correctly set up with cellspace
\renewenvironment{theglossary}%
  {\setlength\cellspacetoplimit{4pt}
   \setlength\cellspacebottomlimit{4pt}
   \setlength\LTleft{0pt}
   \setlength\LTright{0pt}
   \markboth{LIST OF NOTATION}{LIST OF NOTATION}
   \begin{longtable}{Sl @{\extracolsep{\fill}} Sl @{\extracolsep{\fill}} Sl}}%
  {\end{longtable}}%
\cleardoublepage
\microtypesetup{protrusion=false}
\printglossary[type=notation] 
\microtypesetup{protrusion=true}
\endgroup

\end{document}
